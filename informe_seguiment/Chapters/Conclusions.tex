% Chapter Template

\chapter{Conclusions} % Main chapter title

\label{Conclusions} % Change X to a consecutive number; for referencing this chapter elsewhere, use \ref{ChapterX}

Amb la informació presentada anteriorment, podem arribar a les següents conclusions que garanteixen la viabilitat de finalització del projecte:

\begin{itemize}
\item \textbf{Definició i contextualització del sistema}. L'aprofundiment en la temàtica i la recerca i contextualització de l'àmbit d'estudi garanteixen un punt de partida fort per realitzar un treball vàlid i contrastat sota un marc de desenvolupament específic.
\item \textbf{Seguiment acurat de la planificació}. Tal i com s'ha exposat prèviament, no s'han produït modificacions que hagin causat desviacions crítiques en el desenvolupament del projecte. Les poques desviacions viscudes han estat tractades i en cap cas han compromés la integritat del desenvolupament, sinó que han permet una millora o simplement un canvi en els objectius i els resultats finals.
\item \textbf{Integració de coneixements}. Com a Treball de Final de Grau amb l'objectiu d'integrar els coneixements estudiats durant el curs del mateix, aquest document valida breument els conceptes treballats per tal de garantir no només que es tracta d'un projecte amb potencial propi del grau, sinó també de l'especialitat sota la qual es desenvolupa.
\item \textbf{Integració de lleis i regulacions}. Com a part fonamental de la gestió d'un projecte, es presenten els aspectes relacionats amb lleis i regulacions que poden afectar al desenvolupament (en aquest cas, bàsicament la gestió de llicències del codi).
\end{itemize}

En base a aquests punts, podem garantir que el projecte segueix una via satisfactòria, que garanteix els objectius principals del projecte, i que la seva realització portarà a una cloenda completa dels coneixements i l'experiència obtinguda durant el curs del Grau en Enginyeria Informàtica.
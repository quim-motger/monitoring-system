% Chapter Template

\chapter{Gestió del desenvolupament i alternatives} % Main chapter title

\label{GestioIDesenvolupament} % Change X to a consecutive number; for referencing this chapter elsewhere, use \ref{ChapterX}

A continuació, procedim a incloure els detalls relacionats amb la planificació i metodologia de desenvolupament.

%----------------------------------------------------------------------------------------
%	SECTION 1
%----------------------------------------------------------------------------------------

\section{Metodologia de desenvolupament}

Tal i com es va decidir inicialment, el treball s'està desenvolupament mitjançant una simplificació de la metodologia de desenvolupament àgil \textbf{Kanban}. D'aquesta manera, s'ha respectat el desenvolupament de les diferents tasques del projecte en 3 fases:

\begin{enumerate}
\item \textbf{Identificació de tasques.} En base als requisits inicials del projecte, s'han anat afegint les diferents tasques a realitzar mitjançant l'ús de l'eina \textbf{Trello} amb el nivell de detall necessari per considerar-les en una fase prèvia a la seva implementació (\textit{To-Do}).
\item \textbf{Implementació.} Conforme disponibilitat, s'ha anat realitzant la implementació de les diferents tasques en paral·lel (sempre amb un màxim de 2-3 tasques alhora, garantint així l'efectivitat de la feina realitzada). En aquest estat (\textit{Doing}), es realitza la feina corresponent a la realització de la tasca segons la seva naturalesa (disseny, implementació...).
\item \textbf{Validació.} Un cop s'han complert cadascuna de les tasques (\textit{Done}) s'ha acceptat i validat d'acord amb els requisits considerats.
\end{enumerate}


\section{Planificació temporal}

Segons les necessitats d’aquest projecte i les seves característiques, es va classificar el seu desenvolupament en 4 fases: la planificació, el disseny, la implementació, i la documentació i entrega del projecte. A continuació, procedim a detallar l'estat actual del projecte i desviacions viscudes.\\

\textbf{Planificació}\\

Gràcies al curs de GEP es va arribar a completar un percentatge elevat d'aquesta fase, de manera que l'arrencada i desenvolupament del projecte va ser més factible i fluïda. Actualment aquesta fase ja es troba totalment acabada. Respecte a les tasques inicials no hi ha hagut desviacions destacables.\\

\textbf{Disseny}\\

Totes les tasques de configuració, disseny del sistema, estudi d'arquitectura, i primeres documentacions de memòria i components a desenvolupar s'han complert satisfactòriament. En aquesta fase, si bé no s'han viscut modificacions pròpies de les activitats, si que s'han produït algunes alteracions en quant al moment del desenvolupament. En alguns casos, aspectes de disseny i implementació d'alguns dels components han anat més lligat de la mà, seguint estils més propis de les metodologies àgils, ja que s'ha preferit anar dissenyant i implementant de forma paral·lela a mesura que le desenvolupament i la recerca avançava.\\

Tot així, aquestes desviacions no han suposat problemes en el desenvolupament del projecte, ja que tal i com es plantejava a la planificació inicial, la gestió i quantització de la dedicació d'algunes d'aquestes tasques no eren dependents entre elles, i per tant s'ha pogut garantir la continuitat de desenvolupament.\\

\textbf{Implementació}\\

En general s'han finalitzat les tasques d'\textbf{implementació} dels components principals del projecte. Actualment, s'estan aplicant darreres correccions i validacions d'alguns dels components, però el sistema ja disposa d'un nivell de maduresa suficient com per poder realitzar proves i validacions.\\

L'únic punt que encara es troba en les primeres fases de desenvolupament és la \textbf{implementació del dashboard}. Tot i així, això no suposa un problema, ja que per tal de garantir l'èxit del desenvolupament del projecte és va establir que el nivell de detall i funcionalitats d'aquest seria proporcional al temps disponible per dedicar-hi temps i recursos, al tractar-se d'una eina més aviat complementària i orientada a la presentació que no pas al funcionament i satisfacció dels requisits del sistema.\\

En qualsevol cas, desviacions pròpies d'aspectes com p.e. nombre d'hores o quantitat de recursos a destinar no s'han produït.\\

\textbf{Fase final}\\

Tot i que pròpiament no s'han realitzat tasques específiques d'aquesta fase (de fet, si recordem la planificació inicial, aquesta estava planejada per la 1a setmana de juny), això ja s'escau a la planificació del projecte. De fet, en aquest sentit s'han produït possibles desviacions a favor del temps de desenvolupament, ja que tasques com la \textbf{validació de components} (testing) o la \textbf{documentació de la memòria} ja s'han començat i, per tant, no requeriran una dedicació tan elevada en aquesta fase.

\section{Previsió d'alternatives i pla d'acció}

Considerem, a continuació, les previsions d'alternatives que es va realitzar durant el curs de GEP i, si s'escau, els detalls d'aquests desviaments.

\begin{itemize}
\item \textbf{Increment del nº d’hores necessàries.} Fins al punt de desenvolupament actual, no s'ha produït un increment del nº d'hores (més enllà de petites desviacions que s'han anat autoregulant).
\subitem \textbf{Pla d’acció.} \textit{No aplicat}
\item \textbf{Avaria en el hardware.} No hi ha hagut cap imprevist relacionat amb avaries del hardware utilitzat.
\subitem \textbf{Pla d’acció.} \textit{No aplicat}
\item \textbf{Canvis en els requisits del projecte.} En aquest sentit sí que s'han produït algunes modificacions. En general, s'ha realitzat una \textbf{modificació de prioritats} en els requisits i s'ha afegit una nova part dins l'àmbit de desenvolupament del projecte, corresponent a l'adaptació dels monitors i la gestió d'un sistema d'adaptabilitat automàtic mitjançant l'ús de diagrames UML. 
\subitem \textbf{Pla d’acció.} Tot i que ha suposat un canvi en els requisits, no s'han produït modificacions en la càrrega de feina ja que simplement s'ha fet un plantejament diferent dels requisits i necessitats del nostre projecte. Al prendre aquesta decisió amb la suficient antel·lació, no ha suposat un problema de gestió del projecte.
\end{itemize}
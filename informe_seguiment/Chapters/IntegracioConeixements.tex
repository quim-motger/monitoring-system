% Chapter Template

\chapter{Integració de coneixements} % Main chapter title

\label{IntegracioConeixements} % Change X to a consecutive number; for referencing this chapter elsewhere, use \ref{ChapterX}

Un dels objectius del Treball de Final de Grau és constatar l'assimilació dels coneixements estudiants durant el Grau en Enginyeria Informàtica i, concretament en aquest cas, els conceptes propis de l'especialitat d'Enginyeria del software. Aquests conceptes engloben diverses disciplines, que el desenvolupament del projecte treballen, i a continuació s'exposen.\\

\textbf{Disseny de sistemes}\\

La solució proposada per aquest treball consisteix en un \textbf{sistema de monitoratge dotat d'autoadaptabilitat}. Aquesta premisa genèrica, tal i com es plasmarà a la memòria, engloba en realitat un sistema complex format per un conjunt de components independents i amb activitat pròpia que, integrats dins el nostre sistema i definint una interacció específica entre aquests, serveixen a un propòsit superior orientat a la gestió i adaptabilitat de sistemes de monitoratge (és a dir, sistemes encarregats del control de qualitat).\\

En definitiva, la solució plantejada i exposada presenta el \textbf{disseny, implementació i integració de components}, i per tant l'anàlisi i estudi de sistemes software complexos, un dels pilars de l'enginyeria del software.\\

%-----------------------------------
%	SUBSECTION 1
%-----------------------------------
\textbf{Disseny arquitectura software}\\

Cadascun dels components que integren el sistema ha de satisfer uns requisits des d'un punt de vista d'arquitectura del software. Per exemple, la implementació dels monitors ha estat basada en una proposta d'arquitectura genèrica que dota al sistema de l'extensibilitat i heterogeneïtat que volíem assolir.\\

Per aquest motiu, durant el disseny dels components s'han aplicat coneixements relacionats amb el disseny software i l'ús de patrons arquitectònics, conceptes treballats al llarg de l'especialitat cursada.\\

%-----------------------------------
%	SUBSECTION 2
%-----------------------------------

\textbf{Gestió i desenvolupament de projecte}\\

De forma explícita, el desenvolupament d'un Treball de Fi de Grau inclou aspectes relacionats amb la gestió d'un projecte, coneixements també treballats durant el curs de l'especialitat d'enginyeria del software. La priorització de requisits, la transició de fases, etc., s'han aplicat des d'un punt de vista teòric i pràctic.\\

\pagebreak

\textbf{Disseny i implementació de serveis RESTful}\\

Els components del sistema s'exposen com a serveis RESTful per tal de permetre'n una integració i dotar al sistema de la característica de distribució que volíem assolir. Per fer-ho, coneixements d'arquitectura de serveis web s'han hagut d'aplicar per dissenyar i implementar aquests components.\\

\textbf{Disseny i implementació de bases de dades}\\

De forma més genèrica i superficial que els altres conceptes, també ha calgut dissenyar i implementar bases de dades per gestionar el sistema, tot i que la complexitat i els esforços dedicats a aquesta part han estat menors, ja que l'únic propòsit ha estat satisfer la necessitat de persistència de models UML i les dades de monitoratge.\\

\textbf{Altres coneixements d'implementació}\\

Per suposat, tots aquells conceptes propis de les capacitats com a \textit{software developer} (coneixements en programació amb Java, ús de gestor de versions, ús de gestió de dependències, etc.) han estat necessaris.

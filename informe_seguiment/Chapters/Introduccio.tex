% Chapter Template

\chapter{Introducció} % Main chapter title

\label{Introduccio} % Change X to a consecutive number; for referencing this chapter elsewhere, use \ref{ChapterX}

El present document consisteix en l'informe de seguiment del Treball de Final de Grau (TFG) del Grau en Enginyeria Informàtica titulat \textit{Sistema de monitoratge autoadaptable, heterogeni i distribuït}. Com a document presentat durant el procés de desenvolupament del mateix projecte, els objectius principals d'aquest en són dos.\\

En primer lloc, establir una "fotografia" documentada del projecte: és a dir, projectar en quin estat es troba el projecte, quina és la feina que s'ha realitzat fins al moment, en quint punt del desenvolupament es troba, i quines són les tasques que, d'acord amb la contextualització i plantejaments presentats, queda per fer. D'aquesta manera, aquest document pretèn reflectir la realitat del projecte en el moment de la seva presentació.\\

En segon lloc, aquest document serveix de garantia pel desenvolupament i finalització del projecte. D'acord amb els terminis establerts, i a les tasques realitzades i per realitzar, els detalls del treball realitzat i les consideracions pertinents han de ser una prova de la viabilitat de presentació del projecte.\\

Per englobar aquests objectius, es presenten primerament la contextualització i detalls de l'entorn on desenvolupem el treball, així com la informació relacionada amb la gestió i planificació en relació a tot allò establert a GEP. En aquest sentit, s'inclouen consideracions com el sorgiment de desviacions i alternatives, i la integració dels coneixements i aspectes teòrics que han calgut considerar pel desenvolupament del projecte.
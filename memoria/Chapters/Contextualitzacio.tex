% Chapter Template

\chapter{Contextualització} % Main chapter title

\label{Contextualitzacio} % Change X to a consecutive number; for referencing this chapter elsewhere, use \ref{ChapterX}

%----------------------------------------------------------------------------------------
%	SECTION 1
%----------------------------------------------------------------------------------------

\section{Justificació de la temàtica}

És un fet innegable que en les darreres dècades els sistemes software han evolucionat fins al punt d’esdevenir elements clau i imprescindibles de les activitats primàries de qualsevol empresa, organització o institució. La gestió de la informació, els protocols i controls de seguretat, els processos de negoci, etc., són els reptes als quals els CIO de moltes empreses s’han d’enfrontar. Aquests reptes i els seus resultats depenen, en gran mesura, del comportament dels sistemes software que entren en joc dins aquestes activitats. 
El comportament d’aquests sistemes, per tant, és clau. És per aquest motiu que eventualment ha anat prenent força un concepte basat en l’estudi i control de qualitat dels sistemes software: el monitoratge.\\

Com a part de la vida professional d’un enginyer de software, la supervisió i control dels components i sistemes amb els què treballa és un concepte clau amb el qual, d’una forma o altra, ha d’estar familiaritzat. Però el problema que plantegem aquí va més enllà: després del repte de monitorar els sistemes, ens hem de plantejar com dissenyar, gestionar i adaptar aquest monitoratge.\\

En aquest àmbit, i tal com veurem més endavant a l’apartat \textit{2.3. Estat de l’art}, existeix una àmplia recerca que actualment treballa i desenvolupa projectes en relació a aquest àmbit. El potencial d’estudi que ofereix resulta d’un alt interès a causa de la possibilitat de recerca i síntesi i als diferents aspectes i criteris sobre els quals es pot treballar.\\

Així, tant com estudiant com a futur professional del sector de l’enginyeria del software, es poden contemplar diversos criteris per treballar en aquesta temàtica:
\begin{itemize}
\item Un aprofundiment en els coneixements de l’enginyeria i els sistemes software
\item Treball i recerca en conceptes de control de qualitat i fiabilitat
\item Possibilitat de col·laborar i aprofundir en un tema de recerca d’actualitat dins l’enginyeria de serveis i els sistemes d’informació
\item Plantejament d’un projecte complet que pugui servir a tercers interessats en l’estudi de sistemes de monitoratge autoadaptatius
\end{itemize}  


%-----------------------------------
%	SUBSECTION 1
%-----------------------------------
\section{Definició de l'àrea d'estudi}

Nunc posuere quam at lectus tristique eu ultrices augue venenatis. Vestibulum ante ipsum primis in faucibus orci luctus et ultrices posuere cubilia Curae; Aliquam erat volutpat. Vivamus sodales tortor eget quam adipiscing in vulputate ante ullamcorper. Sed eros ante, lacinia et sollicitudin et, aliquam sit amet augue. In hac habitasse platea dictumst.

%-----------------------------------
%	SUBSECTION 2
%-----------------------------------

\subsection{Identificació dels stakeholders}
Morbi rutrum odio eget arcu adipiscing sodales. Aenean et purus a est pulvinar pellentesque. Cras in elit neque, quis varius elit. Phasellus fringilla, nibh eu tempus venenatis, dolor elit posuere quam, quis adipiscing urna leo nec orci. Sed nec nulla auctor odio aliquet consequat. Ut nec nulla in ante ullamcorper aliquam at sed dolor. Phasellus fermentum magna in augue gravida cursus. Cras sed pretium lorem. Pellentesque eget ornare odio. Proin accumsan, massa viverra cursus pharetra, ipsum nisi lobortis velit, a malesuada dolor lorem eu neque.

%----------------------------------------------------------------------------------------
%	SECTION 2
%----------------------------------------------------------------------------------------

\section{Estat de l'art}

Sed ullamcorper quam eu nisl interdum at interdum enim egestas. Aliquam placerat justo sed lectus lobortis ut porta nisl porttitor. Vestibulum mi dolor, lacinia molestie gravida at, tempus vitae ligula. Donec eget quam sapien, in viverra eros. Donec pellentesque justo a massa fringilla non vestibulum metus vestibulum. Vestibulum in orci quis felis tempor lacinia. Vivamus ornare ultrices facilisis. Ut hendrerit volutpat vulputate. Morbi condimentum venenatis augue, id porta ipsum vulputate in. Curabitur luctus tempus justo. Vestibulum risus lectus, adipiscing nec condimentum quis, condimentum nec nisl. Aliquam dictum sagittis velit sed iaculis. Morbi tristique augue sit amet nulla pulvinar id facilisis ligula mollis. Nam elit libero, tincidunt ut aliquam at, molestie in quam. Aenean rhoncus vehicula hendrerit.

%-----------------------------------
%	SUBSECTION 2
%-----------------------------------

\subsection{Projecte SUPERSEDE}
Morbi rutrum odio eget arcu adipiscing sodales. Aenean et purus a est pulvinar pellentesque. Cras in elit neque, quis varius elit. Phasellus fringilla, nibh eu tempus venenatis, dolor elit posuere quam, quis adipiscing urna leo nec orci. Sed nec nulla auctor odio aliquet consequat. Ut nec nulla in ante ullamcorper aliquam at sed dolor. Phasellus fermentum magna in augue gravida cursus. Cras sed pretium lorem. Pellentesque eget ornare odio. Proin accumsan, massa viverra cursus pharetra, ipsum nisi lobortis velit, a malesuada dolor lorem eu neque.
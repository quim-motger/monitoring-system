% Chapter Template

\chapter{Entorn de desenvolupament} % Main chapter title

\label{EinesDesenvolupament} % Change X to a consecutive number; for referencing this chapter elsewhere, use \ref{ChapterX}

El desenvolupament del sistema proposat requereix la integració d'un conjunt de subcomponents independents que, tot i comunicar-se entre ells, presenten una sèrie de característiques tècniques variades. Els requisits de desenvolupament i els entorns sobre els quals aquests components s'han de desenvolupar dependran de la naturalesa i els objectius de cadascun d'aquests. I, en termes genèrics, la gestió del sistema requerirà l'ús d'eines que ens facilitin aquest comportament.\\

Com a punt de partida al desenvolupament i exposició dels diversos components i tecnologies utilitzades, plantegem les tecnologies i elements bàsics que formen part del desenvolupament del projecte per, a partir d'aquí i al llarg dels següents capítols, exposar les tecnologies (llibreries, frameworks, etc.) que integrarem a aquestes per assolir els objectius.

\section{Tecnologies utilitzades}

Procedim a identificar les tecnologies bàsiques, amb les seves versions corresponents:

\begin{itemize}
\item \textbf{Git / GitHub (http://github.com).} Com a software de control de gestions i repositori s'utilitza \textbf{git} i \textbf{GitHub}, respectivament. La completesa i maduresa de git en la seva actualitat, així com les funcionalitats oferides per GitHub i la comoditat de la seva gestió, les fan candidates ideals per a realitzar el desenvolupament del projecte.
\item \textbf{Java 8.} Pràcticament la totalitat del projecte i els seus components s'han implementat utilitzant llenguatge Java. Concretament, la darrera versió Java 8, degut a les millores i la correcció d'alguns bugs que suposa respecte la seva anterior versió, Java 7.
\item \textbf{Eclipse IDE for Java Developers (Neon 4.6.2).} IDE i versió corresponents utilitzats pel desenvolupament dels components. S'ha considerat com l'opció ideal per una banda, per facilitar la compatibilitat i integració amb el projecte SUPERSEDE i els altres components, i per altra banda per la senzillesa i la integració de diferents plug-ins i eines que faciliten el desenvolupament. 
\item \textbf{Eclipse Modeling Tools (Neon 4.6.2).}  IDE complementari al desenvolupament orientat al desenvolupament de projectes de creació i edició de models UML mitjançant l'ús de tecnologies associades que es detallaran més endavant. 
\item \textbf{Gradle 2.13.} Davant la necessitat de gestionar les dependències i la compilació dels diferents components, s'ha triat Gradle com a opció preferent.
\item \textbf{Spring Framework 4.3.9.} Framework popularment conegut orientat al desenvolupament d'aplicacions basades en Java. L'utilitzarem especialment orientat a:
\begin{itemize}
\item Disseny i implementació de serveis web RESTful
\item Desenvolupament d'aplicacions web senzilles
\item Gestió de persistència d'aplicacions
\end{itemize}
\item \textbf{UML2 5.0.0.} Llibreria que encapsula la implementació per la gestió dinàmica mitjançant la plataforma Eclipse de models UML. L'utilitzarem per la lectura i escriptura dels models del nostre sistema de monitoratge.
\item \textbf{Papyrus 2.0.2.} Framework que ofereix un entorn de modelació gràfic dels models UML, els seus components i les seves propietats. Especialment útil per la visualització de models i el seu tractament per facilitat la seva lectura i anàlisi.
\end{itemize}

\section{Implementació i artefactes generats}
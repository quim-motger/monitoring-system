% Chapter Template

\chapter{Introducció} % Main chapter title

\label{Introduccio} % Change X to a consecutive number; for referencing this chapter elsewhere, use \ref{ChapterX}

El present document consisteix en la memòria del Treball de Final de Grau (TFG) del Grau en Enginyeria Informàtica titulat \textit{Sistema de monitoratge autoadaptable, heterogeni i distribuït}. Com a projecte realitzat a la cloenda dels estudis de grau, el desenvolupament i presentació d'aquest projecte tenen dos objectius principals.\\

En primer lloc, la consolidació dels coneixements adquirits durant el transcurs del grau. Aquests coneixements engloben des de la qüestió tècnica i específica de la matèria, amb especial èmfasi en els conceptes i aprenentatges relacionats amb l'especialitat d'Enginyeria del Software (tals com el disseny de components software), fins a aspectes relacionats amb la gestió, realització i documentació de projectes complets, pràctics i funcionals, dels quals aquest TFG n'és un exemple. Al llarg dels capítols que composen aquesta memòria, la justificació, explicació i demostració de les tasques realitzades i els conceptes tractats s'exposen amb la rigurositat adequada a un document acadèmic d'aquesta categoria, demostrant l'assoliment d'aquests coneixements amb la major claredat possible.\\

Per altra banda, aquest projecte pretèn presentar-se com un treball d'investigació, recerca i desenvolupament amb valor propi, dins d'un àmbit i context determinats, amb un objectiu pràctic i aplicable. Més enllà del caire acadèmic, els productes i resultats generats com a conseqüència de la realització d'aquest projecte (components software, disseny i implementació de sistemes, documentació, etc.) esdevenen elements amb valor propi, amb expectatives d'ús  i possibilitats d'expansió dins del seu propi context.\\

Per tal de satisfer aquests dos objectius, aquest projecte presenta el següent propòsit: dissenyar, implementar, gestionar, testejar, validar i mantenir un sistema de monitoratge que satisfaci els criteris d'autoadaptabilitat, heterogeneïtat i distribució (conceptes que s'aprofundiran més endavant). Sota aquesta temàtica, i amb les consideracions prèviament establertes, s'assoliran tant l'objectiu de consolidació de coneixements com la generació d'uns resultats que puguin ser presentats pel seu valor propi i independent.
% Chapter Template

\chapter{Objectius} % Main chapter title

\label{Objectius} % Change X to a consecutive number; for referencing this chapter elsewhere, use \ref{ChapterX}

%----------------------------------------------------------------------------------------
%	SECTION 1
%----------------------------------------------------------------------------------------

Definit el context, l'àrea d'estudi i una aproximació a l'estat de l'art actual d'aquest projecte, cal definir amb el màxim nivell de detall quins seran els objectius principals, així com els objectius específics i l'abast, per tal d'introduïr els conceptes treballats durant el desenvolupament del mateix.

\section{Objectiu general}

L’objectiu principal d’aquest projecte consisteix en la implementació d’un sistema software orientat al monitoratge d’altres sistemes softwares. Aquest sistema haurà de complir 3 característiques principals: ser autoadaptable, heterogeni i distribuït. A continuació procedim a explicar en detall què entendrem per aquestes característiques dins el context d’aquest projecte, en base a la contextualització i els conceptes explicats anteriorment:

\begin{enumerate}
\item \textbf{Autoadaptable}. El sistema de monitoratge generat ha d'estar dotat de capacitats d'adaptabilitat de la seva execució en temps real. Mitjançant la gestió i control de la seva activitat, els diferents monitors han d'oferir eines d'adaptació orientades al control de qualitat del propi sistema. Per fer-ho, caldrà tenir en compte dos punts que es desenvoluparan més endavant: en primer lloc, la dotació dels monitors d'aquestes eines d'adaptació; en segon lloc, el disseny i implementació dels components necessaris per gestionar les adaptacions.
\item \textbf{Heterogeni}. El sistema constarà d’un conjunt de monitors de naturalesa variada i permetrà, mitjançant un disseny i una arquitectura prou genèrica, la integració de nous monitors de diversa índole. Per tant, el sistema haurà d’estar capacitat per gestionar els diversos tipus de monitors tot i les seves diferències en aspectes com el sistema monitorat, la naturalesa del monitoratge, les necessitats de configuració, etc. L'objectiu d'aquesta característica és que el resultat final sigui el més aprofitable i reusable possible.
\item \textbf{Distribuït}. El sistema haurà de permetre desplegar els diferents monitors i els components d'adaptabilitat de forma distribuïda i, per tant, tenir la capacitat de desplegar els diferents components com a elements independents dins el nostre sistema genèric.
\end{enumerate}

Els detalls tècnics de l'assoliment d'aquests 3 objectius es desenvoluparan al llarg d'aquesta memòria. 

%-----------------------------------
%	SUBSECTION 1
%-----------------------------------
\section{Objectius específics}

En base a l’objectiu general prèviament establert, cal definir una sèrie d’objectius específics que ens permetran assolir-lo definint unes vies prou clares com per a facilitar el desenvolupament del projecte. Procedim, doncs, a enumerar aquests objectius:

\begin{itemize}
\item[] \textbf{OBJ1.} Definir una planificació pel desenvolupament del projecte en funció dels requisits.
\item[] \textbf{OBJ2.} Dissenyar una arquitectura software adequada a les necessitats.
\item[] \textbf{OBJ3.} Implementar el sistema de monitoratge.
\item[] \textbf{OBJ4.} Implementar el sistema d'adaptació dels monitors.
\item[] \textbf{OBJ5.} Generació d’un producte final usable, que pugui ser desplegable i reproduïble en format demo.
\item[] \textbf{OBJ6.} Configurar i definir l’entorn de desenvolupament i d’ús del sistema.
\item[] \textbf{OBJ7.} Assegurar qualitat i fiabilitat mitjançant els criteris definits.
\item[] \textbf{OBJ8.} Seguir una metodologia de desenvolupament i testing del sistema.
\item[] \textbf{OBJ9.} Definir una sèrie de casos d’ús per mostrar la usabilitat i comportament real del sistema.
\item[] \textbf{OBJ10.} Documentar i justificar l’evolució del projecte.
\end{itemize}

Aquests objectius engloben les dues vessants d'aquest projecte, ja especificades anteriorment: la generació i documentació d'un Treball de Final de Grau, i el disseny i la implementació del sistema descrit. En qualsevol cas, aquests objectius específics defineixen les "metes finals" d'aquest projecte. Per garantir-ne i comprendre el desenvolupament fins a assolir-los, cal definir les tasques i, per tant, l'abast específic d'aquest projecte.

%----------------------------------------------------------------------------------------
%	SECTION 2
%----------------------------------------------------------------------------------------

\section{Abast del projecte}

Els objectius específics prèviament identificats ens donen una visió acurada de l’abast del nostre projecte i les tasques a realitzar. Tot i així, és important reflectir de forma explícita l’abast d’aquest projecte, enumerant els requisits (o dit d’una altra manera, les tasques o necessitats a satisfer) i delimitant el nostre projecte. Ens basarem per tant en els següents punts:

\begin{itemize}
\item Realitzar una recerca bibliogràfica (basada en l’estat de l’art) per assentar les bases i el context del desenvolupament del projecte.
\item Dissenyar, implementar i documentar un disseny arquitectònic software que satisfaci l’objectiu general i els tres criteris (autoadaptabilitat, heterogeneïtat i distribució) del nostre sistema de monitoratge.
\item Dissenyar, implementar i documentar el sistema d'adaptabilitat dels monitors i realitzar la integració amb els mateixos.
\item Definir una sèrie de casos d'ús (mínim de 3 escenaris) que ens permetin validar les funcionalitats del sistema amb exemples mitjançant l'execució real.
\item Dissenyar i implementar un dashboard que permeti visualitzar l'activitat del sistema de monitoratge i adaptabilitat.
\end{itemize}

Aquests punts estableixen el mínim del que podríem considerar com a necessari per considerar que s’han assolit els objectius esmentats a l’apartat 3 d’aquest document. Tot i així, podem preveure la possibiltiat de permetre’ns augmentar les perspectives, i gràcies a l’ús d’una metodologia àgil (veure apartat 5.1. Metodologia de treball), augmentar l’abast del projecte, amb aspectes com incrementar el nombre de monitors implementats, o augmentar les funcionalitats del dashboard. En qualsevol cas, aquests aspectes serien un afegit secundari que únicament tindrà sentit contemplar amb el transcurs del projecte.
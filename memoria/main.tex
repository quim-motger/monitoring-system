%%%%%%%%%%%%%%%%%%%%%%%%%%%%%%%%%%%%%%%%%
% Masters/Doctoral Thesis 
% LaTeX Template
% Version 2.4 (22/11/16)
%
% This template has been downloaded from:
% http://www.LaTeXTemplates.com
%
% Version 2.x major modifications by:
% Vel (vel@latextemplates.com)
%
% This template is based on a template by:
% Steve Gunn (http://users.ecs.soton.ac.uk/srg/softwaretools/document/templates/)
% Sunil Patel (http://www.sunilpatel.co.uk/thesis-template/)
%
% Template license:
% CC BY-NC-SA 3.0 (http://creativecommons.org/licenses/by-nc-sa/3.0/)
%
%%%%%%%%%%%%%%%%%%%%%%%%%%%%%%%%%%%%%%%%%

%----------------------------------------------------------------------------------------
%	PACKAGES AND OTHER DOCUMENT CONFIGURATIONS
%----------------------------------------------------------------------------------------
\documentclass[
11pt, % The default document font size, options: 10pt, 11pt, 12pt
%oneside, % Two side (alternating margins) for binding by default, uncomment to switch to one side
catalan, % ngerman for German
singlespacing, % Single line spacing, alternatives: onehalfspacing or doublespacing
%draft, % Uncomment to enable draft mode (no pictures, no links, overfull hboxes indicated)
%nolistspacing, % If the document is onehalfspacing or doublespacing, uncomment this to set spacing in lists to single
%liststotoc, % Uncomment to add the list of figures/tables/etc to the table of contents
%toctotoc, % Uncomment to add the main table of contents to the table of contents
%parskip, % Uncomment to add space between paragraphs
%nohyperref, % Uncomment to not load the hyperref package
headsepline, % Uncomment to get a line under the header
chapterinoneline, % Uncomment to place the chapter title next to the number on one line
%consistentlayout, % Uncomment to change the layout of the declaration, abstract and acknowledgements pages to match the default layout
table
]{MastersDoctoralThesis} % The class file specifying the document structure

\usepackage[utf8]{inputenc} % Required for inputting international characters
\usepackage[T1]{fontenc} % Output font encoding for international characters

\usepackage{palatino} % Use the Palatino font by default

\usepackage[backend=bibtex,style=numeric,natbib=true,sorting=none]{biblatex} % Use the bibtex backend with the authoryear citation style (which resembles APA)
\usepackage{eurosym}

\usepackage{listings}

\usepackage{multirow}

\usepackage{enumitem}
\usepackage{float}
\setlist[enumerate]{label*=\arabic*.}

\addbibresource{example.bib} % The filename of the bibliography

\usepackage[autostyle=true]{csquotes} % Required to generate language-dependent quotes in the bibliography

%----------------------------------------------------------------------------------------
%	MARGIN SETTINGS
%----------------------------------------------------------------------------------------

\geometry{
	paper=a4paper, % Change to letterpaper for US letter
	inner=2.5cm, % Inner margin
	outer=3.8cm, % Outer margin
	bindingoffset=.5cm, % Binding offset
	top=1.5cm, % Top margin
	bottom=1.5cm, % Bottom margin
	%showframe, % Uncomment to show how the type block is set on the page
}

%----------------------------------------------------------------------------------------
%	THESIS INFORMATION
%----------------------------------------------------------------------------------------

\thesistitle{Sistema de monitoratge autoadaptable heterogeni i distribuït} % Your thesis title, this is used in the title and abstract, print it elsewhere with \ttitle
\supervisor{Xavier \textsc{Franch Gutierrez}} % Your supervisor's name, this is used in the title page, print it elsewhere with \supname
\examiner{Marc \textsc{Oriol Hilari}} % Your examiner's name, this is not currently used anywhere in the template, print it elsewhere with \examname
\degree{Grau d'Enginyeria Informàtica} % Your degree name, this is used in the title page and abstract, print it elsewhere with \degreename
\author{Joaquim \textsc{Motger de la Encarnación}} % Your name, this is used in the title page and abstract, print it elsewhere with \authorname
\addresses{} % Your address, this is not currently used anywhere in the template, print it elsewhere with \addressname

\subject{} % Your subject area, this is not currently used anywhere in the template, print it elsewhere with \subjectname
\keywords{} % Keywords for your thesis, this is not currently used anywhere in the template, print it elsewhere with \keywordnames
\university{{Universitat Politècnica de Catalunya}} % Your university's name and URL, this is used in the title page and abstract, print it elsewhere with \univname
\department{{Enginyeria de Serveis i Sistemes de la Informació}} % Your department's name and URL, this is used in the title page and abstract, print it elsewhere with \deptname
\group{{Enginyeria del Software}} % Your research group's name and URL, this is used in the title page, print it elsewhere with \groupname
\faculty{{Facultat d'Informàtica de Barcelona}} % Your faculty's name and URL, this is used in the title page and abstract, print it elsewhere with \facname

\AtBeginDocument{
\hypersetup{pdftitle=\ttitle} % Set the PDF's title to your title
\hypersetup{pdfauthor=\authorname} % Set the PDF's author to your name
\hypersetup{pdfkeywords=\keywordnames} % Set the PDF's keywords to your keywords
}

\begin{document}

\frontmatter % Use roman page numbering style (i, ii, iii, iv...) for the pre-content pages

\pagestyle{plain} % Default to the plain heading style until the thesis style is called for the body content

%----------------------------------------------------------------------------------------
%	TITLE PAGE
%----------------------------------------------------------------------------------------

\begin{titlepage}
\begin{center}

%\vspace*{.06\textheight}
\includegraphics[width=\textwidth]{fib.png}\vspace{4.0cm} % University/department logo - uncomment to place it
%{\scshape\LARGE \univname\par}\vspace{1.5cm} % University name
%\textsc{\Large Treball de Final de Grau}\\[1.5cm] % Thesis type

\HRule \\[0.4cm] % Horizontal line
{\huge \bfseries \ttitle\par}\vspace{0.4cm} % Thesis title
\HRule \\[1.5cm] % Horizontal line
 
\begin{minipage}[t]{0.4\textwidth}
\begin{flushleft} \large
\emph{Autor:}\\
\href{}{\authorname} % Author name - remove the \href bracket to remove the link
\end{flushleft}
\end{minipage}
\begin{minipage}[t]{0.4\textwidth}
\begin{flushright} \large
\emph{Director:} \\
\href{}{\supname} % Supervisor name - remove the \href bracket to remove the link  
\end{flushright}
\begin{flushright} \large
\emph{Codirector:} \\
\href{}{\examname} % Supervisor name - remove the \href bracket to remove the link  
\end{flushright}
\end{minipage}\\[6cm]
 
%\vfill

\large \textit{Treball de final de grau presentat sota el marc del}\\[0.3cm] % University requirement text
\degreename\\[0.5cm]
\textit{en l'especialitat de}\\[0.3cm]
\groupname\\[2cm] % Research group name and department name
 
%\vfill
 
\vfill
\end{center}
\end{titlepage}

%----------------------------------------------------------------------------------------
%	DECLARATION PAGE
%----------------------------------------------------------------------------------------
%
%\begin{declaration}
%\addchaptertocentry{\authorshipname} % Add the declaration to the table of contents
%\noindent Jo, \authorname, declaro que aquest Treball de Final de Grau titulat \textit{\ttitle}
%i el treball presentats en el mateix són propis. Declaro que:

%\begin{itemize} 
%\item La feina aquí presentada ha estat desenvolupada durant el curs del Grau en Enginyeria Informàtica
%\item S'ha indicat degudament l'ús de publicacions o components d'autoria externa, utilitzats com a suport pel desenvolupament del projecte, reconeixent l'autoria original dels mateixos
%\item S'ha indicat degudament tota la feina desenvolupada per autoria pròpia
%\item Totes les publicacions terceres consultades pel desenvolupament del treball estan degudament citades
%\item \\
%\end{itemize}
%\noindent I per deixar constància d'aquesta declaració, signo el present document\\[2cm]
 
%\noindent Signatura:\\
%\rule[2cm]{25em}{0.5pt} % This prints a line for the signature
% 
%\noindent Date:\\
%\rule[0.5em]{25em}{0.5pt} % This prints a line to write the date
%\end{declaration}

%----------------------------------------------------------------------------------------
%	QUOTATION PAGE
%----------------------------------------------------------------------------------------

%\vspace*{0.2\textheight}

%\noindent{\itshape A mi madre, quien me dio las herramientas para ser capaz de hacer camino al andar.}\bigbreak

%----------------------------------------------------------------------------------------
%	ABSTRACT PAGE
%----------------------------------------------------------------------------------------

\begin{Abstracte}
\addchaptertocentry{Abstracte} % Add the abstract to the table of contents
El monitoratge consisteix en la tècnica d'observació i control dels sistemes software amb l'objectiu de garantir la seva fiabilitat, la qualitat del servei (Quality of Service, QoS), la seguretat i altres característiques dels sistemes software pròpies de la seva execució en temps real. El monitoratge proporciona, entre d'altres resultats, la informació que permet a un sistema software auto-adaptable modificar la seva execució davant la violació d'uns certs valors sobre aquestes característiques, tals com per exemple un increment elevat del temps de resposta d'un sistema o la detecció d'un nombre elevat d'errors per unitat de temps. De la mateixa manera, els sistemes de monitoratge (com a sistemes software que són) requereixen també poder adaptar la seva execució per satisfer la seva fiabilitat. En aquest context, com podem dotar a un sistema de monitoratge de capacitats auto-adaptables?\\

En base a aquesta premisa, aquest Treball de Final de Grau consisteix en el disseny i implementació d'un sistema de monitoratge autoadaptable, heterogeni i distribuït que integra un conjunt de monitors de naturalesa diversa i permet, mitjançant la gestió i adaptació de diagrames UML, la seva reconfiguració de forma automàtica. La proposta i el desenvolupament plantejats en aquest projecte es validen mitjançant casos d'ús reals i, en especial èmfasi, amb la seva integració dins el marc de SUPERSEDE, un projecte del programa Horizon 2020 enfocat a la gestió del cicle de vida dels serveis software i les aplicacions, amb l'objectiu de millorar l'experiència final de l'usuari en l'ús d'aquests sistemes.\\

\end{Abstracte}


%----------------------------------------------------------------------------------------
%	ACKNOWLEDGEMENTS
%----------------------------------------------------------------------------------------

%\begin{acknowledgements}
%\addchaptertocentry{\acknowledgementname} % Add the acknowledgements to the table of contents
%The acknowledgments and the people to thank go here, don't forget to include your project advisor\ldots
%\end{acknowledgements}

%----------------------------------------------------------------------------------------
%	LIST OF CONTENTS/FIGURES/TABLES PAGES
%----------------------------------------------------------------------------------------

\tableofcontents % Prints the main table of contents

\listoffigures % Prints the list of figures

\listoftables % Prints the list of tables

%----------------------------------------------------------------------------------------
%	ABBREVIATIONS
%----------------------------------------------------------------------------------------

%\begin{abbreviations}{ll} % Include a list of abbreviations (a table of two columns)

%\textbf{LAH} & \textbf{L}ist \textbf{A}bbreviations \textbf{H}ere\\
%\textbf{WSF} & \textbf{W}hat (it) \textbf{S}tands \textbf{F}or\\

%\end{abbreviations}

%----------------------------------------------------------------------------------------
%	PHYSICAL CONSTANTS/OTHER DEFINITIONS
%----------------------------------------------------------------------------------------

%\begin{constants}{lr@{${}={}$}l} % The list of physical constants is a three column table

% The \SI{}{} command is provided by the siunitx package, see its documentation for instructions on how to use it

%Speed of Light & $c_{0}$ & \SI{2.99792458e8}{\meter\per\second} (exact)\\
%Constant Name & $Symbol$ & $Constant Value$ with units\\

%\end{constants}

%----------------------------------------------------------------------------------------
%	SYMBOLS
%----------------------------------------------------------------------------------------

%\begin{symbols}{lll} % Include a list of Symbols (a three column table)

%$a$ & distance & \si{\meter} \\
%$P$ & power & \si{\watt} (\si{\joule\per\second}) \\
%Symbol & Name & Unit \\

%\addlinespace % Gap to separate the Roman symbols from the Greek

%$\omega$ & angular frequency & \si{\radian} \\

%\end{symbols}

%----------------------------------------------------------------------------------------
%	DEDICATION
%----------------------------------------------------------------------------------------

%\dedicatory{For/Dedicated to/To my\ldots} 

%----------------------------------------------------------------------------------------
%	THESIS CONTENT - CHAPTERS
%----------------------------------------------------------------------------------------

\mainmatter % Begin numeric (1,2,3...) page numbering

\pagestyle{thesis} % Return the page headers back to the "thesis" style

% Include the chapters of the thesis as separate files from the Chapters folder
% Uncomment the lines as you write the chapters

% Chapter Template

\chapter{Introducció} % Main chapter title

\label{Introduccio} % Change X to a consecutive number; for referencing this chapter elsewhere, use \ref{ChapterX}

El present document consisteix en l'informe de seguiment del Treball de Final de Grau (TFG) del Grau en Enginyeria Informàtica titulat \textit{Sistema de monitoratge autoadaptable, heterogeni i distribuït}. Com a document presentat durant el procés de desenvolupament del mateix projecte, els objectius principals d'aquest en són dos.\\

En primer lloc, establir una "fotografia" documentada del projecte: és a dir, projectar en quin estat es troba el projecte, quina és la feina que s'ha realitzat fins al moment, en quint punt del desenvolupament es troba, i quines són les tasques que, d'acord amb la contextualització i plantejaments presentats, queda per fer. D'aquesta manera, aquest document pretèn reflectir la realitat del projecte en el moment de la seva presentació.\\

En segon lloc, aquest document serveix de garantia pel desenvolupament i finalització del projecte. D'acord amb els terminis establerts, i a les tasques realitzades i per realitzar, els detalls del treball realitzat i les consideracions pertinents han de ser una prova de la viabilitat de presentació del projecte.\\

Per englobar aquests objectius, es presenten primerament la contextualització i detalls de l'entorn on desenvolupem el treball, així com la informació relacionada amb la gestió i planificació en relació a tot allò establert a GEP. En aquest sentit, s'inclouen consideracions com el sorgiment de desviacions i alternatives, i la integració dels coneixements i aspectes teòrics que han calgut considerar pel desenvolupament del projecte.
% Chapter Template

\chapter{Contextualització} % Main chapter title

\label{Contextualitzacio} % Change X to a consecutive number; for referencing this chapter elsewhere, use \ref{ChapterX}

%----------------------------------------------------------------------------------------
%	SECTION 1
%----------------------------------------------------------------------------------------

\section{Justificació de la temàtica}

És un fet innegable que en les darreres dècades els sistemes software han evolucionat fins al punt d’esdevenir elements clau i imprescindibles de les activitats primàries de qualsevol empresa, organització o institució. La gestió de la informació, els protocols i controls de seguretat, els processos de negoci, etc., són els reptes als quals els CIO de moltes empreses s’han d’enfrontar. Aquests reptes i els seus resultats depenen, en gran mesura, del comportament dels sistemes software que entren en joc dins aquestes activitats. 
El comportament d’aquests sistemes, per tant, és clau. És per aquest motiu que eventualment ha anat prenent força un concepte basat en l’estudi i control de qualitat dels sistemes software: el monitoratge.\\

Com a part de la vida professional d’un enginyer de software, la supervisió i control dels components i sistemes amb els què treballa és un concepte clau amb el qual, d’una forma o altra, ha d’estar familiaritzat. Però el problema que plantegem aquí va més enllà: després del repte de monitorar els sistemes, ens hem de plantejar com dissenyar, gestionar i adaptar aquest monitoratge.\\

En aquest àmbit, i tal com veurem més endavant a l’apartat \textit{2.3. Estat de l’art}, existeix una àmplia recerca que actualment treballa i desenvolupa projectes en relació a aquest àmbit. El potencial d’estudi que ofereix resulta d’un alt interès a causa de la possibilitat de recerca i síntesi i als diferents aspectes i criteris sobre els quals es pot treballar.\\

Així, tant com estudiant com a futur professional del sector de l’enginyeria del software, es poden contemplar diversos criteris per treballar en aquesta temàtica:
\begin{itemize}
\item Un aprofundiment en els coneixements de l’enginyeria i els sistemes software
\item Treball i recerca en conceptes de control de qualitat i fiabilitat
\item Possibilitat de col·laborar i aprofundir en un tema de recerca d’actualitat dins l’enginyeria de serveis i els sistemes d’informació
\item Plantejament d’un projecte complet que pugui servir a tercers interessats en l’estudi de sistemes de monitoratge autoadaptatius
\end{itemize}  


%-----------------------------------
%	SUBSECTION 1
%-----------------------------------
\section{Definició de l'àrea d'estudi}

Nunc posuere quam at lectus tristique eu ultrices augue venenatis. Vestibulum ante ipsum primis in faucibus orci luctus et ultrices posuere cubilia Curae; Aliquam erat volutpat. Vivamus sodales tortor eget quam adipiscing in vulputate ante ullamcorper. Sed eros ante, lacinia et sollicitudin et, aliquam sit amet augue. In hac habitasse platea dictumst.

%-----------------------------------
%	SUBSECTION 2
%-----------------------------------

\subsection{Identificació dels stakeholders}
Morbi rutrum odio eget arcu adipiscing sodales. Aenean et purus a est pulvinar pellentesque. Cras in elit neque, quis varius elit. Phasellus fringilla, nibh eu tempus venenatis, dolor elit posuere quam, quis adipiscing urna leo nec orci. Sed nec nulla auctor odio aliquet consequat. Ut nec nulla in ante ullamcorper aliquam at sed dolor. Phasellus fermentum magna in augue gravida cursus. Cras sed pretium lorem. Pellentesque eget ornare odio. Proin accumsan, massa viverra cursus pharetra, ipsum nisi lobortis velit, a malesuada dolor lorem eu neque.

%----------------------------------------------------------------------------------------
%	SECTION 2
%----------------------------------------------------------------------------------------

\section{Estat de l'art}

Sed ullamcorper quam eu nisl interdum at interdum enim egestas. Aliquam placerat justo sed lectus lobortis ut porta nisl porttitor. Vestibulum mi dolor, lacinia molestie gravida at, tempus vitae ligula. Donec eget quam sapien, in viverra eros. Donec pellentesque justo a massa fringilla non vestibulum metus vestibulum. Vestibulum in orci quis felis tempor lacinia. Vivamus ornare ultrices facilisis. Ut hendrerit volutpat vulputate. Morbi condimentum venenatis augue, id porta ipsum vulputate in. Curabitur luctus tempus justo. Vestibulum risus lectus, adipiscing nec condimentum quis, condimentum nec nisl. Aliquam dictum sagittis velit sed iaculis. Morbi tristique augue sit amet nulla pulvinar id facilisis ligula mollis. Nam elit libero, tincidunt ut aliquam at, molestie in quam. Aenean rhoncus vehicula hendrerit.

%-----------------------------------
%	SUBSECTION 2
%-----------------------------------

\subsection{Projecte SUPERSEDE}
Morbi rutrum odio eget arcu adipiscing sodales. Aenean et purus a est pulvinar pellentesque. Cras in elit neque, quis varius elit. Phasellus fringilla, nibh eu tempus venenatis, dolor elit posuere quam, quis adipiscing urna leo nec orci. Sed nec nulla auctor odio aliquet consequat. Ut nec nulla in ante ullamcorper aliquam at sed dolor. Phasellus fermentum magna in augue gravida cursus. Cras sed pretium lorem. Pellentesque eget ornare odio. Proin accumsan, massa viverra cursus pharetra, ipsum nisi lobortis velit, a malesuada dolor lorem eu neque.
% Chapter Template

\chapter{Objectius} % Main chapter title

\label{Objectius} % Change X to a consecutive number; for referencing this chapter elsewhere, use \ref{ChapterX}

%----------------------------------------------------------------------------------------
%	SECTION 1
%----------------------------------------------------------------------------------------

Definit el context, l'àrea d'estudi i una aproximació a l'estat de l'art actual d'aquest projecte, cal definir amb el màxim nivell de detall quins seran els objectius principals, així com els objectius específics i l'abast, per tal d'introduïr els conceptes treballats durant el desenvolupament del mateix.

\section{Objectiu general}

L’objectiu principal d’aquest projecte consisteix en la implementació d’un sistema software orientat al monitoratge d’altres sistemes softwares. Aquest sistema haurà de complir 3 característiques principals: ser autoadaptable, heterogeni i distribuït. A continuació procedim a explicar en detall què entendrem per aquestes característiques dins el context d’aquest projecte, en base a la contextualització i els conceptes explicats anteriorment:

\begin{enumerate}
\item \textbf{Autoadaptable}. El sistema de monitoratge generat ha d'estar dotat de capacitats d'adaptabilitat de la seva execució en temps real. Mitjançant la gestió i control de la seva activitat, els diferents monitors han d'oferir eines d'adaptació orientades al control de qualitat del propi sistema. Per fer-ho, caldrà tenir en compte dos punts que es desenvoluparan més endavant: en primer lloc, la dotació dels monitors d'aquestes eines d'adaptació; en segon lloc, el disseny i implementació dels components necessaris per gestionar les adaptacions.
\item \textbf{Heterogeni}. El sistema constarà d’un conjunt de monitors de naturalesa variada i permetrà, mitjançant un disseny i una arquitectura prou genèrica, la integració de nous monitors de diversa índole. Per tant, el sistema haurà d’estar capacitat per gestionar els diversos tipus de monitors tot i les seves diferències en aspectes com el sistema monitorat, la naturalesa del monitoratge, les necessitats de configuració, etc. L'objectiu d'aquesta característica és que el resultat final sigui el més aprofitable i reusable possible.
\item \textbf{Distribuït}. El sistema haurà de permetre desplegar els diferents monitors i els components d'adaptabilitat de forma distribuïda i, per tant, tenir la capacitat de desplegar els diferents components com a elements independents dins el nostre sistema genèric.
\end{enumerate}

Els detalls tècnics de l'assoliment d'aquests 3 objectius es desenvoluparan al llarg d'aquesta memòria. 

%-----------------------------------
%	SUBSECTION 1
%-----------------------------------
\subsection{Objectius específics}

En base a l’objectiu general prèviament establert, cal definir una sèrie d’objectius específics que ens permetran assolir-lo definint unes vies prou clares com per a facilitar el desenvolupament del projecte. Procedim, doncs, a enumerar aquests objectius:

\begin{itemize}
\item \textbf{OBJ1.} Definir una planificació pel desenvolupament del projecte en funció dels requisits.
\item \textbf{OBJ2.} Dissenyar una arquitectura software adequada a les necessitats.
\item \textbf{OBJ3.} Implementar el sistema de monitoratge.
\item \textbf{OBJ4.} Implementar el sistema d'adaptació dels monitors.
\item \textbf{OBJ5.} Generació d’un producte final usable, que pugui ser desplegable i reproduïble en format demo.
\item \textbf{OBJ6.} Configurar i definir l’entorn de desenvolupament i d’ús del sistema.
\item \textbf{OBJ7.} Assegurar qualitat i fiabilitat mitjançant els criteris definits.
\item \textbf{OBJ8.} Seguir una metodologia de desenvolupament i testing del sistema.
\item \textbf{OBJ9.} Definir una sèrie de casos d’ús per mostrar la usabilitat i comportament real del sistema.
\item \textbf{OBJ10.} Documentar i justificar l’evolució del projecte.
\end{itemize}

Aquests objectius engloben les dues vessants d'aquest projecte, ja especificades anteriorment: la generació i documentació d'un Treball de Final de Grau, i el disseny i la implementació del sistema descrit. En qualsevol cas, aquests objectius específics defineixen les "metes finals" d'aquest projecte. Per garantir-ne i comprendre el desenvolupament fins a assolir-los, cal definir les tasques i, per tant, l'abast específic d'aquest projecte.

%----------------------------------------------------------------------------------------
%	SECTION 2
%----------------------------------------------------------------------------------------

\section{Abast del projecte}

Els objectius específics prèviament identificats ens donen una visió acurada de l’abast del nostre projecte i les tasques a realitzar. Tot i així, és important reflectir de forma explícita l’abast d’aquest projecte, enumerant els requisits (o dit d’una altra manera, les tasques o necessitats a satisfer) i delimitant el nostre projecte. Ens basarem per tant en els següents punts:

\begin{itemize}
\item Realitzar una recerca bibliogràfica (basada en l’estat de l’art) per assentar les bases i el context del desenvolupament del projecte.
\item Dissenyar, implementar i documentar un disseny arquitectònic software que satisfaci l’objectiu general i els tres criteris (autoadaptabilitat, heterogeneïtat i distribució) del nostre sistema de monitoratge.
\item Dissenyar, implementar i documentar el sistema d'adaptabilitat dels monitors i realitzar la integració amb els mateixos.
\item Definir una sèrie de casos d'ús (mínim de 3 escenaris) que ens permetin validar les funcionalitats del sistema amb exemples mitjançant l'execució real.
\item Dissenyar i implementar un dashboard que permeti visualitzar l'activitat del sistema de monitoratge i adaptabilitat.
\end{itemize}

Aquests punts estableixen el mínim del que podríem considerar com a necessari per considerar que s’han assolit els objectius esmentats a l’apartat 3 d’aquest document. Tot i així, podem preveure la possibiltiat de permetre’ns augmentar les perspectives, i gràcies a l’ús d’una metodologia àgil (veure apartat 5.1. Metodologia de treball), augmentar l’abast del projecte, amb aspectes com incrementar el nombre de monitors implementats, o augmentar les funcionalitats del dashboard. En qualsevol cas, aquests aspectes serien un afegit secundari que únicament tindrà sentit contemplar amb el transcurs del projecte. 
% Chapter Template

\chapter{Gestió i desenvolupament} % Main chapter title

\label{GestioIDesenvolupament} % Change X to a consecutive number; for referencing this chapter elsewhere, use \ref{ChapterX}

Abans d'entrar en els detalls del projecte, necessitem definir sota quins criteris i quines pràctiques realitzarem la gestió i el desenvolupament del projecte.

%----------------------------------------------------------------------------------------
%	SECTION 1
%----------------------------------------------------------------------------------------

\section{Metodologia de desenvolupament}

Les necessitats i requisits específics del projecte aniran fortament relacionades amb la recerca i l’avenç del propi transcurs del projecte. Si bé els objectius específics queden clars, les tasques a desenvolupar aniran evolucionant dinàmicament. Per aquesta raó, en aquest cas serà adequat seguir una metodologia de desenvolupament àgil. I, en concret, es seguirà una simplificació de la metodologia Kanban. \\

En base als requisits establerts inicialment durant la planificació del projecte, i al llarg del seu transcurs, s’aniran generant una sèrie de tasques que s’afegiran a un backlog o to-do list; és a dir, el conjunt de tasques amb la mínima granularitat que aporti valor al projecte com a producte entregable. D’acord a les necessitats, s’aplicarà una priorització, i aquestes tasques s’aniran afegint com a tasques realitzant o en progrés. Conforme aquestes tasques es completin, s’afegiran al llistat de tasques realitzades o done, mantenint així un control dels requisits que s’estan satisfent i el seu grau de completesa. \\

\begin{figure}
\centering
\includegraphics[width=8cm]{Figures/Figure3}
\decoRule
\caption[Simplificació de la metodologia Kanban]{Simplificació de la metodologia Kanban}
\label{fig:Figura3}
\end{figure}

Per garantir la integritat del sistema, cadascuna d’aquestes tasques serà desenvolupada fora de l’entorn de producció, en un entorn (o branca) separats. D’aquesta manera, el desenvolupament no afectarà al producte provisional generat en cada moment del desenvolupament del projecte. I, alhora, permetrà fer un seguiment més exacte de l’estat de cada tasca o funcionalitat a implementar. \\

S’ha considerat millor opció a, per exemple, alternatives àgils com Scrum, degut a diversos factors. P.e., a Kanban les entregues o releases són constants, i no acotades temporalment. Considerarem més important, per tant, la metodologia basada en el producte final. \\

%-----------------------------------
%	SUBSECTION 1
%-----------------------------------
\section{Recursos}

Per definir les necessitats de recursos per satisfer la realització del projecte, els classificarem segons el següent criteri: recursos \textbf{humans}, recursos \textbf{materials} i recursos \textbf{software}.\\

\noindent \textbf{\large Recursos humans}\\

\noindent Pel domini de nostre projecte, basat en 1 desenvolupador principal i 2 gestors de projecte (1 director + 1 co-director), considerarem les seves hores de treball com recursos humans. Estimarem, i considerant els següents aspectes:

\begin{itemize}
\item El TFG es correspon a 18 ECTS (3 crèdits ECTS GEP + 15 crèdits ECTS TFG)
\item 1 ECTS = 25-30 hores de treball
\item Durada aproximada TFG = 22 setmanes
\end{itemize}

Podem estimar, per tant, pel cas del desenvolupador (alumne) una dedicació d’unes \textbf{24 hores} a la setmana. Pel cas dels gestors, farem una estimació aproximada de \textbf{50 hores} en total per part dels dos rols, com a tasques de suport.\\

\noindent \textbf{\large Recursos materials}\\

\noindent Els recursos materials per aquest projecte són relativament senzills. Bàsicament:

\begin{itemize}
\item \textbf{Portàtil Lenovo G-50}. Màquina principal amb la qual es durà a terme el projecte
\item \textbf{Materials d'impressió}. Necessaris per documentació, impressió de memòria, etc.
\end{itemize}

\noindent \textbf{\large Recursos software}\\

\noindent Tot i que aquests es discutiran amb més detall al capítol \textit{5. Eines de desenvolupament}, podem identificar inicialment la necessitat d'alguns recursos software principals, tals com:

\begin{itemize}
\item \textbf{Gestor de versions}. El codi (emmagatzemament, desenvolupament i evolució) requereix un control i manteniment, motiu pel qual s'estableix com a necessitat una eina d'aquest tipus
\item \textbf{IDE}. És necessari l'ús d'un entorn de desenvolupament integrat per desenvolupar el projecte, dissenyar l'arquitectura, realitzar la implementació, configurar l'entorn, etc.
\item \textbf{Sistema de compilació automàtic}. Necessari per gestionar la compilació i les dependències dels projectes.
\item \textbf{Framework desenvolupament web}. Segons les necessitats que es defineixin al llarg del projecte, es triarà una tecnologia específica per desenvolupar el dashboard definit com a requisit al projecte.
\end{itemize}

Les opcions triades per suplir les necessitats d'aquests recursos i altres recursos software identificats com a necessaris es plantejaran més endavant.

\section{Planificació temporal}

Segons les necessitats d’aquest projecte i les seves característiques, classificarem el desenvolupament en 4 fases: la planificació, el disseny, la implementació, i la documentació i entrega del projecte. 

\subsection{Descripció de fases}

\noindent \textbf{\large Planificació}\\

\noindent En primer lloc cal una fase inicial o fase de planificació. L’objectiu principal d’aquesta fase del projecte és definir els aspectes bàsics que definiran la naturalesa del projecte: els objectius, la metodologia de treball, el desenvolupament, els requisits, la planificació, etc. És a dir, tot allò relacionat amb l’establiment de les premisses sobre les quals ens basarem per desenvolupar el nostre projecte. \\

Aquesta fase del projecte va fortament lligada al desenvolupament del curs GEP, juntament amb altres activitats. En definitiva, les tasques a realitzar són les següents:

\begin{itemize}
\item \textbf{P1.} Investigació i recerca en quant al monitoratge i l’adaptabilitat de sistemes software.
\item \textbf{P2.} Definir l’abast i el context del projecte. 
\item \textbf{P3.} Definir les fases i tasques del projecte, així com una planificació temporal de desenvolupament.
\item \textbf{P4.} Preparar una gestió econòmica i una anàlisi de sostenibilitat.
\item \textbf{P5.} Establir els requisits i necessitats d’acord a l’especialitat d’Enginyeria del Software.
\item \textbf{P6.} Definir les eines de desenvolupament i els llenguatges amb els que treballar i implementar el sistema i el dashboard.
\item \textbf{P7.} Definir els casos d'ús per validar i implementar el sistema.
\end{itemize}

\noindent \textbf{\large Disseny}\\

Conforme la fase de planificació avanci, podrem començar a realitzar el disseny del nostre sistema des d’un punt de vista de requisits i també arquitectònic (en referència a arquitectura del software). És a dir: l’objectiu d’aquesta fase és passar dels conceptes i objectius definits a la planificació a un “mapa” o “esquema” que serveixi de guia pel desenvolupament del projecte (subjecte, per descomptat, a possibles canvis i adaptacions al llarg del desenvolupament). \\

Principalment, les tasques a incloure en aquesta fase són:

\begin{itemize}
\item \textbf{D1.} Definir els entorns de configuració sota els quals es desenvoluparà el projecte
\item \textbf{D2.} Dissenyar l’arquitectura software del sistema de monitoratge i dels monitors
\item \textbf{D3.} Definir els criteris d’adaptabilitat amb els quals es dotarà al sistema
\item \textbf{D4.} Configurar un entorn de configuració d’acord amb els criteris definits
\item \textbf{D5.} Documentar els avanços referents a aquesta fase (memòria)
\end{itemize}

\noindent \textbf{\large Implementació}\\

\noindent Aquesta fase tindrà la major càrrega de feina de tot el projecte. Partirà d’uns criteris i objectius ben definits i estructurats a partir de les anteriors fases. Les tasques es correspondran principalment a totes aquelles tasques d’implementació i de testing del nostre sistema. \\

Per tant, identificarem com a tasques:

\begin{itemize}
\item \textbf{I1.} Implementació de l’arquitectura genèrica del sistema de monitoratge.
\item \textbf{I2.} Implementació del sistema de monitors.
\item \textbf{I3.} Implementació del sistema d'adaptabilitat.
\item \textbf{I4.} Integració dels sistemes.
\item \textbf{I5.} Implementació d’un dashboard que permeti visualitzar l’activitat del sistema
\item \textbf{I6.} Generació de documentació referent al desenvolupament i la implementació (documentació d’APIs, README per desplegar el sistema, manual d’usuari del dashboard, etc.)
\item \textbf{I7.} Ampliació del sistema (afegir monitors, ampliar dashboard) en funció de les necessitats i/o de la disponibilitat temporal
\item \textbf{I8.} Redacció i ampliació de la memòria
\end{itemize}

\noindent \textbf{\large Fase final}\\

\noindent Finalment, identificarem una darrera fase del projecte que servirà de cloenda per preparar l’entrega i defensa final de la feina realitzada, així com tancar possibles tasques pendents i assegurar el correcte funcionament del sistema i la generació d’un producte final adequat. \\

Les tasques principals seran:

\begin{itemize}
\item \textbf{F1.} Testing de les funcionalitats del sistema
\item \textbf{F2.} Comprovació de la satisfacció dels objectius i requisits
\item \textbf{F3.} Finalització i revisió de la memòria
\item \textbf{F4.} Preparació de la defensa final
\end{itemize}

\subsection{Previsió d'alternatives i pla d'acció}

La planificació temporal i de tasques plantejada, així com el consum de recursos, contempla un desenvolupament del treball regular i sense imprevistos. Tot i així, hem de considerar alternatives al desenvolupament fruït d’imprevistos, desviacions o altres factors no contemplats dins de la normalitat. Identificarem, per tant, les possibles següents desviacions:

\begin{itemize}
\item \textbf{Increment del nº d’hores necessàries.} Aquest problema pot ser derivat per diverses causes (inhabilitació temporal del desenvolupador, dificultats tècniques en el desenvolupament, etc.). Això pot provocar, per una banda, una necessitat de més hores, i per altra banda, més dedicació per unitat de temps.
\subitem \textbf{Pla d’acció.} La tasca corresponent a la implementació del dashboard serà adaptada en funció de l’estat del projecte arribat al moment. És a dir: al tractar-se d’un requisit molt flexible en quant a la seva complexitat (podem aspirar a un dashboard molt complet i multifuncional, o bé establir els requisits mínims per satisfer els criteris d’acceptació del TFG), podem permetre’ns retallar hores d’aquesta tarda, prioritzant aspectes crucials (com p.e. la implementació dels monitors). 
\item \textbf{Avaria en el hardware.} És possible que durant el desenvolupament del projecte el hardware utilitzat (en aquest cas, el portàtil Lenovo G-50) pateixi alguna avaria. Al tractar-se de la principal eina de desenvolupament, això pot afectar als terminis de desenvolupament.
\subitem \textbf{Pla d’acció.} Dues mesures complementàries: per una banda, en tot moment es mantindran diverses còpies de seguretat de tots els artefactes generats (documentació, software, entorns de configuració…) per garantir-ne la recuperabilitat; per altra banda, disposarem d’un entorn de treball alternatiu (un sistema operatiu instal·lat a un disc dur extern) on podrem seguir el desenvolupament del nostre projecte paral·lelament mentre l’avaria es soluciona.
\item \textbf{Canvis en els requisits del projecte.} Podem suposar que el desenvolupament pràctic del projecte ens durà a concloure nous canvis necessaris en quant als requisits prèviament establerts.
\subitem \textbf{Pla d’acció.} En primer lloc, sotmetre a constant revisió el projecte per tal d’evitar al màxim que es produeixi un canvi de requisits significatiu. Per fer-ho, constantment es revisaran requisits, nivell de viabilitat, i satisfacció envers els terminis establerts. Si, per contra, es produeix un canvi significatiu inevitable, el nº d’hores total del projecte (aproximats) permet un petit increment fruit d’aquesta desviació, per tal de garantir que la resta de tasques es duen a terme correctament.
\end{itemize}

%-----------------------------------
%	SUBSECTION 2
%-----------------------------------

\section{Viabilitat}

Un dels principals objectius del Treball de Final de Grau és plasmar la capacitat de l'estudiant de generar, mitjançant els coneixements obtinguts, un producte o projecte real, amb una utilitat i uns objectius aplicables al nostre entorn. Per aquest motiu, i com estudiants, cal assumir la responsabilitat del projecte i estudiar-ne la seva viabilitat en dos sentits. Per una banda, la seva viabilitat econòmica, mitjançant una estimació pressupostària dels costos del projecte en un àmbit professional. Per altra banda, el seu impacte econòmic, social i ambiental des d'un punt de vista de sostenibilitat.

%----------------------------------------------------------------------------------------
%	SECTION 2
%----------------------------------------------------------------------------------------

\subsection{Estimació pressupostària}

Per simplificar al màxim l’estudi dels costos i, alhora, clarificar i estudiar-ne la justificació, procedirem a identificar i estimar els diferents costos associats al projecte segons el seu tipus.\\

\noindent \textbf{\large Despeses directes}\\

\noindent Entraran dins la classificació de despeses directes aquelles despeses derivades directament de la realització d’activitats previstes pel desenvolupament del projecte. Per una major precisió, relacionarem directament les activitats definides a l’anterior entregable amb els costos directes.  \\

\begin{table}[htb]
\centering
\label{PressupostActivitats}
\resizebox{\textwidth}{!}{%
\begin{tabular}{lrrrrr}
\hline \textbf{ACTIVITAT}                           & {\color[HTML]{000000} \textbf{HORES TOTALS}} & {\color[HTML]{000000} \textbf{DEVELOPER}} & {\color[HTML]{000000} \textbf{DIRECTOR}} & {\color[HTML]{000000} \textbf{CODIRECTOR}} & {\color[HTML]{000000} \textbf{COST TOTAL}} \\ \hline
{\color[HTML]{000000} \textbf{Planificació}} & {\color[HTML]{000000} \textbf{149}}          & {\color[HTML]{000000} \textbf{125}}            & {\color[HTML]{000000} \textbf{12}}       & {\color[HTML]{000000} \textbf{12}}         & {\color[HTML]{000000} \textbf{2475,00\euro}} \\
\hline
Abast i context                              & 30                                           & 25                                             & 2,5                                      & 2,5                                        & 500,00\euro                                        \\
Planificació                                 & 12                                           & 10                                             & 1                                        & 1                                          & 200,00\euro                                        \\
Gestió econòmica i sostenibilitat            & 12                                           & 10                                             & 1                                        & 1                                          & 200,00\euro                                        \\
Requisits d'especialitat                     & 12                                           & 10                                             & 1                                        & 1                                          & 200,00\euro                                        \\
Recerca de la temàtica                       & 38                                           & 30                                             & 4                                        & 4                                          & 650,00\euro                                        \\
Eines de desenvolupament                     & 20                                           & 20                                             & 0                                        & 0                                          & 300,00\euro                                        \\
Definir casos d'ús                           & 25                                           & 20                                             & 2,5                                      & 2,5                                        & 425,00\euro                                        \\
\hline
\textbf{Disseny}                             & \textbf{102}                                 & \textbf{88}                                    & \textbf{4}                               & \textbf{10}                                & \textbf{1670,00\euro}                              \\
\hline
Entorn de configuració                       & 15                                           & 15                                             & 0                                        & 0                                          & 225,00\euro                                        \\
Arquitectura software                        & 32                                           & 25                                             & 2                                        & 5                                          & 550,00\euro                                        \\
Criteris d'autoadaptabilitat                 & 25                                           & 20                                             & 2                                        & 3                                          & 425,00\euro                                        \\
Integració continuada                        & 20                                           & 18                                             & 0                                        & 2                                          & 320,00\euro                                       \\
Documentació memòria                         & 10                                           & 10                                             & 0                                        & 0                                          & 150,00\euro                                        \\
\hline
\textbf{Implementació}                       & \textbf{203}                                 & \textbf{186}                                   & \textbf{0}                               & \textbf{17}                                & \textbf{3215,00\euro}                              \\
\hline
Implementació arquitectura genèrica                       & 30                                           & 26                                             & 0                                        & 4                                          & 490,00\euro                                        \\
Implementació monitors                       & 70                                           & 65                                             & 0                                        & 5                                          & 1100,00\euro                                        \\
Implementació sistema d'adaptabilitat                      & 50                                           & 45                                             &0                                        & 5                                          & 800,00\euro                                        \\
Integració sistemes                      & 20                                           & 20                                             & 0                                        & 0                                          & 300,00\euro                                        \\
Implementació dashboard                      & 33                                           & 30                                             & 0                                        & 3                                          & 525,00\euro                                        \\
Documentació de components                   & 15                                           & 15                                             & 0                                        & 0                                          & 225,00\euro                                        \\
Ampliació del sistema                        & 15                                           & 15                                             & 0                                        & 0                                          & 225,00\euro                                        \\
Documentació memòria                         & 10                                           & 10                                             & 0                                        & 0                                          & 150,00\euro                                        \\
\hline
\textbf{Fase final}                          & \textbf{85}                                  & \textbf{75}                                    & \textbf{4}                               & \textbf{6}                                 & \textbf{1375,00\euro}                              \\
\hline
Finalització tasques                         & 10                                           & 10                                             & 0                                        & 0                                          & 150,00\euro                                        \\
Testing                                      & 19                                           & 15                                             & 2                                        & 2                                          & 325,00\euro                                        \\
Comprovació satisfacció                      & 14                                           & 10                                             & 2                                        & 2                                          & 250,00\euro                                        \\
FInalització memòria                         & 25                                           & 25                                             & 0                                        & 0                                          & 375,00\euro                                        \\
Preparació defensa final                     & 17                                           & 15                                             & 0                                        & 2                                          & 275,00\euro                                        \\
\hline
\textbf{Total}                               & \textbf{539}                                 & \textbf{474}                                   & \textbf{20}                              & \textbf{45}                                & \textbf{8735,00\euro}       \\
\hline                      
\end{tabular}}
\caption{Costos directes}
\end{table}

Per aquest cas, farem les següents assumpcions:

\begin{itemize}
\item En el projecte intervindran 3 agents: el desenvolupador (alumne), i dos gestors de projectes (director i codirector). La principal diferència entre el director i codirector, per aquest cas, serà la involucració de cadascun en el desenvolupament del projecte segons la fase (el director tindrà major pes durant la fase inicial, mentre que el codirector donarà més suport al desenvolupament).
\item L’estimació del cost serà de 12\euro /hora pel desenvolupador i 25\euro /hora pels gestors del projecte, en base a la informació actual que podem trobar referent a aquest aspecte pels rols de programador junior i gestor de projecte a portals com InfoJobs.
\end{itemize}

Per cada activitat estimarem un nº d’hores total i un grau d’implicació de cada rol. Les unitats corresponen a hores pel nº d’hores totals i de cada rol, i a \euro pel cost total.\\

\noindent \textbf{\large Despeses indirectes i amortitzacions}\\

\noindent Considerarem les següents despeses indirectes i amortitzacions:

\begin{itemize}
\item \textbf{Impressions a paper}. Considerarem 150 fulls / memòria, a 3 memòries a entregar per la defensa final + 1 de provisional; com a afegit, 200 fulls per articles, documentació… fan un total de 800 fulls.
\item \textbf{Electricitat}. Cost i consum en base a referències de característiques de portàtil i preu estàndard de companyies elèctriques.
\item \textbf{Amortització}. portàtil Lenovo G-50. Cost en base a compra; percentatge d’amortització en base a les hores útils de vida aproximada i les hores de rendiment esperades de programació, documentació, etc.
\item \textbf{Software}. El desenvolupament serà basat en software lliure i, per tant, sense despesa addicional, però el considerarem com a part del pressupost per possibles desviacions.
\end{itemize}

\begin{table}[htb]
\centering
\label{PressupostIndirectes}
\resizebox{\textwidth}{!}{%
\begin{tabular}{lrrrrr}
\hline \textbf{CONCEPTE}                           & {\color[HTML]{000000} \textbf{COST UNITARI}} & {\color[HTML]{000000} \textbf{UNITATS}} & {\color[HTML]{000000} \textbf{COST}}\\ 
\hline
Impressions a paper                              & 0,03\euro /full                                           & 800 fulls                                             & 24,00\euro\\
Electricitat                                 & 0,20\euro /kWh                                           & 225 kWh                                           & 45,00\euro\\
Amortització portàtil Lenovo G-50            & 450,00\euro /portàtil                                           & 0,15 (amortitzat)                                             & 67,50\euro   \\
Software                     & 0,00\euro /mes                                           & 5 mesos                                             & 0,00\euro \\                                       
\hline
\textbf{Total}                               &                              &                                & \textbf{136,50\euro}       \\
\hline                      
\end{tabular}}
\caption{Costos indirectes i amortitzacions}
\end{table}


\noindent \textbf{\large Contingències}\\

\noindent Afegirem, en base als costos directes i indirectes prèviament desglosats, un 15\% sobre el total en concepte de contingències.

\begin{table}[htb]
\centering
\label{PressupostContingencies}
%\resizebox{\textwidth}{!}{%
\begin{tabular}{lrrrrr}
\hline \textbf{CONCEPTE}                           & {\color[HTML]{000000} \textbf{COST BASE}} & {\color[HTML]{000000} \textbf{\% CONTINGÈNCIES}} & {\color[HTML]{000000} \textbf{TOTAL}}\\ 
\hline
Costos directes                             & 8735,00\euro /full                                           & 15                                             & 1310,25\euro\\
Costos indirectes                                 & 136,50\euro /kWh                                           & 15                                           & 20,475\euro\\
\hline
\textbf{Total}                               &                              &                                & \textbf{1330,725\euro}       \\
\hline                      
\end{tabular}%}
\caption{Contingències}
\end{table}

\noindent \textbf{\large Imprevistos}\\

\noindent Identificarem 2 imprevistos en base a la planificació:

\begin{itemize}
\item \textbf{Prolongació de les hores / ampliació del termini}. Contemplarem la possibilitat de requerir més temps de l’estimat per acabar el projecte, considerant una desviació de fins a 50 hores (que podríem considerar la dedicació aproximada de dues setmanes a mitja jornada). Considerarem una probabilitat raonable del 20%.
\item \textbf{Avaria de hardware}. Problemes en l’ús del material hardware (en aquest cas, exclusivament el portàtil). Considerarem una probabilitat més remota, del 5\%, i el pitjor dels casos, que equivaldria a la substitució total del cost del portàtil.
\end{itemize}

\begin{table}[htb]
\centering
\label{PressupostImprevistos}
\resizebox{\textwidth}{!}{%
\begin{tabular}{lrrrrr}
\hline \textbf{CONCEPTE}                           &  \textbf{COST UNITARI} &  \textbf{\ UNITATS} & \textbf{COST TOTAL} & \textbf{\% PROBABILITAT} &  \textbf{TOTAL} \\
\hline
Ampliació del termini                             & 12,00\euro /hora                                           & 50 hores                                             & 600,00\euro & 20 & 120,00\euro \\
Avaria del hardware                                 & 450,00\euro /ordinador & 1 ordinador                                           & 450,00\euro & 5 & 22,50\euro \\
\hline
\textbf{Total}                               &                              &                                & & & \textbf{142,50\euro}       \\
\hline                      
\end{tabular}}
\caption{Imprevistos}
\end{table}

\noindent \textbf{\large Pressupost global}\\

\noindent Un cop valorat costos indirectes, costos indirectes i amortitzacions, contingències i possibles imprevistos, podem donar una versió completa de l’estimació pressupostària per la realització del projecte.

\begin{table}[htb]
\centering
\label{PressupostGlobal}
%\resizebox{\textwidth}{!}
\caption{Resum global del pressupost}
\end{table}

Per tant, el pressupost final és de \textbf{9867,475\euro}.

\subsection{Control de gestió}

El pressupost exposat ja inclou com a part de la partida destinada una part generada per imprevistos i desviacions amb possibilitats de produir-se i que pretenen precisament realitzar un control i manteniment de la gestió i evolució del projecte i els recursos (veure apartats 2.1.3. i 2.1.4.).\\

Tot i així, podem considerar oportú establir uns mecanismes, o tasques específiques, dedicades al control periòdic que permetin fer un seguiment de l’activitat de gestió de projecte i, per tant, no només preveure de forma teòrica aspectes com desviacions pressupostàries, sinó detectar al llarg de l’evolució del projecte quan això succeeixi.\\

Per fer-ho es proposa realitzar un control de desviacions durant la transició de fases; és a dir, treballarem amb un model de plantilla que ens calculi una sèrie de desviacions (en funció de diversos criteris), que al finalitzar cada fase ens permeti obtenir un feedback objectiu i ràpid de possibles desviacions respecte al pressupost inicial.\\

Per fer-ho, i basant-nos en la bibliografia utilitzada a GEP, farem servir els següents indicadors:

\begin{itemize}
\item \textbf{Desviament de mà d’obra en preu} = (cost estimat - cost real) * consum hores real
\item \textbf{Desviament en la realització d’una tasca en consum} = (consum estimat - consum real) * cost real
\item \textbf{Desviament total en la realització de tasques} = cost total estimat tasca - cost total real tasca
\item \textbf{Desviament total de despeses fixes} = total costos fixes pressupostat - total costos fixes real
\end{itemize}

Considerarem aquests 4 indicadors, ja que ens seran els més útils per detectar desviacions, p.e., en quant a la realització de les activitats descrites al Gantt, en termes tant de dedicació en quantitat d’hores total com per tasca, i també en aspectes com despeses fixes (p.e. amortitzacions). Mitjançant la comprovació d’aquests indicadors, podrem veure el grau de desviament i actuar en conseqüència segons les necessitats. 

\subsection{Sostenibilitat econòmica, social i ambiental}

Procedim a fer un anàlisi de les 3 dimensions de la sostenibilitat en referència a aquest projecte, per posteriorment avaluar fent servir la matriu de sostenibilitat el grau de satisfacció d’aquest àmbit en funció dels criteris establerts.\\

\noindent \textbf{\large Dimensió econòmica}\\

\noindent La dimensió econòmica està satisfactòriament treballada gràcies al pressupost prèviament exposat, basat en dades objectives i específiques (p.e., activitats reals a realitzar durant el projecte), que inclou despeses materials i humanes. Aquests costos i temps de dedicació inclouen aspectes crítics tals com possibles desviacions i imprevistos, i una assignació proporcional dels recursos assignats a la rellevància de cada tasca. Es tracta d’un projecte realitzat amb el cost mínim, però suficient (tenint en compte sempre que és necessari afegir extres per desviacions), garantint la seva satisfacció però sense despeses innecessàries, el que garanteix la seva viabilitat econòmica.\\

\noindent \textbf{\large Dimensió social}\\

\noindent L’objectiu principal del treball és aprofundir en el control de qualitat i monitoritatge de sistemes software mitjançant el desenvolupament de software lliure reaprofitable. Tal i com vam veure a l’entregable 1 (a l’apartat Estat de l’art), existeix marge d’investigació i treball en aquest àmbit, i l’àmplia gama de serveis software poden extreure un cert benefici en base als avenços (o si més no, la recerca i síntesi) que aquest projecte pugui aportar. Des del punt de vista dels desenvolupadors (usuaris reals d’aquest projecte, ja que seran els que l’utilitzaran), aportem noves eines i facilitem criteris d’autoadaptabilitat per monitors de control de qualitat. També, però, tindrà conseqüències en els usuaris dels sistemes monitorats, ja que la recol·lecció de dades d’aquests està orientada a la millora de la qualitat dels serveis oferts per aquests sistemes. Aquest és un aspecte que cada vegada requereix més profunditat, motiu pel qual podem considerar l’existència d’una necessitat dins el mercat actual. Podem considerar que no existeixen col·lectius afectats negativament.\\

\noindent \textbf{\large Dimensió ambiental}\\

\noindent Els recursos plantejats tant pel desenvolupament del projecte com el consum necessari per la seva vida útil són mínims, i inclouen únicament aquells derivats del manteniment d’un sistema software. No existirà contaminació destacada més allà de la generada pel consum d’electricitat del dispositiu portàtil a utilitzar pel desenvolupament del projecte o la impressió de papers (que es limitarà al mínim necessari). Es tracta, a més, d’un projecte que té per objectiu ser reaprofitat per tercers projectes (plantejant estructures, arquitectures, monitors, etc. reaprofitables).\\

\noindent \textbf{\large Matriu de sostenibilitat}\\

\noindent En base als anteriors criteris establerts, assignarem les següents puntuacions a la matriu de sostenibilitat, considerant únicament la Planificació per cadascuna de les 3 dimensions:

\begin{table}[htb]
\centering
\label{MatriuSostenibilitat}
%\resizebox{\textwidth}{!}
\caption{Matriu de sostenibilitat}
\end{table}

Podem considerar, juntament amb la informació prèviament exposada, les següents justificacions:

\begin{itemize}
\item \textbf{Dimensió econòmica}. S’assoleixen satisfactòriament criteris econòmics amb rigor i detall (basat en pressupost) i es presenta informació verídica en quant a costos, optimitzats per un ajustament adequat.
\item \textbf{Dimensió social}. Tot i que l’impacte pot no ser especialment destacable, sí que té un mercat profitós i aporta beneficis dins el seu sector que el fan un projecte positiu des del punt de vista social.
\item \textbf{Dimensió ambiental}. No aporta un benefici destacable directe però sí que assoleix els criteris d’eficiència ambiental en quant al consum de recursos o l’empremta ecològica del projecte, que podrà ser reutilitzat per projectes tercers.
\end{itemize} 
% Chapter Template

\chapter{Anàlisi de requisits} % Main chapter title

\label{AnalisiRequisits} % Change X to a consecutive number; for referencing this chapter elsewhere, use \ref{ChapterX}

Definits els objectius i l'abast del nostre projecte, és necessari procedir a traduïr aquests en requisits específics que el nostre sistema ha de satisfer. I que, en conseqüència, guiaran les posteriors tasques de disseny i implementació dels diferents components a desenvolupar.

\section{Visió general del sistema}

De forma prèvia a la identificació dels requisits, i a partir de la informació prèviament exposada, podem presentar una breu visió general del nostre sistema. En aquesta part no es presentaran detalls més enllà de la naturalesa, objectius i funcionalitats generals del sistema i els seus components, ja que aquests es desenvoluparan més endavant, un cop els requisits estiguin definits.\\

En primer lloc, establim de nou la premisa d'aquest projecte: el \textbf{disseny}, la \textbf{implementació} i \textbf{validació} d'un sistema de \textbf{monitoratge} que satisfaci les característiques d'\textbf{adaptabilitat}, \textbf{heterogeneïtat} i \textbf{distribució} (característiques explicades al \textit{Capítol 3. Objectius}). En base al context del projecte SUPERSEDE (presentat al \textit{Capítol 2. Contextualització}), i segons aquest objectiu, el nostre sistema haurà d'incloure dues vessants:

\begin{itemize}
\item Un \textbf{sistema de monitoratge} de serveis i components software tercers.
\item Un \textbf{sistema d'adaptabilitat} que permeti adaptar l'activitat del sistema de monitoratge.
\end{itemize}

El component clau de l'activitat del monitoratge és el que anomenem \textbf{monitor}. Un monitor no és més que un component software (independentment de la seva naturalesa o la tecnologia amb la qual està desenvolupat) que interactua amb un component software i col·lecciona informació relacionada amb la seva activitat, tal i com s'explica al \textit{Capítol 2.2. Estat de l'art}. Per tal de generar un sistema de monitoratge dins el nostre projecte, haurem de tenir en compte diversos factors.\\

En primer lloc, necessitarem definir una \textbf{arquitectura genèrica} que ens permeti definir l'estructura i arquitectura bàsica dels monitors que inclourem al nostre projecte. D'aquesta manera, mitjançant criteris que s'estudiaran més endavant, el nostre sistema disposarà d'un component genèric a partir del qual podrem generar \textbf{monitors específics}, independentment de la seva activitat en termes específics. Així, garantit la característica d'\textbf{heterogeneïtat}, el nostre sistema permetrà la seva extensió mitjançant la implementació de nous monitors que es puguin integrar al sistema.\\

En segon lloc, haurem de considerar per una banda que aquests monitors han de ser components independents que es puguin desplegar de forma distribuïda i que la seva activitat pugui actuar com a unitat per sí mateixa. Per altra banda, per gestionar la integració del nostre sistema, necessitarem definir components que \textbf{integri} aquest conjunt de monitors en un únic punt i sigui capaç de gestionar l'activitat dels mateixos.\\

Paral·lelament al sistema de monitoratge, necessitem dissenyar i implementar una part del sistema que \textbf{gestioni les configuracions dels monitors} (és a dir, les diferents activitats de monitoratge) i pugui gestionar les adaptacions sobre els monitors. Per gestionar tot aquest subdomini del projecte, s'utilitzaran un \textbf{conjunt de models UML} amb els quals es modelaran tots els detalls relacionats amb les configuracions i les adaptacions dels monitors: configuracions actuals, propostes de noves configuracions, detalls sobre mecanismes d'adaptacions, etc. Mitjançant aquest conjunt de models, que més endavant es detallaran, el sistema podrà \textbf{computar i aplicar de forma automàtica adaptacions} sobre els monitors desplegats. Per garantir el funcionament i la validació del sistema, caldrà que aquests dos subcomponents estiguin integrats i es puguin comunicar entre ells, seguint els criteris d'adaptació.\\

Finalment, com a tasca complementària, el nostre sistema inclourà un \textit{dashboard} consultor que permeti visualitza les diferents adaptacions que el sistema realitza sobre els monitors, per tal de poder observar i validar l'activitat del sistema d'acord amb els requisits establerts.

%-----------------------------------
%	SUBSECTION 1
%-----------------------------------
\section{Requisits}

Per formalitzar els conceptes prèviament exposats, definirem una sèrie de requisits prou genèrics que ens serviran per guiar el desenvolupament del nostre sistema. Els detalls d'aquests requisits s'aniran exposant a mesura que es presenti la recerca i el treball realitzats.

Classificarem aquest requisits en 3 tipus: \textbf{funcionals}, \textbf{arquitecturals} i \textbf{de qualitat}.

\subsection{Funcionals}

\begin{itemize}
\item[\textbf{RF-1}] \textbf{Alta d'una nova activitat de monitoratge.} El sistema ha de permetre inicialitzar un nou procés de monitoratge en un dels monitors integrats. Aquesta activitat ha de ser configurable segons els paràmetres de configuració defintis pel propi monitor.
\item[\textbf{RF-2}] \textbf{Baixa d'una activitat de monitoratge.} Donada una activitat de monitoratge existent per un monitor específic, el sistema ha de permetre aturar aquesta activitat.
\item[\textbf{RF-3}] \textbf{Modificació d'una activitat de monitoratge.} Donada una activitat de monitoratge existent per un monitor específic, el sistema ha de permetre modificar els paràmetres de configuració de l'activitat de monitoratge del monitor.
\item[\textbf{RF-4}] \textbf{Consulta d'una activitat de monitoratge.} Donada una activitat de monitoratge existent per un monitor específic, el sistema ha de permetre consultar les dades i paràmetres de configuració d'aquella activitat de monitoratge.
\item[\textbf{RF-5}] \textbf{•}
\end{itemize}

\subsection{Arquitecturals}

\begin{itemize}
\item[\textbf{RA-1}] \textbf{Arquitectura genèrica per monitors extensible.} L'arquitectura proposada per la implementació dels monitors ha de ser genèrica, independent de la seva activitat, i extensible per qualsevol tipus de monitor.
\item[\textbf{RA-2}] \textbf{Integració dels monitors.} El sistema ha d'integrar els diversos monitors mitjançant un únic punt d'entrada i garantir, mitjançant l'arquitectura genèrica, la operabilitat amb els mateixos.
\item[\textbf{RA-3}] \textbf{}
\end{itemize}

\subsection{De Qualitat}




% Chapter Template

\chapter{Eines de desenvolupament} % Main chapter title

\label{EinesDesenvolupament} % Change X to a consecutive number; for referencing this chapter elsewhere, use \ref{ChapterX}

%----------------------------------------------------------------------------------------
%	SECTION 1
%----------------------------------------------------------------------------------------

\section{Tecnologies utilitzades}

Lorem ipsum dolor sit amet, consectetur adipiscing elit. Aliquam ultricies lacinia euismod. Nam tempus risus in dolor rhoncus in interdum enim tincidunt. Donec vel nunc neque. In condimentum ullamcorper quam non consequat. Fusce sagittis tempor feugiat. Fusce magna erat, molestie eu convallis ut, tempus sed arcu. Quisque molestie, ante a tincidunt ullamcorper, sapien enim dignissim lacus, in semper nibh erat lobortis purus. Integer dapibus ligula ac risus convallis pellentesque.

%-----------------------------------
%	SUBSECTION 1
%-----------------------------------
\section{Entorn de configuració}

Nunc posuere quam at lectus tristique eu ultrices augue venenatis. Vestibulum ante ipsum primis in faucibus orci luctus et ultrices posuere cubilia Curae; Aliquam erat volutpat. Vivamus sodales tortor eget quam adipiscing in vulputate ante ullamcorper. Sed eros ante, lacinia et sollicitudin et, aliquam sit amet augue. In hac habitasse platea dictumst.

% Chapter Template

\chapter{Sistema de monitoratge} % Main chapter title

\label{SistemaMonitoratge} % Change X to a consecutive number; for referencing this chapter elsewhere, use \ref{ChapterX}

En aquest punt tenim la base necessària per procedir a exposar el treball realitzat des d'un punt de vista de disseny software i implementació dels diferents components. Agafant com a referència la solució proposada al \textit{Capítol 5. Visió general del sistema}, procedirem a desenvolupar els detalls tècnics de cadascun dels components que integren aquest sistema. Començarem per explicar els detalls relacionats amb el disseny i la implementació dels monitors.

\section{Monitors}

Recordem que, dins el nostre context, un \textbf{monitor} consisteix en un \textbf{component software} autònom amb una activitat regular orientada al control de qualitat d'un altre sistema software. Aquest control de qualitat es basa en una col·lecció de dades obtingudes a través d'aquest segon sistema, que ens aporten informació pròpia del context monitorat. Amb aquestes dades, un sistema capacitat per processar i analitzar aquestes dades, es poden generar suggerències de modificacions. Aquesta darrera part, però, queda fora de l'abast del nostre projecte, que centrarà l'activitat dels monitors en la seva tasca principal: la col·lecció de dades sota una sèrie de criteris específics.\\

En general, per tant, volem que el nostre sistema disposi d'un conjunt de monitors heterogenis (i, per tant, de naturalesa i comportament diferents), que puguin ser capaços de gestionar \textbf{processos de monitoratge} de forma paral·lela. És a dir: cadascun d'aquests monitors ha de ser capaç d'inicialitzar processos de monitoratge que s'executin en paral·lel i en segon pla, col·lectin dades d'acord als criteris de cadascun d'aquests processos, i les redireccionin a un tercer component software, encarregat del seu anàlisi.Per tant, necessitem que cadascun d'aquests monitors satisfaci els següents requisits funcionals:

\begin{enumerate}
\item \textbf{Inicialització de procés de monitoratge.} El monitor ha de poder rebre una petició per inicialitzar un nou procés de monitoratge amb una sèrie de paràmetres de configuració que defineixin aquest procés de monitoratge.
\item \textbf{Modificació de procés de monitoratge.} Donat un procés de monitoratge ja existent, el sistema ha de permetre la seva reconfiguració. És a dir: el sistema ha de permetre modificar els paràmetres d'aquest procés i, en conseqüència, alterar el comportament del procés (d'acord amb els criteris que es presenten a continuació.
\item \textbf{Aturada de procés de monitoratge.} Donat un procés de monitoratge ja existent, el sistema ha de permetre la seva aturada. Davant aquesta petició, el procés s'atura, i per tant es deixen de recol·lectar dades sota aquells criteris de monitoratge.
\end{enumerate}

\subsection{Especificacions tècniques}

Tal i com establiem com a objectiu principal del projecte, aquest sistema de monitoratge ha de ser \textbf{heterogeni}. La conseqüència principal d'aquesta característica és que necessitem definir un sistema que contempli que cadascun d'aquests requisits funcionals es garanteixen en la integració dels monitors implementats, i que per tant esdevenen casos d'ús complets i satisfactoris del nostre context. Per fer-ho, i donat que els monitors seran el principal punt de variabilitat del nostre sistema, hem d'afegir un cert nivell d'abstracció, un \textbf{desacoblament} entre els detalls específics de cadascun dels monitors, que ens resulten indiferents per la resta del sistema.\\

Per tant, el primer pas que hem de realitzar és \textbf{dissenyar una arquitectura} i uns \textbf{criteris de configuració genèrics} que satisfacin dos criteris: primerament, que ens permetin integrar tots els monitors implementats sota aquesta proposta al nostre sistema; i en segon lloc, que permetin una independència suficient com per garantir el criteri d'heterogeneïtat dels monitors. Desenvoluparem aquesta proposta analitzant els següents punts:

\begin{enumerate}
\item \textbf{Comportament intern del monitor.} Anàlisi de les necessitats i detalls tècnics del funcionament intern del procés de monitoratge.
\item \textbf{Redirecció de dades col·lectades.} Especificacions tècniques del mètode de gestió i redirecció de dades.
\item \textbf{Configuració dels monitors.} D'acord amb les necessitats anteriors, descriure quins paràmetres necessitarem per configurar els monitors.
\end{enumerate}

\subsubsection{Comportament intern del monitor}

La implementació d'un monitor representa, des d'un punt de vista semàntic, un component software encarregat de monitorar un component software concret. En aquest context específic, entendrem \textbf{monitorar} com la col·lecció periòdica d'un conjunt de dades produïdes de l'execució del sistema software monitorat. En base a aquesta definició, entendrem com a \textbf{procés de monitoratge} el cicle següent:

\begin{enumerate}
\item Inicialització de les estructures de col·lecció de dades
\item Captura de dades durant el transcurs d'un període de temps determinat
\item Enviament de les dades a un tercer component software
\end{enumerate}

Així, de forma periòdica, un procés de monitoratge recull durant un període de temps (o \textbf{\textit{time slot}}) específic totes les dades que el monitor ha estat configurat per recollir. Per tal que els monitors es puguin explotar al màxim, és imprescindible que aquests permetin l'\textbf{execució en paral·lel} de diversos processos de monitoratge, amb possibles diferències en els seus paràmetres de configuració (p.e., aquest \textit{time slot}).\\

Davant aquesta proposta, el comportament i potencial d'un monitor queda molt limitat, ja que elements com p.e. el mètode de recollida de dades, o fins i tot les dades recollides, queden molt limitats. En general, és molt possible que un sistema software pugui ser monitorat mitjançant diferents tècniques, com per exemple l'ús d'APIs, llibreries o components externs, etc. Per aquesta raó, si volem permetre que el nostre monitor ofereixi flexibilitat en aquest aspecte, hem de permetre l'ús de diferents eines, o \textbf{\textit{tools}}, que aquest monitor pot utilitzar indistintament per executar els processos de monitoratge.\\

La integració de diverses \textit{tools} dins un monitor ens permeten no només una variabilitat en l'execució de la col·lecció de dades, sinó també una major fiabilitat i qualitat del monitor com a component software. Ens permet reaccionar, entre d'altres, davant escenaris on l'ús d'un sistema de monitoratge específic deixa de funcionar (p.e. una API que no dona resposta), ja que davant la detecció d'aquest error el canvi de \textit{tool} utilitzada ens permet que el monitor no deixi de ser usable.\\

En resum, necessitem que el monitor sigui capaç de gestionar un nombre indefinit de processos de monitoratge en paral·lel, amb configuracions diferents, i que utilitzin el conjunt de \textit{tools} implementades.

\subsubsection{Redirecció de dades col·lectades}

Un dels objectius de les especificacions tècniques dels monitors és permetre la seva integració, en primer lloc, dins el context del nostre projecte, i en segon lloc, a sistemes tercers que permetin l'anàlisi de les dades recollides durant la seva activitat de monitoratge. D'aquesta manera augmentem el valor propi dels components dissenyats en el projecte, reutilitzable en contexts diferents al plantejat. Per aquesta raó, i per completar l'activitat del monitor, hem de contemplar com dissenyar l'enviament i redirecció de les dades que cada monitor recull i formata durant la seva activitat. \\

Per facilitar aquest aspecte, i permetre també la seva integració dins el sistema general de SUPERSEDE, els monitors integraran la implementació d'enviament de les seves dades a través d'\textbf{Apache Kafka.} Es tracta d'una plataforma distribuïda de \textit{streaming} que ofereix la possibilitat de crear i configurar \textit{pipelines} de dades en temps real que actuen com a canal de comunicació entre diverses aplicacions o components software. L'arquitectura és senzilla: un component software, anomenat \textbf{\textit{producer}}, es comunica amb el servidor Kafka i envia dades de forma periòdica a un \textit{pipeline} específic d'aquest servidor, prèviament configurat i identificament amb el que anomenem Kafka \textit{topic}, un identificador únic d'aquell \textit{pipeline} per aquell desplegament de Kafka. Aquest flux de dades s'encua al servidor, i es redireccionen a uns altres sistemes o aplicacions, anomenats \textbf{\textit{consumers}}, que reben i processen les dades d'un pipeline específic a mesura que es van enviant i processant.\\

\begin{figure}
\centering
\includegraphics[width=11cm]{Figures/Figure7}
\decoRule
\caption[Exemple d'arquitectura de la comunicació amb Kafka]{Exemple d'arquitectura de la comunicació amb Kafka}
\label{fig:Kafka}
\end{figure}

En general, l'avantatge principal de Kafka i la justificació del seu ús pel nostre context és que permet una integració còmode i fiable entre diferents aplicacions que necessiten comunicar dades de forma periòdica, garantint la seva arribada. Kafka ofereix una sèrie d'APIs per configurar els \textit{producers} i \textit{consumers}, així com una configuració relativament senzilla del propi servidor. Gràcies a aquestes característiques, podem aprofitar els propis monitors per actuar com a \textit{producers} d'un \textit{stream} de dades, que es correspondrà amb les dades monitorades durant els processos d'execució, i distribuïr-les als diferents \textit{pipelines} o Kafka \textit{topics}. Així, davant possibles ampliacions i expansions d'aquest projecte, podem fàcilment incorporar components d'anàlisi gràcies al desacoblament entre la lògica interna del monitor i l'enviament i captura de dades que l'arquitectura de Kafka ens ofereix.\\

La lògica interna genèrica proposada per la implementació dels monitors serà, per tant, l'ús de l'API de \textit{producer} de Kafka per part dels monitors, pel qual cada procés de monitoratge enviarà de forma periòdica dades a un servidor Kafka específic, o Kafka \textit{endpoint}, i dins aquest desplegament, a un \textit{pipeline} o Kafka \textit{topic} específic. A la figura ~\ref{fig:Kafka} s'exposa com funciona aquesta comunicació entre els monitors i els \textit{consumers}, components que poden ser de qualsevol tipus sempre i quant implementin la lògica de \textit{consumers} pròpia e Kafka.

\subsubsection{Paràmetres de configuració}

Davant les especificacions anteriors, i per garantir el màxim nivell de personalització i configuració dels monitors, cadascun dels processos de monitoratge actius en un monitor ha de permetre definir els paràmetres relacionats amb les possibles variacions i diferències entre aquests processos, tant en temps de creació com durant la seva reconfiguració. En aquest sentit, es proposen els següents paràmetres com a genèrics per a totes les implementacions de monitors:

\begin{itemize}
\item \textbf{\textit{Time slot}}. Expressat en segons, indica la durada de cada període de monitoratge de dades (és a dir, temps que transcorre cada vegada que s'envia un nou \textit{stream} de dades).
\item \textbf{\textit{Tool name}}. Nom de l'eina (\textit{tool}) utilitzada per aquell procés de monitoratge, i que per tant implica la col·lecció d'unes dades específiques utilitzant una tècnica específica.
\item \textbf{\textit{Kafka endpoint}}. Adreça que apunta al servidor on es troba desplegat el sistema Kafka on s'han d'enviar les dades generades, ja sigui \textit{localhost} o URL pública.
\item \textbf{\textit{Kafka topic}}. Identifica el \textit{pipeline} de dades del servidor Kafka on el monitor (\textit{producer} dins el context de Kafka) ha d'enviar les dades.
\end{itemize}

És possible, tal i com veurem més endavant, que alguns monitors requereixin de paràmetres de configuració addicionals, propis del funcionament intern específic del monitor. Per aquest motiu, a banda de proporcionar una configuració bàsica pels monitors amb els paràmetres anteriors, cal permetre també una extensibilitat en quant a paràmetres de configuració.

\subsection{Arquitectura genèrica}

\begin{figure}
\centering
\includegraphics[width=14cm]{Figures/Figure5}
\decoRule
\caption[Arquitectura software genèrica d'un monitor]{Arquitectura software genèrica d'un monitor}
\label{fig:Figura5}
\end{figure}

Es proposa l'arquitectura definida a la figura ~\ref{fig:Figura5} a extendre per cadascuna de les implementacions de monitors. Per entendre aquesta arquitectura, a continuació s'expliquen cadascun dels elements integrats:

\begin{itemize}
\item \textbf{MonitoringParams}. Classe abstracta que cada monitor ha d'implementar que conté, de base, els paràmetres de configuració dels monitors genèrics per tots aquests. Addicionalment, cada implementació de monitor pot afegir els paràmetres i la lògica associada a aquests que consideri oportuns.
\item \textbf{ParserConfiguration}. Interfície que cada monitor ha d'implementar i que defineix un mètode per transformar un objecte JSON en una instància de la classe \textit{MonitoringParams} implementada pel propi monitor. L'objectiu és permetre així que el monitor pugui processar JSON com a format de comunicació estàndar per les peticions de configuracions, facilitant el seu ús desplegat com a servei web (especialment útil per l'Integrated Framework dins el context SUPERSEDE, tal i com s'explica al \textit{Capítol 5. Visió general del sistema}).
\item \textbf{ToolInterface}. Interfície parametritzada amb una especialització de la classe \textit{MonitoringParams} que defineix una instància de procés de monitoratge per una \textit{tool} específica. Defineix els següents mètodes:
\begin{itemize}
\item \textbf{\textit{addConfiguration(T)}} -> inicialitza el procés de monitoratge amb els paràmetres definits per la instància de T (subclasse de MonitoringParams)
\item \textbf{\textit{deleteConfiguration()}} -> atura el procés de monitoratge i elimina la instància de la \textit{tool}
\item \textbf{\textit{updateConfiguration(T)}} -> actualitza els paràmetres de configuració del procés iniciat en segon pla per la instància de la \textit{tool}
\end{itemize}
\item \textbf{ToolDispatcher}. Classe que actua com a controlador del monitor, rebent totes les peticions relacionades amb els processos de monitoratge i gestionant les diferents instàncies en execució. Defineix un \textit{ParserConfiguration} per processar la traducció de JSON (format estàndar) a \textit{MonitoringParams} pels següents mètodes:
\begin{itemize}
\item \textbf{\textit{addConfiguration(JSONObject)}} -> processa els paràmetres definits al JSONObject i inicialitza una instància de la \textit{tool} corresponent amb els paràmetres associats
\item \textbf{\textit{deleteConfiguration(int)}} -> atura el procés de monitoratge identificat per l'id proporcionat
\item \textbf{\textit{updateConfiguration(JSONObject, int)}} -> actualitza els paràmetres de configuració definits al JSONObject del procés de monitoratge identificat per int
\end{itemize}
Per tal de gestionar la \textit{tool} utilitzada en cada procés de monitoratge (que representarà una nova instància de la implementació de la interfície \textit{ToolInterface}) utilitzant el paràmetre \textit{toolName} de configuració, s'utilitza el patró \textit{reflection}. Aquest patró de disseny software es caracteritza per la modificació en temps d'execució del comportament d'un sistema; i, en el nostre cas, ens interessa per permetre  la instanciació de les diferents \textit{tools} utilitzant el seu nom sense necessitat de coneixe'l. Mitjançant l'ús del nom de la tool, i coneixent el \textit{package} on es troben implementades les tools, podem instanciar aquella tool partint únicament del nom, sense necessitat de cap altre tipus de context.
\item \textbf{MonitoringData}. Interfície que cada monitor implementa amb les dades i format que el monitor genera fruït de la seva activitat, amb la implementació d'un mètode \textit{toJsonObject()} per definir un format genèric de les dades a enviar
\item \textbf{KafkaCommunication}. Classe que implementa la comunicació amb el servidor de Kafka, i que permet enviar les dades generades per l'activitat de monitoratge. Implementa 4 mètodes segons les dues lògiques de comunicació possibles: la integrada al sistema SUPERSEDE (utilitzant \textit{proxies} de IF), i una configuració personalitzable a un Kafka endpoint específic:
\begin{itemize}
\item \textbf{\textit{initProxy()}} -> inicialitza la instància de \textit{proxy} que permet la comunicació amb el Kafka \textit{server} desplegat a IF
\item \textbf{\textit{generateResponseIF(List<MonitoringData>)}} -> envia el llistat d'instàncies de dades monitorades a través del \textit{proxy} inicialitzat
\item \textbf{\textit{initProducer(String)}} -> inicialitza un \textit{producer} de Kafka que es comunica amb l'\textit{endpoint} especificat
\item \textbf{\textit{generateResponseKafka(List<MonitoringData>)}} -> envia el llistat d'instàncies de dades monitorades a través del \textit{producer} inicialitzat
\end{itemize}
\end{itemize}

Aquesta és per tant la proposta d'una arquitectura genèrica per la implementació de cadascun dels monitors a integrar en el nostre sistema. Les especificacions tècniques tant des d'un punt de vista de requisits funcionals com requisits arquitectònics o de qualitat queden garantides, així com la seva extensibilitat d'acord amb les característiques específiques necessàries de cada monitor. La proposta genèrica s'implementa en un projecte del qual les implementacions de monitors específics han d'estendre com a subprojecte, utilitzant les funcionalitats de Gradle a tal efecte.

\subsection{Implementació de monitors}

Un cop exposada l'arquitectura i especificacions genèriques dels monitors, el següent pas és procedir a la implementació dels diferents monitors que integrarem al nostre sistema. Aquestes implementacions esdevindran possibles casos d'ús a executar, però únicament suposen exemples dins d'un ample ventall de possibilitats d'implementació.\\

En aquest projecte presentem la implementació de 3 monitors classificats en 2 tipus de monitors diferents: \textbf{monitors de xarxes socials} i \textbf{monitors de botigues d'aplicacions}. Pel primer tipus, es proposa un monitor de la xarxa social \textbf{Twitter}. Pel segon tipus, es proposen dos monitors: un monitor de \textbf{Google Play} i un altre de l'\textbf{App Store}. 

\subsubsection{Twitter Monitor}

L'objectiu d'aquest monitor és supervisar i recol·lectar els tuits publicats a la xarxa social de Twitter durant un període de temps concret (definit al procés de monitoratge) que compleixen una sèrie de característiques específiques, d'acord amb els criteris que poden resultar d'interès en quant a la informació d'aquests tuits. \\

Per permetre aquesta activitat de monitoratge, i tal i com procedirem amb cadascun dels monitors, necessitem implementar com a mínim una \textit{tool} amb la qual realitzar el procés de monitoratge. Definirem una \textit{tool} anomenada \textbf{TwitterAPI} que utilitzarà \textbf{twitter4j}, una llibreria lliure no oficial de Java que permet utilitzar una arquitectura ja definida per connectar-se a les APIs de Twitter, i entre d'altres a la \textbf{Stream API}, una API que permet obrir \textit{streams} de dades per capturar i processar tots els tuits publicats en temps real que compleixin una sèrie de característiques específiques. Aquests \textit{streams} s'executaran com a \textit{threads} en segon pla i en paral·lel d'acord amb el nº de processos de monitoratge oberts.\\

Pel nostre cas, definirem dos paràmetres per filtrar els tuits monitorats: l'\textbf{autor} del tuit i l'aparició d'un \textbf{conjunt de paraules} específic al contingut del tuit. Per configurar cadascun dels processos de monitoratge d'acord a aquests dos criteris, caldrà afegir els següents paràmetres:

\begin{itemize}
\item \textbf{accounts} - llistat amb els identificadors dels autors dels quals volem obtenir els tuits 
\item \textbf{keywordExpression} - expressió booleana formada per combinacions AND, OR i NOT de diferents \textit{keywords} que el contingut del tuit ha de satisfer per ser monitorat
\end{itemize}

La \textit{Stream API} ofereix paràmetres de configuració per realitzar el \textit{tracking} dels tuits que satisfan ambdues condicions. En el cas del primer paràmetre \textit{accounts} no cal realitzar cap transformació, ja que únicament necessitem els noms únics de les comptes dels autors per obtenir els seus tuits. Contràriament, l'API únicament ofereix l'oportunitat de filtrar per combinacions de paraules expressades en la seva \textit{forma normal disjuntiva}, o \textbf{FND}. És a dir, expressions del format:\\

\centerline{\textit{$X_{1} \,\, OR \,\, X_{2} \,\, OR \,\, .. \,\, OR \,\, X_{n}$}}\bigskip

\noindent
on $X_{n}$ és una expressió booleana formada per una combinació indefinida d'operands AND:\\

\centerline{\textit{$keyword_{1} \,\, AND \,\, keyword_{2} \,\, AND \,\, .. \,\, AND \,\, keyword_{n}$}}\bigskip

Davant aquest fet, i per evitar forçar un format específic del paràmetre de configuració d'entrada, necessitem implementar una lògica interna pròpia de la \textit{tool} TwitterAPI que transformi qualsevol expressió booleana en la seva expressió FND. \\

\begin{figure}
\centering
\includegraphics[width=14cm]{Figures/Figure6}
\decoRule
\caption[Arquitectura software del monitor de Twitter]{Arquitectura software del monitor de Twitter}
\label{fig:Figura6}
\end{figure}

Donats aquests detalls podem utilitzar l'arquitectura genèrica proposada anteriorment per extendre la implementació del monitor de Twitter. Aquesta arquitectura i els components implementats es troben definits a la figura ~\ref{fig:Figura6}, on podem veure aquells components reutilitzats de l'arquitectura genèrica remarcats en groc, per poder fer una comparativa ràpida sobre les classes i components que ha calgut implementar per poder definir un monitor:

\begin{itemize}
\item \textbf{TwitterMonitoringParams.} Subclasse de la classe abstracta \textit{MonitoringParams} que hereta per una banda els atributs comuns a tots els monitors (\textit{kafkaEndpoint, toolName, kafkaTopic} i \textit{timeSlot}) i per altra banda en defineix dos nous: la \textit{keywordExpression} i un llistat d'\textit{accounts}.
\item \textbf{TwitterParserConfiguration.} Implementació de la interfície que defineix la transformació del format d'entrada del monitor (JSONObject) a la instància de \textit{TwitterMonitoringParams} amb la qual el monitor (i en conseqüència la seva \textit{tool}) treballarà.
\item \textbf{TwitterMonitoringData.} Implementació de la interfície que defineix la transformació de les dades recollides durant un cicle del procés de monitoratge d'una \textit{tool} al format de sortida del monitor (JSONObject).
\item \textbf{TwitterAPI.} Implementació de la interfície que defineix la lògica de les tres operacions de qualsevol \textit{tool}: iniciar una nova configuració, modificar-ne una d'existent, i eliminar-ne una d'existent. En aquest cas disposem d'una única implementació, però tal i com veurem més endavant, l'arquitectura permet sense problema generar un conjunt de \textit{tools} diferenciades, cadascuna amb les seves característiques i dades internes, que utilitzaran les mateixes implementacions de formats de dades definides anteriorment. Aquest component serà l'encarregat d'utilitzar \textit{twitter4j} per configurar la crida a la Stream API i obrir el \textit{thread} encarregat d'anar rebent i emmagatzemant els tuits rebut. Addicionalment, serà també responsabilitat d'aquesta tool comunicar-se amb \textit{KafkaCommunication} per utilitzar el mètode de comunicació (a través de IF o personalitzat) que correspongui, d'acord amb les necessitats de la \textit{tool}.
\end{itemize}

A la figura \ref{fig:Figura12} es pot observar un exemple de l'objecte JSON generat a partir de la transformació definida per la implementació de la interfície \textit{MonitoringData}. Aquest objecte serà enviat, a través de la \textit{tool} i utilitzant la classe \textit{KafkaCommunication}, al Kafka \textit{endpoint} i Kafka \textit{topic} definits a la configuració del monitoratge. De les dades generades per la comunicació amb l'API, recollirem: \textit{idItem} (identificador del tuit), \textit{timeStamp} (moment en que s'ha realitzat el tuit), \textit{message} (contingut del mateix), \textit{author} (l'identificador de l'autor), i \textit{link} (enllaç al tuit a la plataforma Twitter). 

\begin{figure}[!h]
\centering
\includegraphics[width=14cm]{Figures/Figure12}
\decoRule
\caption[Exemple dades de sortida generades pel monitor de Twitter]{Exemple dades de sortida generades pel monitor de Twitter}
\label{fig:Figura12}
\end{figure}

Cal destacar que gràcies al disseny de l'arquitectura proposada anteriorment per la implementació d'un monitor integrat al nostre sistema (o, de fet, sense necessitat que l'objectiu final sigui la seva integració) ha requerit una extensió relativament senzilla. Únicament ha calgut implementar aquells aspectes propis de cada monitor, que en són 4: els paràmetres específics de configuració (1), el mapejat dels paràmetres d'entrada (2), el mapejat de les dades de sortida (3), i el funcionament de les \textit{tools} (en aquest cas, twitter4j) per realitzar els processos de monitoratge (4). D'aquesta manera satisfem un dels principals objectius: dotar al nostre sistema de la màxima extensibilitat i heterogeneïtat possible, amb la possibilitat de personalitzar l'activitat i semàntica de cada monitor partint d'una arquitectura comuna. \\

\noindent{\large{\textbf{Exposició com a servei REST}}}\\

El següent pas d'acord amb les necessitats tècniques per la integració és exposar el monitor i les seves funcionalitats com un servei web RESTful. D'aquesta manera, mitjançant la documentació d'una API per accedir a les diferents operacions d'alta, baixa i modificació d'un procés de monitoratge, podem integrar els monitors al IF del projecte SUPERSEDE i permetre així el seu accés a través de la plataforma d'integració. En aquest sentit l'ús de JSON com a format de comunicació d'entrada i sortida ens facilita la seva exposició com a servei web, que utilitzarà també el format JSON com a \textit{payload} per fer les crides. La documentació de l'API del monitor de Twitter es pot trobar a l'apèndix ~\ref{AppendixA}, on podem trobar en detall les peticions REST implementades, així com el format dels \textit{inputs} i \textit{outputs} de cadascuna de les crides.\\

Per tal de poder concebre com funcionaria la comunicació amb el monitor per iniciar una nova instància de monitoratge, la figura ~\ref{fig:Figura8} mostra un exemple d'objecte JSON utilitzat.

\begin{figure}[!h]
\centering
\includegraphics[width=11cm]{Figures/Figure8}
\decoRule
\caption[Exemple JSON de configuració del monitor de Twitter]{Exemple JSON de configuració del monitor de Twitter}
\label{fig:Figura8}
\end{figure}

En referència a aquesta possible petició, la figura ~\ref{fig:Figura9} mostra un exemple de la resposta (en cas d'èxit en la configuració del monitor) que retorna aquest monitor per la petició anterior. En aquesta figura, definim 2 paràmetres de retorn. Primerament,  \texttt{idConf}, que conté l'identificador de la nova instància de monitoratge creada (variable indispensable per tal que sigui usable i poder realitzar adaptacions posteriors). En segon lloc \texttt{status}, que indica si la petició s'ha realitzat amb èxit o s'ha produït algun error.

\begin{figure}[!h]
\centering
\includegraphics[width=8cm]{Figures/Figure9}
\decoRule
\caption[Exemple resposta amb èxit del monitor de Twitter]{Exemple amb èxit del monitor de Twitter}
\label{fig:Figura9}
\end{figure}

Donat el cas que s'hagi produït algun error, el format de resposta anterior no és vàlid (no s'ha creat cap instància i, per tant, cap identificador pot ser produït) ni suficient (no ens aporta informació de quin ha estat l'error). En aquest cas, un exemple de resposta seria l'exposat a la figura ~\ref{fig:Figura10}.

\begin{figure}[!h]
\centering
\includegraphics[width=10cm]{Figures/Figure10}
\decoRule
\caption[Exemple resposta amb error del monitor de  Twitter]{Exemple resposta amb error del monitor de Twitter}
\label{fig:Figura10}
\end{figure}

\subsubsection{Google Play Monitor}

Aquest segon monitor, encarregat del monitoratge de dades de la botiga d'aplicacions dels sistemes operatius Android, \textbf{Google Play}, pretén recollir les dades referents a les crítiques o \textit{reviews} que els usuaris de Google Play fan de les aplicacions que es descarreguen. De manera semblant al monitor de Twitter, però en un context i amb un objectiu diferent, recull dades dels usuaris i els missatges que publiquen al voltant d'un focus específic.\\

En aquest cas, i aprofitant que tenim definit un cas d'ús simple d'un monitor amb una sola \textit{tool}, podem procedir a explotar l'arquitectura dels nostres monitors i definir un monitor que utilitzi més d'una \textit{tool} per realitzar els processos de monitoratge. En aquest cas, s'ha fet una recerca a la xarxa sobre els diversos sistemes i components que existeixen (d'ús total o parcialment lliure) per realitzar el monitoratge, i s'han triat els dos següents degut a les seves característiques:

\begin{itemize}
\item \textbf{GooglePlayAPI}. \textit{Tool} que implementa una comunicació en 2n pla, similar al sistema de \textit{threads} emprat pel monitor de Twitter, amb la Google Play Developer API. Diem "similar" degut al fet que aquesta API, que funciona mitjançant un sistema d'autenticació OAuth 2.0, no permet obrir un \textit{stream} de dades com es presentava amb el monitor de Twitter, sino que permet obtenir, mitjançant crides API, les dades referents al conjunt total de \textit{reviews} publicades per una aplicació determinada en el moment de la crida. D'aquesta manera no podem obtenir un \textit{stream} de dades real de forma directa, sino que serà responsabilitat de la pròpia \textit{tool} simular aquest \textit{stream} de dades. Per fer-ho, realitzarem de forma periòdica (segons el \textit{timeSlot} definit) crides a l'API per obtenir aquestes dades, i la \textit{tool} s'encarregarà de recollir i retornar únicament les \textit{reviews} realitzades durant el període de monitoratge. Com a avantatge principal, aquesta \textit{tool} permet fer \textbf{fins a 60 crides} en 1 hora, un nombre molt elevat especialment si ho contrastem amb altres eines. Per contra, com a limitació únicament permet obtenir dades d'aplicacions de les quals l'usuari que s'ha autenticat n'és el propietari.
\item \textbf{GooglePlay-AppTweak}. Aquesta eina permet obtenir als usuaris registrats un ample ventall de dades de les aplicacions publicades a GooglePlay. En aquest cas, l'autenticació està basada en crides amb \textit{token}, i el sistema és similar a la \textit{tool} de GooglePlayAPI: necessitem simular dins el comportament de la mateixa \textit{tool} un \textit{stream} de dades parsejant els resultats de la consulta de les \textit{reviews} públiques fins al moment de la crida. En aquest cas, i en contraposició amb l'anterior \textit{tool}, com a avantatge principal aquesta eina permet \textbf{obtenir informació de qualsevol app} publicada al mercat, independentment de la seva propietat. Per contra, la seva versió gratuïta (ofereix un servei de pagament) únicament permet realitzar 100 crides al mes. Una xifra que pot resultar baixa però que, considerant la naturalesa de l'entorn monitorat, podem considerar suficient en alguns casos, ja que equivaldria a 3 crides diàries. 
\end{itemize}

La tria d'aquestes dues \textit{tools} es basa en les seves característiques d'ús i limitacions, complementàries entre elles, que ens permeten configurar un monitor prou adaptable a les necessitats del procés de monitoratge. Addicionalment a les consideracions anteriors, les dades obtingudes per ambdues \textit{tools} no són les mateixes (tot i que coincideixen en un alt percentatge), i per tal caldrà gestionar tal i com es comentarà més endavant aquesta irregularitat.\\

Per aquest cas d'ús, i degut al que hauria de ser l'escenari més habitual (monitorar en temps reals els comentaris que es realitzen sobre una aplicació específica),es defineix com a paràmetre de configuració la pròpia aplicació a monitorar, identificada pel \textbf{nom del \textit{package}}:

\begin{itemize}
\item \textbf{packageName} - identificador únic d'una aplicació publicada a Google Play
\end{itemize}

\begin{figure}
\centering
\includegraphics[width=14cm]{Figures/Figure11}
\decoRule
\caption[Arquitectura software del monitor de Google Play]{Arquitectura software del monitor de Google Play}
\label{fig:Figura11}
\end{figure}

La implementació del monitor basada en l'arquitectura genèrica definida presenta diverses similituds en comparació amb el monitor de Twitter, amb la principal diferència que, per aquest cas, tenim 2 \textit{tools} implementades a utilitzar.

\begin{itemize}
\item \textbf{GooglePlayMonitoringParams.} Subclasse de la classe abstracta \textit{MonitoringParams} que hereta els atributs comuns a tots els monitors, i afegeix com a atribut particular del monitor un string \textit{packageName}.
\item \textbf{GooglePlayParserConfiguration.} Implementació de la interfície que defineix la transformació del format d'entrada del monitor (JSONObject) a la instància de \textit{GooglePlayMonitoringParams}.
\item \textbf{GooglePlayMonitoringData.} Implementació de la interfície que defineix la transformació de les dades recollides al format JSON de sortida. En aquest cas hem d'afegir la consideració prèviament introduïda sobre la variabilitat de les dades entre les dues \textit{tools}. Davant aquest punt, podem explotar els avantatges de l'arquitectura presentada, que ens dona dues opcions:
\begin{enumerate}
\item Podem implementar una única classe que estendrà la interfície \textit{MonitoringData} i contindrà tots els paràmetres (comuns i no comuns) de les dades generades durant el procés de monitoratge. En aquest cas, la pròpia \textit{tool} crearà instàncies d'objectes monitorats amb les dades de les quals disposi, i aquelles que no pugui definir (i per tant, que prenguin valors buits o nuls) simplement no apareixeran en la transformació a objecte JSON que enviarem al Kafka \textit{endpoint}. D'aquesta manera, amb una única classe satisfem les necessitats del monitor.
\item Alternativament podem modelar un conjunt de classes que implementin la interfície, de manera que cada \textit{tool} utilitzi una d'aquestes implementacions. D'aquesta manera, disposaríem d'una implementació de \textit{MonitoringData} per cada \textit{tool} d'acord amb la informació que proporciona cadascuna d'aquestes.
\end{enumerate}
\begin{figure}[!h]
\centering
\includegraphics[width=14cm]{Figures/Figure13}
\decoRule
\caption[Exemple dades de sortida generades pel monitor de Google Play]{Exemple dades de sortida generades pel monitor de Google Play}
\label{fig:Figura13}
\end{figure}
Gràcies al disseny, cada desenvolupador podrà utilitzar l'opció que m'es s'adeqüi a les seves necessitats. La 1a opció pot resultar adequada quan p.e. el subconjunt de dades que volem obtenir sigui independent de la \textit{tool}, i per contra la 2a opció ens farà servei per desacoblar totalment les dades entre diferents tools. Pel nostre cas d'ús, considerarem la 1a opció, ja que dins la integració de SUPERSEDE, l'anàlisi d'aquestes dades serà independent de la \textit{tool} utilitzada.
\item \textbf{GooglePlayAPI}. Implementació de \textit{ToolInterface} que utilitza l'API de Google Play Developer. Aquesta \textit{tool} implementa una comunicació mitjançant autenticació OAuth 2.0 mitjançant els \textit{tokens} obtinguts al registrar un usuari de Google Play com a \textit{developer}. Per aquest projecte s'ha utilitzat un compte personal de Google Play compartit amb altres estudiants del Grau en Enginyeria Informàtica, per tal de poder probar i validar aquesta \textit{tool} (ja que, tal i com especificat anteriorment, únicament ens permet obtenir dades de les aplicacions de les quals l'usuari n'és l'autor). Degut al gran volum de dades que es poden generar al demanar les \textit{reviews} d'una aplicació, l'API funciona mitjançant un sistema de paginació. És a dir: la crida REST per obtenir les reviews retorna un subconjunt de mida relativament petita i, addicionalment, un \textit{token} que permet referenciar el següent subconjunt (o pàgina) de reviews. Aquest token s'utilitza per realitzar una nova crida, que conté un nou subconjunt de reviews i, addicionalment, un nou \textit{token} per la següent pàgina. D'aquesta manera, les dades venen paginades i subdividides. És responsabilitat de la \textit{tool} implementar la lògica per realitzar les crides de forma iterativa. 
Aquest tractament pot ser un procés relativament lent (en termes computacionals). Però això no suposa un trencament amb la filosofia de "fotografiar" el sistema en un moment determinat: en el moment que es fa la crida REST, l'API captura les \textit{públiques} en aquell moment, i construeix els objectes de dades paginats, de tal manera que encara que es triguin uns segons en computar totes les pàgines de dades, aquestes sempre seran una representació del moment en que s'ha fet la primera crida, corresponent al final d'un cicle de monitoratge de durada definida al \textit{timeSlot.}
\item \textbf{AppTweak}. Implementació de \textit{ToolInterface} que utilitza l'API de AppTweak. A diferència de l'anterior, tant l'autenticació com la lògica per obtenir les dades és molt més senzilla. En el cas de la primera, funciona mitjançant l'ús d'un \textit{token} a la capçalera de la crida REST, únic per usuari a la plataforma. En el cas de la segona, una sola crida REST retorna totes les dades corresponents a les \textit{reviews} d'aquella aplicació.
\end{itemize}

\noindent{\large{\textbf{Exposició com a servei REST}}}\\

De forma anàloga a l'exposició com a servei del monitor de Twitter, cal dissenyar i implementar un servei REST per desplegar les funcionalitats del monitor de Google Play. La figura ~\ref{fig:Figura14} mostra un exemple d'un possible objecte JSON de configuració del monitor.\\

\begin{figure}[!h]
\centering
\includegraphics[width=11cm]{Figures/Figure14}
\decoRule
\caption[Exemple JSON de configuració del monitor de GooglePlay]{Exemple JSON de configuració del monitor de GooglePlay}
\label{fig:Figura14}
\end{figure}

Tot i que el paràmetre de \textit{toolName} ja l'havíem vist en l'anterior monitor, ja que és necessari per la instanciació de la \textit{tool} utilitzant el patró reflexió, en aquest cas cobra una especial importància, ja que al disposar d'un conjunt de \textit{tools} podem utilitzar els noms d'aquestes per configurar el monitor; en aquest cas, amb les \textit{tools} de \textbf{GooglePlay} i \textbf{AppTweak}.

\subsubsection{App Store Monitor}

\noindent{\large{\textbf{Exposició com a servei REST}}}\\

\section{Monitor Manager}

Partint de l'arquitectura genèrica dels monitors, i l'exemplificació d'aquesta en casos d'ús reals (i per tant a seva implementació), disposem dels primers components essencials per construïr el nostre sistema. En aquest punt disposem d'un subconjunt de monitors que satisfan les seves especificacions tècniques individuals, relacionades amb l'activitat de monitoratge. Però necessitem anar un pas mes enllà en l'evolució del nostre sistema per satisfer els objectius generals d'adaptabilitat, i procedir a desenvolupar components d'integració.\\

Per aquest objectiu, el primer pas és el desenvolupament del \textbf{Monitor Manager}. Aquest component, tal com el seu nom indica (i com s'ha presentat al \textit{Capítol 5. Visió general del sistema}), s'encarrega de la gestió de tots els monitors desplegats al sistema. Per gestió entenem la satisfacció dels següents requisits funcionals:

\begin{enumerate}
\item \textbf{Inicialització de procés de monitoratge per a un monitor específic.} El component ha de ser capaç de rebre una petició d'inicialització de procés de monitoratge per un dels monitors integrats al sistema, i redireccionar aquesta petició al monitor corresponent, de manera que aquest pugui processar i executar la petició.
\item \textbf{Modificació de procés de monitoratge per a un monitor específic.} Donat un procés de monitoratge existent per a un monitor específic integrat al sistema, el component ha de rebre una petició de reconfiguració d'aquest, processar-la i redireccionar-la al monitor corresponent.
\item \textbf{Aturada de procés de monitoratge per a un monitor específic.} Donat un procés de monitoratge existent per a un monitor específic integrat al sistema, el component ha de rebre una petició per aturar-lo, processar-la i redireccionar-la al monitor corresponent.
\end{enumerate}

Com es pot extreure de les funcionalitats anteriors, l'objectiu principal d'aquest component és \textbf{processar} i \textbf{redireccionar} les peticions relacionades amb accions sobre els monitors desplegats. D'aquesta manera, i enfocat al màxim nivell d'independència entre components, estem afegint un nivell d'abstracció per sobre dels monitors que permetrà la comunicació amb la resta de components del sistema sense necessitat de conèixer els detalls semàntics i tècnics de cada implementació dels monitors.\\

Per tant, com a part de la tasca en el disseny i desenvolupament d'aquest monitor, caldrà definir un punt d'entrada únic pels 3 tipus d'operacions: \textbf{iniciar}, \textbf{modificar} i \textbf{aturar} un procés de monitoratge en un monitor específic. La lògica interna del Monitor Manager, per tant, s'haurà d'encarregar de processar aquestes peticions, identificar a quin monitor pertoca, i redireccionar-la al mateix.

\subsection{Especificacions tècniques}

Les especificacions tècniques en les que ens basarem per dissenyar i implementar el Monitor Manager requereixen satisfer una \textbf{integració} que actuï com a pont \textbf{únic} entre components tercers del sistema i els nostres monitors. Seguint els criteris presentats anteriorment, caldrà considerar els següents punts:

\begin{enumerate}
\item \textbf{INPUT - Petició de configuració de monitor}. Descripció dels paràmetres i el format genèric d'aquests per realitzar la redirecció de la petició.
\item \textbf{\textit{ACTION} - Processat de la petició}. Tractament del format genèric dels paràmetres i interpretació d'acord amb el monitor adreçat.
\item \textbf{\textit{OUTPUT} - Redirecció al monitor}. Enviament de la petició processada cap al monitor.
\end{enumerate}

\subsubsection{INPUT - Petició de configuració de monitor}

La petició de configuració s'ha de rebre en un format genèric que el propi Monitor Manager sigui capaç d'encapsular, identificar el monitor al qual cal enviar-la, i redireccionar-la posteriorment. Per fer-ho, independentment del format, necessitem identificar dos factors claus que el Monitor Manager ha de rebre:

\begin{itemize}
\item El \textbf{monitor específic} al qual s'ha de redireccionar la petició
\item Els \textbf{paràmetres de configuració} per aquell monitor: \textit{toolName} + \textit{kafkaEndpoint} + \textit{kafkaTopic} + \textit{timeSlot} + [paràmetres específics]
\end{itemize} 

Respecte al 2n punt, es tracta d'informació formatada que el monitor específic necessita, i que per tant podem reaprofitar i mantenir estructurada tal i com s'ha documentat prèviament. Respecte al 1r punt, en canvi, es tracta d'informació que el Monitor Manager necessita addicionalment per processar la redirecció.\\

En aquest punt cal decidir si aquesta informació l'afegim de forma implícita a l'objecte JSON de configuració, o bé si busquem una alternativa que ens permeti mantenir l'objecte JSON intacte. En el primer cas, la informació relativa a la configuració queda totalment integrada en un sol objecte JSON que utilitzarem per \textbf{redireccionar} i \textbf{configurar} el monitor. Però això obliga a modificar el format d'aquest JSON, afegint informació addicional que el monitor no necessita. En el segon cas, en canvi, podem mantenir un objecte de configuració que es propaga entre els diferents components: primer com a \textit{input} del Monitor Manager, i després com a \textit{output} redireccionat com a \textit{input} al monitor específic. Alternativament, però, cal buscar una forma de comunicar al Monitor manager a quin monitor trobarà la \textit{tool} sobre la qual hem d'iniciar un procés de monitoratge.\\

\begin{figure}[!h]
\centering
\includegraphics[width=11cm]{Figures/Figure15}
\decoRule
\caption[Exemple JSON de configuració de monitor al Monitor Manager]{Exemple JSON de configuració de monitor al Monitor Manager}
\label{fig:Figura15}
\end{figure}

En aquest sentit aprofitarem la necessitat d'exposició dels components com a serveis RESTful per, mitjançant el propi disseny de la API que implementarà el Monitor Manager, definir aquest paràmetre. En el  disseny d'aquesta API, present a l'apèndix ~\ref{AppendixA}, proposem com a paràmetre dins la URL de les 3 crides a implementar (creació, modificació i eliminació) el propi identificador del monitor. Així, exemplificant la crida per crear una configuració sobre el monitor de Twitter, podem veure a la figura ~\ref{fig:Figura15} la URL que defineix el recurs per realitzar aquesta operació (on \textit{monitorName} és el paràmetre de la URL que identifica el monitor a redireccionar), així com un exemple de JSON que rebrà. Com es pot apreciar, aquest presenta els mateixos camps que el JSON definit als monitors.\\

Amb aquestes dades d'entrada, el Monitor Manager ja és capaç de realitzar la redirecció al monitor corresponent.

\subsubsection{ACTION - Processat de la petició}

Un cop definida la informació i el seu format d'entrada necessaris per satisfer les especificacions tècniques, cal avaluar com partim d'aquesta informació a la petició de configuració de monitors.\\

Ja que la única tasca a realitzar és la redirecció (ja que les dades no cal que siguin tractades), hem de partir del paràmetre \textit{monitorName} per identificar el monitor al qual redireccionar. El component IF exposa, de manera separada, la implementació de classes o \textit{proxies} diferents per cada monitor implementat. És a dir: per cada monitor que vulguem integrar i registrar al nostre sistema, necessitarem afegir-ho a IF per garantir-ne la integració de la comunicació, mitjançant la implementació d'un \textit{proxy} amb els mètodes de comunicació que ofereix. Per tant, la responsabilitat del Monitor Manager serà utilitzar el paràmetre \textit{monitorName} per discernir entre els \textit{proxies} implementats per IF i seleccionar aquell que es correspongui al monitor al qual la petició ha d'anar adreçada. 

\subsubsection{OUTPUT - Redirecció al monitor}

Finalment, identificat i instanciat el \textit{proxy} el Monitor Manager executarà una crida \textit{addConfiguration}, \textit{updateConfiguration} o \textit{deleteConfiguration} en funció de l'operació enviada. De nou en aquest sentit s'aprofita l'\textit{input} d'aquest component mitjançant la seva exposició com a servei: cada mètode de creació, modificació i eliminació crida al mètode corresponent del \textit{proxy} definit per \textit{monitorName}. A aquest \textit{proxy} s'enviarà la instància de configuració rebuda d'entrada, reaprofitant exactament el mateix format, mantenint així la uniformitat de les dades desitjada.

\subsection{Implementació del Monitor Manager}

En definitiva, d'acord amb les necessitats establertes, la implementació del Monitor Manager es basarà simplement en el disseny i implementació d'un servei REST que implementi els 3 mètodes definits anteriorment, i que la seva lògica interna s'encarregui simplement d'identificar, a partir del nom del monitor rebut com a paràmetre, el \textit{proxy} que implementa la comunicació amb aquell monitor.\\

Podeu consultar el projecte i la seva implementació al següent enllaç:  \url{https://github.com/supersede-project/monitor_feedback/tree/develop_quim-motger/monitormanager}

\section{Orchestrator}

Aquest component forma part de la integració d'aquest projecte dins el projecte SUPERSEDE, i per tant el disseny i desenvolupament d'aquest ha estat únicament realitzat com a part d'aquest TFG de forma parcial. Concretament, aquelles parts referents a l'activitat de monitoratge i reconfiguració de monitors.\\

La seva tasca principal és actuar de pont entre els components encarregats d'aplicar adaptacions (\textit{Enactments}) i aquells encarregats del monitoratge (\textit{Monitoring}), que recordem forma part del cicle MAPE-k descrit anteriorment. Aquest és necessari en el moment que diversos sistemes d'adaptabilitat apliquen modificacions en diversos sistemes encarregats de monitorar dades. Aquest TFG n'és un cas específic. Si, per contra, aquest hagués d'actuar com a sistema independent, l'abstracció garantida pel Monitor Manager, que unifica tots els monitors sota un únic punt d'entrada, ja seria suficient per facilitar la comunicació, el manteniment i l'extensió de components. Per contra, si haguéssim volgut afegir la lògica del Monitor Manager al mateix Orchestrator, i aprofitar la seva necessitat dins SUPERSEDE per estalviar-nos una capa en el sistema, ens veuríem obligats a aplicar modificacions i manteniment a l'Orchestrator, un component genèric d'integració, davant modificacions específiques pel nostre cas d'ús. Per tal d'evitar això, s'ha decidit mantenir aquests dos components per separat, i garantir un desacoblament de components el més alt possible.\\

Addicionalment, a aquest component se li assigna una responsabilitat superior associada a la gestió i control del sistema de monitoratge i les seves especificacions. Amb això es fa referència a la persistència i control de metadades relacionades amb els monitors, que ens permeten tenir de forma integrada coneixement sobre el nostre sistema. Aspectes com, p.e., els \textbf{tipus de monitors} integrats, que engloben un conjunt de monitors treballant sobre una mateixa àrea, o bé les \textbf{tools implementades} per cada tipus de monitor.\\

\subsection{Especificacions tècniques}

En general, per tant, podem concebre l'Orchestrator com un \textbf{component d'integració} i un \textbf{repositori de metadades}. Per tant, independentment de les funcionalitats que caldrà implementar (que a continuació detallarem), caldrà tenir en compte dues característiques bàsiques:

\begin{itemize}
\item Haurà de gestionar la \textbf{persistència} de dades relacionades amb els monitors, i per tant caldrà afrontar el \textbf{disseny} i la \textbf{implementació} d'una base de dades.
\item Paral·lelament, caldrà implementar un controlador exposat com a \textbf{servei REST} que exposi les seves funcionalitats i, addicionalment, permeti la seva integració a IF.
\end{itemize}

Per definir els detalls d'aquestes dues branques, necessitem primer definir amb quines metadades estarem treballant i com estructurarem el seu contingut. Definirem, essencialment, tres entitats principals:

\begin{enumerate}
\item \textbf{Monitor configuration}. Instància ja definida anteriorment, que representa l'execució d'una \textit{tool} amb uns paràmetres de configuració específics.
\item \textbf{Monitor tool}. Engloba les metadades associades a aquella \textit{tool} i té associades totes les configuracions en procés d'execució. D'aquesta manera, les dades referents a les configuracions actives per a una \textit{tool} no es troben exclusivament al desplegament del propi monitor, el que suposaria la necessitat d'accedir al propi monitor. Contràriament, aquestes dades es trobaran integrades a l'Orchestrator, el que facilitarà la seva gestió quan es tracti de casos exclusivament de lectura de dades. Aquest darrer propòsit, en qualsevol cas, no entra part del context d'aquest projecte, sinó que està orientat a la integració del TFG amb SUPERSEDE i l'explotació d'aquestes dades.
\item \textbf{Monitor type}. Engloba un conjunt de \textit{tools} que pertanyen a monitors d'una mateixa categoria, com per exemple \textit{Social Networks} o \textit{Market Places}. Aquest agrupament per tipus, com en el cas anterior, tampoc és rellevant pel nostre context, però ens ofereix integració de dades que resulta d'interès en termes analítics.
\end{enumerate}

% Please add the following required packages to your document preamble:
% \usepackage{multirow}
\begin{table}[htb]
\centering
\label{InstanciacioOrchestrator}
\begin{tabular}{|l|l|l|}
\hline
\textbf{Monitor Type}            & \textbf{Monitor Tool}          & \textbf{Monitor configuration} \\ \hline
\multirow{3}{*}{Social Networks} & \multirow{3}{*}{TwitterAPI}    & TwitterAPI-Conf1               \\ \cline{3-3} 
                                 &                                & TwitterAPI-Conf2               \\ \cline{3-3} 
                                 &                                & TwitterAPI-Conf3               \\ \hline
\multirow{6}{*}{Market Places}   & \multirow{2}{*}{GooglePlayAPI} & GooglePlay-Conf1               \\ \cline{3-3} 
                                 &                                & GooglePlay-Conf2               \\ \cline{2-3} 
                                 & GooglePlay-AppTweak            & GooglePlay-AppTweak-Conf1      \\ \cline{2-3} 
                                 & \multirow{2}{*}{iTunesApple}   & iTunesApple-Conf1              \\ \cline{3-3} 
                                 &                                & iTunesApple-Conf2              \\ \cline{2-3} 
                                 & AppStore-AppTweak              & AppStore-AppTweak-Conf1        \\ \hline
\end{tabular}
\caption{Exemple d'instanciació del sistema de monitoratge}
\end{table}

Per tal d'entendre bé aquest esquema, podem visualitzar un esquema conceptual a la taula 7.1 que presenta un exemple de constitució d'un sistema de monitoratge en actiu com el que hem anat descrivint al llarg d'aquest projecte. Bàsicament tenim dos tipus de monitors, \textbf{Social Networks} i \textbf{Market Places}. Pel primer tenim definida una única \textit{tool}, \textbf{TwitterAPI}; pel segon, tenim definides 4, dues per cada \textit{market place} (\textit{AppStore} i \textit{GooglePlay}). Fixem-nos en el detall que en cap moment estem contemplant l'entitat \textbf{monitor} en aquest esquema. Això es deu al fet que, dins el nostre esquema, l'entitat monitor només és significativa des d'un punt de vista físic, com a punt integrat de desplegament que agrupa un conjunt de \textit{tools}. Podríem considerar que existeixen punts comuns entre \textit{tools} dins un mateix monitor, com p.e. la implementació dels paràmetres d'entrada o de sortida. Però la nostra arquitectura no restringeix aquest aspecte. Dins un mateix monitor, cada \textit{tool} pot implementar aquestes interfícies i classes d'acord amb les seves necessitats. D'aquesta manera, el concepte monitor únicament és significatiu des del punt de vista d'implementació i desplegament físic, no lògic o de metadades. Per tant, no considerarem aquesta entitat dins el nostre esquema conceptual.\\

Finalment, a la taula podem veure un conjunt d'exemple de configuracions. Aquest no vindrà limitat per cap restricció, i tot i que cada monitor s'ha d'encarregar en última instància de gestionar-ho, serà a través del Orchestrator i el Monitor Manager que es gestionaran les altes, baixes i modificacions d'aquests.

\subsubsection{INPUT - Peticions de configuració del sistema de monitoratge}

Mitjançant l'exposició de les funcionalitats com a servei REST, l'Orchestrator haurà de rebre el conjunt de peticions amb dos objectius principals: la \textbf{gestió de metadades} del nostre sistema de monitoratge, i la \textbf{redirecció de les peticions} associades directament a configuracions del sistema. La definició d'aquests mètodes d'entrada no ve a ser més que el disseny d'un controlador basat en les operacions \textbf{CRUD} (\textit{\textbf{C}reate}, \textit{\textbf{R}ead}, \textit{\textbf{U}pdate}, \textit{\textbf{D}elete}) per les entitats \textit{Monitor Type}, \textit{Monitor Tool} i \textit{Monitor Configuration}. A la taula 7.2 podem visualitzar un resum conceptual dels mètodes que necessitarem definir, i quins paràmetres caldrà definir per cadascun d'ells.\\

\begin{table}[htb]
\centering
\label{PeticionsOrchestrator}
\begin{tabular}{|p{4cm}|p{6cm}|p{3cm}|}
\hline
\textbf{Operació}                            & \textbf{Definició}                                                        & \textbf{Paràmetres}                                                                                                           \\ \hline
\multicolumn{3}{|c|}{\textbf{Monitor Type}}                                                                                                                                                                                                                        \\ \hline
Llistat de Monitor Types                     & Retorna el llistat de tots els Monitor Types instanciats al sistema       & /                                                                                                                                       \\ \hline
Obté un Monitor Type                         & Retorna les dades d'un Monitor Type específic                             & - nomTipus                                                                                                                              \\ \hline
Alta de Monitor Type                         & Crea una nova instància de Monitor Type                                   & - nomTipus                                                                                                                              \\ \hline
Baixa de Monitor Type                        & Elimina una instància de Monitor Type existent                            & - nomTipus                                                                                                                              \\ \hline
\multicolumn{3}{|c|}{\textbf{Monitor Tool}}                                                                                                                                                                                                                        \\ \hline
Llistat de Tools per Monitor Type            & Retorna el llistat de totes les Tools implementades per a un Monitor Type & - nomTipus                                                                                                                              \\ \hline
Obté una Tool per un Monitor Type            & Retorna les dades d'una Tool per un Monitor Type                          & \begin{tabular}[c]{@{}l@{}}- nomTipus\\ - nomTool\end{tabular}                                                                          \\ \hline
Alta de Tool per un Monitor Type             & Crea una nova Tool per un Monitor Type                                    & \begin{tabular}[c]{@{}l@{}}- nomTipus\\ - nomTool\\ - nomMonitor\end{tabular}                                                           \\ \hline
Baixa de Tool per un Monitor Type            & Elimina una Tool existent per un Monitor Type                             & \begin{tabular}[c]{@{}l@{}}- nomTipus\\ - nomTool\end{tabular}                                                                          \\ \hline
\multicolumn{3}{|c|}{\textbf{Monitor Configuration}}                                                                                                                                                                                                               \\ \hline
Llistat de Configurations per una Tool       & Retorna el llistat de totes les Configurations actives per una Tool       & \begin{tabular}[c]{@{}l@{}}- nomTipus\\ - nomTool\end{tabular}                                                                          \\ \hline
Obté una Configuration per una Tool          & Retorna les dades d'una Configuration per una Monitor Tool                & \begin{tabular}[c]{@{}l@{}}- nomTipus\\ - nomTool\\ - idConf\end{tabular}                                                               \\ \hline
Alta de Configuration per una Tool           & Crea una nova Configuration per una Monitor Tool                          & \begin{tabular}[c]{@{}l@{}}- nomTipus\\ - nomTool\\ - timeSlot\\ - kafkaEndpoint\\ - kafkaTopic\\ - {[}custom{]}\end{tabular} \\ \hline
Modificació d'una Configuration per una Tool & Modifica els paràmetres d'una Configuration existent per una Monitor Tool & \begin{tabular}[c]{@{}l@{}}- nomTipus\\ - nomTool\\ - idConf\\ - timeSlot\\ - kafkaEndpoint\\ - kafkaTopic\\ - {[}custom{]}\end{tabular} \\ \hline                                                                                                                    
Baixa d'una Configuration per una Tool       & Elimina una Configuration existent per una Monitor Tool                   & \begin{tabular}[c]{@{}l@{}}- nomTipus\\ - nomTool\\ - idConf\\ - timeSlot\\ - kafkaEndpoint\\ - kafkaTopic\\ - {[}custom{]}\end{tabular} \\ \hline
\end{tabular}
\caption{Llistat de peticions d'entrada del Orchestrator}
\end{table}

Respecte aquests mètodes, cal considerar els identificadors de cadascuna d'aquestes entitats:

\begin{itemize}
\item \textbf{Monitor Type}. S'identifica per un \textbf{nom} únic
\item \textbf{Monitor Tool}. S'identifica per un \textit{Monitor Type} (un nom de \textit{Monitor Type}) + un \textbf{nom} únic de tool
\item \textbf{Monitor Configuration}. S'identifica per un \textit{Monitor Type} (un nom de \textit{Monitor Type}) + una \textit{Monitor tool} (un nom de \textit{Monitor Tool}) + un \textbf{id} únic de la configuració
\end{itemize}

En general, tota la informació d'entrada que necessita per executar cadascuna de les operacions es basa en identificar el recurs sobre el qual aplicar l'acció. Aquesta identificació a nivell de configuració, que és la darrera entitat que a nosaltres ens interessa considerar, es basa en un triplet de la forma:\\

\centerline{\textbf{monitorType + monitorTool + idConf}}\bigskip

Només en un cas necessitem afegir més informació, que serà pels dos casos a treballar en aquest projecte: la \textbf{creació d'una configuració} i la seva \textbf{reconfiguració} (o actualització). En aquest cas, i tal i com s'ha definit en l'especificació i disseny dels monitors, caldrà afegir d'una banda aquells paràmetres obligatoris per a tot monitor, i addicionalment aquells propis de cada \textit{tool}.\\

\subsubsection{ACTION - Actualització de metadades}

El comportament intern de l'Orchestrator serà relativament senzill, ja que únicament necessita contemplar l'actualització de metadades i el processament de les peticions realitzades per redirigir-les, si s'escau, al monitor corresponent.\\

Considerant primer \textbf{l'actualització de metadades}, l'Orchestrator actualitzarà d'acord a l'operació executada modificacions a la base de dades d'acord amb \textbf{creació}, \textbf{actualització} i eliminació de les entitats prèviament definides. En cas que es tracti d'operacions de lectura, només caldrà accedir a la base de dades i llegir les dades associades a la petició realitzada. Aquest aspecte de l'acció interna el considerarem trivial, ja que es tracta únicament de modificacions associades a una BDD relacional (veure més endavant a \textit{Implementació del Orchestrator}), i no resulta d'especial interès per la finalitat del nostre projecte.\\

\subsubsection{OUTPUT - Redirecció de peticions}

Per altra banda cal definir la \textbf{redirecció de peticions} a monitors. Fixem-nos que, al disposar de totes les dades integrades, aquesta únicament caldrà contemplar-la pels casos en els que no només cal realitzar modificacions sobre les pròpies metadades (o bé simplement realitzar lectures), sinó també realitzar canvis relacionats amb l'execució real dels monitors. Això per tant inclourà: \textbf{alta} de configuració, \textbf{reconfiguració} d'una configuració, i \textbf{eliminació} d'una configuració. En aquests 3 casos, l'Orchestrator haurà de realitzar la redirecció amb el Monitor Manager, utilitzant la integració amb IF.\\

El mateix problema de criteri de redirecció sorgeix que en el cas del Monitor Manager. Necessitem saber quina \textit{tool} va associada a quin monitor, per tal d'indicar al Monitor Manager a quin monitor es troba implementada aquella \textit{tool}. El Monitor Manager requeria aquest paràmetre en la petició, ja que ell no té constància d'aquestes dades. Però a l'Orchestrator sí que disposem d'aquesta metadada, associada a una \textit{tool}, i definida en el moment de creació, tal i com observem també a la taula 7.2. Per tant, a l'hora d'instanciar el \textit{proxy} de IF per comunicar-se amb el Monitor Manager, cridarem al mètode corresponent al Monitor identificat per aquest atribut. D'aquesta manera la redirecció queda satisfeta.\\

\subsection{Implementació de l'Orchestrator}

Sense entrar en molt detall en els aspectes d'implementació de les especificacions prèviament descrites, podem trobar els detalls d'aquesta a: 

\begin{itemize}
\item Disseny de l'esquema de la \textbf{base de dades}, definit a l'apèndix ~\ref{AppendixA}
\item Disseny de l\textbf{API RESTful}, definit a l'apèndix ~\ref{AppendixA}
\end{itemize}

Degut a que el comportament i els detalls específics no són d'especial interés pel domini d'aquest projecte, no s'entrarà amb més detall en la informació referent a la implementació tècnica, que s'ha realitzat seguint les instruccions i els requisits tècnics específics del component Orchestrator genèric implementat per \textit{partners} del projecte SUPERSEDE.
% Chapter Template

\chapter{Modelatge UML de configuracions dels monitors} % Main chapter title

\label{ModelatgeConfiguracions} % Change X to a consecutive number; for referencing this chapter elsewhere, use \ref{ChapterX}


Finalitzada l'especificació de disseny i tècnica dels monitors i les seves configuracions, així com els detalls de la seva implementació, ja tenim definit un sistema de monitoratge que satisfà els criteris d'\textbf{heterogeneïtat} i \textbf{distribució}. Aquestes característiques queden garantides dins el sistema de monitoratge com a unitat independent.\\

El següent pas és gestionar una extensió del sistema que permeti gestionar l'adaptabilitat dels monitors, i concretament que aquesta pugui ser automatitzada, sense necessitat de definir explícitament crides a peticions de reconfiguració al Orchestrator. Tot i que queda fora de l'abast el disseny d'un sistema d'anàlisi i detecció automàtica de reconfiguracions a aplicar (l'equivalent al sistema d'\textbf{A}nàlisi dins el \textit{MAPE-k}), el nostre objectiu és dotar al sistema de monitoratge d'un sistema d'adaptabilitat, que sigui capaç de processar i computar aquestes reconfiguracions.\\

\section{Requisits del modelatge}

Per poder definir els models amb els quals hem de treballar, primer necessitem definir les necessitats del nostre sistema. En termes genèrics, el sistema d'adaptabilitat a dissenyar necessita resoldre la següent problemàtica:\\

\textit{Donada una \textbf{instància} del sistema de monitoratge, el sistema ha de rebre una \textbf{proposta de reconfiguració} dels monitors, basada en la modificació d'una \textbf{característica} específica, aplicant un \textbf{conjunt d'accions} sobre els \textbf{elements del sistema} determinats.}\\

Per major aclariment, anem a analitzar el significat de cadascun dels conceptes introduïts en aquesta definició:

\begin{enumerate}
\item \textbf{Instància del sistema}. Referent al conjunt de \textbf{classes} que defineixen els diferents tipus de configuracions persistents al sistema (és a dir, les diferents implementacions de processos de monitoratge associats a les diferents \textit{tools}), i les \textbf{instàncies} actives corresponents (o processos de monitoratge actius).
\item \textbf{Proposta de reconfiguració}. Necessitem modelar, per una banda, el conjunt de característiques que podem referenciar i editar dins el nostre sistema. Aquestes inclouen, per exemple, la definició de l'atribut \textit{timeSlot} de les instàncies dels monitors de Twitter dins el tipus de monitor SocialNetwork. Per altra banda, necessitem modelar, basat en aquest model de característiques, propostes de configuracions específiques, on es defineixin p.e. valors específics d'aquest \textit{timeSlot}.
\item \textbf{Característica específica}. En relació al punt anterior, que engloba un conjunt de característiques modificables en el sistema, necessitem referenciar quina característica volem modificar.
\item \textbf{Conjunt d'accions}. A partir de les propostes de configuració, i l'aplicació d'una característica específica, hem de definir les diferents accions que s'han de realitzar sobre la instància actual del sistema. Aquestes accions inclouran, seguint amb el mateix exemple, la modificació p.e. d'un atribut com \textit{timeSlot}.
\item \textbf{Elements del sistema}. Necessitem referenciar sobre quins elements del sistema volem aplicar els canvis. És a dir: hem de ser capaços de referenciar, dins el model del sistema, quines instàncies cal modificar i quines no, per tal d'aplicar les accions basades en les característiques anteriors només als que ens interessin.
\end{enumerate}

\section{Disseny de models UML}

Partint d'aquests requisits conceptuals, necessitem definir el conjunt de models amb els quals treballarem per representar tots aquests punts, i permetre així la computació automàtica de canvis dins el nostre sistema. Procedirem presentant cadascun dels models, explicant breument en què consisteixen, i més profundament com s'apliquen al nostre projecte.\\

La implementació i gestió d'aquests models es realitzarà utilitzant l'eina Papyrus, introduïda anteriorment al \textit{Capítol 5. Eines de desenvolupament}. Concretament, pel tractament i gestió de la seva implementació es farà servir la llibreria UML2, versió 5.0.0.

\subsection{Base Model}

Definirem \textit{Base Model} com un diagrama de classes UML que defineix les diferents classes que representen cadascun dels diferents tipus de configuracions de monitors i les seves instàncies. Conceptualment, per tant, és senzill de concebre dins el nostre context, ja que no ve a ser res més que una representació en UML del sistema. D'acord amb el nostre disseny, les necessitats d'aquest disseny UML són les següents: 

\begin{itemize}
\item Cal definir una \textbf{classe abstracta} que representi l'abstracció genèrica de totes les configuracions: és a dir, que englobi aquelles propietats (atributs) compartits entre totes les \textit{tools} del nostre sistema. En el nostre cas, aquests seran: \textit{timeSlot}, \textit{kafkaEndpoint}, \textit{kafkaTopic}, \textit{toolName} i el propi identificador \textit{id}.
\item A continuació cal afegir les implementacions d'aquesta classe abstracta basades en les possibles diferents configuracions que es poden donar d'alta en el nostre sistema. En definitiva: totes aquelles varietats de configuracions segons els diferents paràmetres a definir (que en el nostre sistema ve a ser equivalent a les diferències entre tipus de monitors).
\item Per cadascuna d'aquestes implementacions, cal modelar el conjunt de configuracions (és a dir, instàncies de classes), amb els valors dels atributs definits.
\end{itemize}

\begin{figure}
\centering
\includegraphics[width=13cm]{Figures/Figure16}
\decoRule
\caption{Exemple de Base Model del sistema de monitoratge}
\label{fig:Figura16}
\end{figure}

A la figura ~\ref{fig:Figura16} podem visualitzar un exemple de \textit{Base Model} que representa aquesta fotografia d'un sistema on tenim definits el monitor de Twitter i el de Google Play. \textit{AMonitorConfiguration} representa la classe abstracta dels diferents tipus de configuracions; \textit{SocialNetworks} i \textit{MarketPlaces} representen les extensions de configuracions genèriques amb els paràmetres addicionals definits; i finalment, \textit{OlympicGamesMarketPlacesMonitor} i \textit{OlympicGamesTwitterMonitor} són instàncies dels diferents tipus de configuracions (no s'ha afegit el monitor d'AppStore per simplicitat en l'exemplificació). També veiem addicionalment una classe anomenada \textit{MonitoringSystemConfiguration}, en relació d'agregació a \textit{AMonitorConfiguration}, i una instància d'aquesta mateixa. Aquestes són classes establertes com a requisits dins el context SUPERSEDE, i per tant no són necessàries a tenir en compte.\\

L'objectiu principal de l'adaptabilitat del sistema serà aplicar canvis en aquest \textit{Base Model} que defineix l'estat actual del sistema. Així, el sistema d'adaptabilitat actualitzarà per una banda el \textit{Base Model} actual, aplicant els canvis pertinents, i computarà les diferències amb l'anterior per definir reconfiguracions que s'enviaran al Orchestrator.

\subsection{Features}

Aquests canvis, que anomenem de forma genèrica, són modificacions dins el diagrama de classes del sistema, i per tant inclou aspectes com la modificació del valor d'un atribut d'una instància de configuració. Però aquestes modificacions no poden ser aleatòries: suposant que no considerem el comportament del sistema de \textit{\textbf{P}lanificació} dins el \textit{MAPE-k}, i per tant no definim l'autogeneració de noves propostes de configuracions del sistema, necessitem modelar de forma definida una sèrie de casos, o configuracions predefinides, que defineixin un \textbf{conjunt de combinacions de característiques} adreçades a aplicar-se en diferents casos. És a dir: necessitem modelar d'alguna forma diferents alternatives per valors dels diferents atributs de les configuracions.\\
 
En termes específics, el que estarem fent serà definir un conjunt de propostes que aplicarem per a casos específics. Podríem, per exemple, modelar una proposta de configuració que disminueixi el valor del \textit{timeSlot} quan el volum de dades obtingudes pel monitoratge sigui molt baix, i vulguem rebre aquestes amb menys periodicitat; o bé podríem re-orientar les dades a un \textit{kafkaEndpoint} diferent quan, per diferents motius, el \textit{kafkaEndpoint} actual estigui caigut i les dades rebotin. Davant aquests exemples, i un nombre indeterminat (d'acord amb les nostres necessitats), podem definir diferents casos o propostes que representin semànticament un canvi en el sistema.\\

Per modelar aquest aspecte, introduïm dos tipus de models addicionals: el \textbf{\textit{Feature Model}} i les \textbf{\textit{Feature Configurations}}.

\subsubsection{Feature Model}

Un \textit{Feature Model} (FM), o model de característiques, en la seva definició genèrica, és un diagrama que permet gestionar el conjunt d'aspectes comuns i variables dins d'un sistema i els seus components. Representa, en definitiva, una modelització estandarditzada que defineix una jerarquia entre aquestes característiques i estableix les diferents opcionalitats de configuració dins un sistema.\\

\begin{figure}
\centering
\includegraphics[width=14cm]{Figures/Figure17}
\decoRule
\caption{Exemple de Feature Model del sistema de monitoratge}
\label{fig:Figura17}
\end{figure}

Aplicat al nostre context, un \textit{Feature Model} ens serveix per definir quines característiques del nostre sistema podem referenciar i modificar, i com s'estructuren aquestes des d'un punt de vista jeràrquic d'acord amb les implementacions de la classe abstracta de configuració de monitors. Un dels casos d'ús que contemplarem, ja que permet validar el nostre sistema alhora que mostrar fàcilment la repercussió dels canvis, és la reconfiguració de l'atribut \textit{timeSlot} d'una instància de monitor. A la figura ~\ref{fig:Figura17} podem observar un exemple de \textit{Feature Model} del nostre sistema. Com podem veure, representa una modelació conceptual del sistema de monitoratge i la seva jerarquia: un sistema de monitoratge, amb el conjunt de tipus de monitors, que implementen un conjunt de monitors específics, on cadascun d'ells engloben una sèrie de \textit{features}, és a dir, característiques que podem referenciar del nostre sistema. En aquest exemple, si ens fixem per exemple en el cas de Twitter, tenim per una banda la \textit{feature} \textit{timeSlot}, representada com un Integer de valor variable, i \textit{keywordExpression}, que en comptes de tenir un valor (com podria ser un string) editable, defineix dues característiques opcionals, que representen dos possibilitats de valor associades a aquesta \textit{feature}. De manera similar trobem el cas del monitor de Google Play i d'App Store, on la \textit{feature} \textit{toolName} és només configurable amb dos opcions en cada cas, segons les \textit{tools} que tenen implementades, i de nou el \textit{timeSlot}. Cal destacar que aquest es tracta únicament d'un exemple de \textit{Feature Model}: les diferents \textit{features} es podrien ampliar i/o modificar, p.e. afegint noves \textit{tools} conforme s'implementin, o bé afegint altres paràmetres a modificar (p.e. \textit{kafkaEndpoint} o \textit{kafkaTopic}).

\subsubsection{Feature Configuration}

Una \textit{Feature Configuration} (FC) representa una configuració específica d'un \textit{Feature Model}. És a dir: davant les diferents opcionalitats i variabilitats que un \textit{Feature Model} defineix, com poden ser la personalització del valor d'una \textit{feature}, o bé la selecció entre un conjunt d'opcions, la \textit{Feature Configuration} és la modelació d'un cas específic d'aquestes \textit{features}. A diferència del \textit{Feature Model}, que presenta totes les opcions del sistema, la \textit{Feature Configuration} presenta només una opció específica per aquells punts variables dins el \textit{Feature Model}. Aquestes opcions específiques reben el nom de \textbf{selections} (seleccions).\\

Aplicat al nostre context, i seguint amb l'exemple anterior, l'objectiu d'una \textit{Feature Configuration} es modelar una proposta de configuració del sistema de monitoratge. Així, definim uns valors específics pels atributs de les configuracions. A la figura ~\ref{fig:Figura18} podem visualitzar un exemple basat en el \textit{Feature Model} de la figura ~\ref{fig:Figura17}.\\

\begin{figure}
\centering
\includegraphics[width=13cm]{Figures/Figure18}
\decoRule
\caption{Exemple de Feature Configuration del sistema de monitoratge}
\label{fig:Figura18}
\end{figure}

Si ens fixem en aquesta \textit{Feature Configuration} i la comparem amb el \textit{Feature Model}, veiem que tots els punts de variabilitat que aquesta segona definia s'especifiquen amb un valor o opcions específics. Per aquest cas, p.e., es proposa un valor de timeSlot específic per les configuracions de tots els tipus de monitors; paral·lelament, en el cas del monitor de Twitter s'especifica la \textit{KeywordExpression1} com a opció a triar per aquest paràmetre, i les tools de \textit{GooglePlayAPI} i \textit{iTunes} per les \textit{tools} dels monitors de Market Places. En definitiva, és la representació d'una proposta de configuració de les \textit{features} definides. Orientat al nostre cas d'ús, aquesta proposta ens permet traduir-les al \textit{Base Model} en una sèrie de modificacions d'atributs.\\

\subsection{Pattern Model}

Aquesta modificació d'atributs, però, únicament ens està definint una acció genèrica. És a dir, la semàntica que podem desprendre d'una \textit{Feature Configuration} és simplement la definició d'una proposta general per les configuracions dels monitors. El problema, però, està en que la \textit{Feature Configuration} defineix tot el conjunt del sistema (en aquest cas, tots els atributs de tots els monitors), però no ens dona cap informació sobre quines instàncies del \textit{Base Model} cal aplicar aquests canvis. És a dir: suposem que al \textit{Base Model} de la figura ~\ref{fig:Figura16} volem actualitzar el \textit{timeSlot} de la instància \textit{OlympicGamesTwitterMonitor}, però no de la instància \textit{OlympicGamesMarketPlacesMonitor}.\\

Amb aquest objectiu, necessitem afegir informació addicional, una forma de modelar la identificació dels elements sobre els quals volem aplicar els canvis de reconfiguració (aplicant criteris variats, segons cada cas). Per satisfer aquesta necessitat, utilitzarem els \textit{Pattern Models}. Aquests models defineixen, mitjançant un llenguatge específic de la llibreria UML2 (\textit{Viatra Query Language}, VQL), patrons de cerca que permeten retornar elements (classes, instàncies, relacions, atributs, etc.) dins un diagrama UML.\\ 

\begin{figure}
\centering
\includegraphics[width=11cm]{Figures/Figure19}
\decoRule
\caption{Exemple de Pattern Model del sistema de monitoratge}
\label{fig:Figura19}
\end{figure}

A la figura ~\ref{fig:Figura19} podem veure un exemple de \textit{Pattern} aplicat al nostre cas d'estudi. Sense entrar en els detalls específics de la sintaxi, aquest patró anomenat \textit{twitterTimeSlot} retorna tots els objectes de tipus \textit{Slot}, que en l'especificació UML2 venen a ser els atributs d'una instància d'una classe, que compleixen dues característiques: que reben per nom \textit{timeSlot} (1) i que són atributs d'una instància de la classe amb nom \textit{twitter} (2). Així, quan vulguem aplicar de forma dinàmica adaptacions sobre el sistema, i concretament sobre els \textit{timeSlots} (1) de les instàncies de monitoratge del monitor de Twitter (2), podem referenciar aquests objectes amb la màxima eficiència, utilitzant les eines que la pròpia llibreria ens dona per treballar dinàmicament amb models UML.\\

De manera similar a les FC, necessitarem definir un \textit{pattern} diferent per cada cas d'adaptació, d'acord amb els criteris específics. Tot i que això no exclou que alguns siguin reutilitzables entre diferents reconfiguracions. El potencial del VQL ens permet generar, amb una sintaxi relativament senzilla, patrons de cerca de tot tipus i que filtrin la cerca segons tot tipus de criteris: valors dels atributs de les instàncies, tipus de configuracions, relacions amb superclasses, etc. 

MENCIONA EL MQ

\subsection{Profile Model}

A partir dels \textit{patterns} definits, som capaços d'obtenir i referenciar aquests  

\subsection{Aspect Model}


% Chapter Template

\chapter{Sistema d'adaptabilitat} % Main chapter title

\label{DissenySistema} % Change X to a consecutive number; for referencing this chapter elsewhere, use \ref{ChapterX}

Els models UML prèviament descrits ens permeten dissenyar i modelar de forma estandarditzada el sistema de monitoratge. Proporciona, semànticament, tots els recursos necessaris per traduir propostes de configuracions de sistemes en accions reals sobre els monitors implementats.\\

El següent pas és el disseny i implementació d'un sistema que, a partir dels models anteriors, i de tot el domini que defineixen, sigui capaç de gestionar la persistència dels models, obtenir-ne els necessaris per aplicar reconfiguracions, actualitzar els models d'acord a aquestes modificacions, i traduir la informació implícita als models en accions de reconfiguració reals a enviar al sistema de monitoratge.\\

Procedirem a presentar els diferents components que componen aquest sistema.

\section{Model Repository}

El primer component que cal introduir per començar a entendre el \textit{workflow} del sistema d'adaptabilitat és el \textbf{Model Repository}. Aquest component actua en termes genèrics com a un repositori que gestiona la persistència del conjunt de models UML que intervenen en el modelatge del sistema i la seva reconfiguració. Això, per tant, implica tot el conjunt de models descrits anteriorment: \textit{Base Model}, \textit{Feature Model}, \textit{Feature Configuration}, \textit{Pattern Model}, \textit{Profile Model} i \textit{Adaptability Model}.\\

Principalment, les funcionalitats que ha de satisfer són les següents:

\begin{enumerate}
\item Gestionar la persistència dels models en disc
\item Mapejar i encapsular l'estructura de directoris definida per estructurar els models segons el seu tipus
\item Encapsular els mètodes CRUD bàsics per la gestió dels models, aïllant la lògica interna de la resta de components
\item Encapsular mètodes extensius que permetin obtenir models sota unes certes característiques
\end{enumerate}

D'aquesta manera, podem concebre aquest component com una abstracció entre la naturalesa interna dels models i la lògica associada a la càrrega/descàrrega de fitxers amb el nostre sistema, assignant la responsabilitat d'aquests primers punts al Model Repository.\\

Una primera aproximació que ens podríem plantejar seria implementar un component unitari que definís un controlador (que després s'exposaria com a servei per integrar a IF) amb tots els mètodes necessaris per la gestió dels models. Si al nostre sistema únicament resultés d'interès les operacions bàsiques de models, aquesta seria una bona alternativa, ja que estaríem simplificant l'arquitectura del sistema, i hi hauria poc marge per la millora. Però si anem un pas més enllà, i plantegem les necessitats de reconfiguració i adaptabilitat del nostre sistema, veiem que un controlador basat en operacions CRUD únicament ens permetria definir reconfiguracions on tots els models que intervenen en l'adaptació són triats estàticament, no dinàmicament. És a dir: el potencial que oferiria el Model Repository seria referenciar els models implicats en una adaptació mitjançant els seus identificadors, que haurien de ser introduïts manualment.\\

Això per tant implica que no podríem aplicar escenaris d'adaptació com per exemple:\\

\centerline{\textit{Actualitzar el sistema de monitoratge amb la darrera Feature Configuration computada}}\bigskip

Si tornem al cicle \textit{MAPE-k}, suposant un sistema de \textit{\textbf{P}lanificació} que produeix noves propostes de configuració (FC), aquest enviaria periòdicament aquests models al nostre sistema d'adaptabilitat. Si el nostre sistema és capaç de consultar les dates en què aquestes propostes es van afegint, és capaç de computar quina és la darrera afegida. I en definitiva, és capaç de computar, de forma automàtica i sense necessitat de donar-li cap mena d'informació, quins canvis aplicar al sistema de monitoratge.\\

Aquest escenari és clau pel nostre projecte. L'automatització del nostre sistema ve donada precisament gràcies a la persistència dels models UML i a l'actualització d'aquests, responsabilitat d'un altre sistema, a partir dels quals aquest és capaç de llegir, processar i traduir en modificacions reals sobre les activitats de monitoratge. Aquest escenari, però, és només un exemple de possible escenari d'adaptabilitat, basat en un criteri com la data, que serà el que nosaltres farem servir pel desenvolupament del projecte. Però el potencial està en veure que els criteris poden ser diversos, sempre i quan es treballin amb metadades que podem extreure a partir dels models, com és el cas de la data de generació de les propostes de reconfiguració.\\

Addicionalment a aquest problema, alguns models tenen dependències amb altres models UML. El cas més evident, l'\textit{Adaptability Model}, té dependències amb \textit{Feature Configurations}, que alhora tenen dependències amb \textit{Feature Models}, i també amb \textit{Pattern Models} i \textit{Profile Models}. La resolució i gestió d'aquestes dependències pot arribar a ser extremadament complicada si, al carregar un d'aquests models, necessitem computar i resoldre quines són aquestes dependències cada vegada que volem carregar el Model Repository a memòria per executar una adaptació des del component encarregat d'aquest aspecte, l'Adapter, que satisfent el criteri de distribució del nostre sistema pot no tenir accés al mateix repositori en disc que el Model Repository. \\

Partint d'aquestes necessitats, és evident que necessitem estendre els mètodes de lectura de models a mètodes més complerts, on utilitzem dades auxiliars per fer cerques dins el nostre repositori. És en aquest punt quan aquesta primera aproximació resulta ineficient: si volem accedir a les metadades dels fitxers dels models, tals com la data de creació, l'autor, el sistema que l'ha computat, etc. (més endavant les veurem amb més detall), carregar tots els fitxers dinàmicament per accedir a aquestes dades per fer la cerca resulta molt ineficient. És per això que, alternativament, utilitzarem la següent arquitectura per gestionar la persistència de models:

\begin{itemize}
\item \textbf{Model Repository Manager.} Component que gestiona les metadades dels models, emmagatzemades en una base de dades relacionals, i que estén un controlador amb els mètodes de cerca per obtenir les metadades dels models desitjats.
\item \textbf{Model Repository Client.} Component que es comunica amb el Model Repository Manager per obtenir les dades dels models, gestiona la persistència del repositori de models, i encapsula els mètodes i objectes per obtenir i treballar amb els models associats a les reconfiguracions.
\end{itemize}

D'aquesta manera, el primer assumeix les responsabilitats de cerca i gestió de models en base a les seves metadades, i el segon assumeix la responsabilitat principal d'encapsular programàticament l'accés als models, per tal que la resta de components del sistema d'adaptabilitat puguin aïllar-se de la lògica interna d'aquest punt.

\subsection{Model Repository Manager}

Satisfent les necessitats del Model Repository Manager, necessitem contemplar els següents punts:

\begin{enumerate}
\item Disseny i implementació d'una base de dades relacional que emmagatzemi les metadades per cada tipus de model.
\item Disseny i implementació d'un component que accedeixi a la base de dades, i que exposi a través d'un controlador els mètodes de consulta i modificació dels models.
\end{enumerate}

\subsubsection{Disseny de la base de dades}

En primer lloc necessitem definir quines seran les metadades que considerarem per cada model. Generalment, podem considerar que tots els models definits tindran les següents dades:

\begin{itemize}
\item \textbf{Id.} Identificador d'aquell model, únic pel tipus de model (\textit{Base Model}, \textit{Feature Model}...) que representa.
\item \textbf{Name.} Nom del fitxer del model (sense l'extensió).
\item \textbf{AuthorId.} Identificador de l'autor del model.
\item \textbf{CreationDate.} Data de creació del model.
\item \textbf{LastModificationDate.} Data de la darrera modificació del fitxer del model.
\item \textbf{FileExtension.} Extensió del fitxer (.uml, .vql, .yamfc ...)
\item \textbf{SystemId.} Utilitzat dins el context SUPERSEDE per identificar els models que corresponen als diferents escenaris. En el nostre cas, aquest sempre serà \textit{MonitoringReconfiguration}.
\item \textbf{RelativePath.} Ruta relativa del directori \textit{root} on s'emmagatzema el model.
\item \textbf{Dependencies.} Llistat d'identificadors dels models dels quals depèn aquest model.
\end{itemize}

Tot i així, addicionalment existeix la possibilitat d'estendre atributs específics pels diferents models. Considerarem útils pel nostre context els següents:

\begin{itemize}
\item \textbf{Base Model}
\begin{itemize}
\item \textbf{Status.} Indica si el model ha estat computat per un sistema extern (\textit{Computed}), si és el resultat d'una adaptació dins el sistema d'adaptabilitat (\textit{Enacted}), o bé si ha estat dissenyat manualment (\textit{Designed}).
\end{itemize}
\item \textbf{Feature Configuration}
\begin{itemize}
\item \textbf{Status.} Indica si la \textit{Feature Configuration} ha estat computada per un sistema extern (\textit{Computed}), si s'ha aplicat com a adaptació dins el sistema (\textit{Enacted}), o bé si ha estat dissenyada manualment (\textit{Designed}).
\end{itemize}
\item \textbf{Adaptability Model}
\begin{itemize}
\item \textbf{Feature Id.} Identificador de la \textit{feature} referenciada per l'Adaptability Model.
\end{itemize}
\end{itemize}

A partir d'aquestes dades podem procedir al disseny de la base de dades del repositori de metadades. Al existir atributs comuns i atributs diferenciats, hem de decidir quin tipus d'herència apliquem a la base de dades. Recordant les tres opcions, aplicades al nostre cas obtindríem el següent:

\begin{itemize}
\item \textbf{\textit{Single table inheritance.}} Definir una única taula a la base de dades \textit{Model} que inclogui tots els atributs possibles, inclosos els específics, i prengui valors nulls per aquells que no tenen aquell atribut.
\item \textbf{\textit{Class table inheritance.}} Definir una taula genèrica \textit{Model} i N taules addicionals per cada tipus que referenciïn la primera, amb els atributs addicionals per cada cas.
\item \textbf{\textit{Concrete table inheritance.}} Definir una taula per cada tipus de model i replicar els atributs comuns.
\end{itemize}

En el nostre cas, optarem per l'opció \textit{concrete table inheritance}. La raó principal d'aquesta opció és que, tot i compartir la major part de les dades, les entitats de models amb tipus diferents mai tindrà sentit contemplar-les conjuntament. És a dir: qualsevol lectura o modificació de models es farà sobre un model (o conjunt de models) d'un tipus específic, mai sobre models de forma genèrica (no tindrà sentit concebir l'entitat \textit{Model} abstracta). Si considerem la segona opció, veiem que seria molt ineficient, ja que caldria fer \textit{joins} internes per obtenir les dades els models, i per la mateixa raó que la ja esmentada sabem que seria un cost innecessari. Per tant, optarem per definir una taula per cada tipus, mantenint així la independència de les metadades entre models. A l'apèndix ~\ref{AppendixA} podem veure el disseny proposat de la base de dades, d'acord als 6 tipus de models definits.\\

D'aquesta manera, la identificació d'un model queda definida per \textbf{id + ModelType}, el primer com a atribut explícit de metadades i el segon com a metadada implícita derivada de la taula en la qual emmagatzemem el model.

\subsubsection{Disseny i implementació del domini i controlador}

Modelat l'esquema de les metadades a la base de dades, podem procedir a implementar les classes de domini amb les quals treballarem per operar sobre aquestes. Partint del disseny anterior necessitem definir dos aspectes. En primer lloc, necessitem \textbf{modelar la jerarquia} de models, segons els atributs genèrics i els atributs específics de cada model. En segon lloc, definir com modelarem i identificarem el llistat de dependències de cada model.\\

\begin{figure}
\centering
\includegraphics[width=11cm]{Figures/Figure21}
\decoRule
\caption{Disseny del domini del Model Repository Manager}
\label{fig:Figura21}
\end{figure}

A la figura ~\ref{fig:Figura21} es presenta el diagrama simplificat de la proposta de disseny. Per la definició dels models, definim una classe genèrica \textit{GenericModel} amb tots els atributs genèrics a tots els models. Addicionalment, s'implementa una classe per cada tipus de model específic (al diagrama apareixen \textit{BaseModel} i \textit{ProfileModel}, a mode d'exemple). Als atributs que apareixen al diagrama cal afegir els mètodes \textit{getters} i \textit{setters} tradicionals, no afegits per simplificació de l'esquema, així com els constructors. Per sobre de \textit{GenericModel} definim una classe abstracta \textit{IModel}, requisit establert per SUPERSEDE per futures extensions. \\

Paral·lelament es presenta una proposta de gestió de les dependències. Ja que els nostres models es troben identificats per la parella \textit{id + ModelType}, es proposa la definició d'una interfície \textit{IModelId} genèrica, totalment genèrica i sense cap mètode, oberta a possibles extensions i refactoritzacions necessàries pel context de SUPERSEDE. Per gestionar aquestes dependències pel nostre context, definim una nova interfície, \textit{ITypedModelId}, que defineix els mètodes per extreure les dues dades necessàries per identificar un model, \textit{getNumber} i \textit{getModelType}. D'aquesta interfície definim una implementació, que obté aquests dos valors definits als mètodes anteriors mapejant directament aquests atributs. D'aquesta manera, el llistat de dependències dels models usats en la reconfiguració de monitors seran instàncies de la classe \textit{TypedModelId} amb els dos atributs \textit{id} i \textit{ModelType}.\\ 

Amb aquest domini podem procedir a implementar el controlador i, posteriorment, la seva exposició com a servei REST per la integració amb IF. Sense entrar en gaires detalls tècnics sobre aquesta part (molt similar a les anteriors exposicions com serveis), definirem les 5 operacions CRUD bàsiques, sempre considerant per \textit{ModelType}. També serà responsabilitat del Model Repository Manager, associada a cadascuna d'aquestes operacions, la gestió del contingut dels models. Aquesta responsabilitat consistirà bàsicament en emmagatzemar en disc, segons on es trobi desplegat aquest component, els fitxers dels models, amb el contingut determinat. Aquesta persistència es gestionarà definint el contingut del model com un string, tant per indicar al Model Repository Manager el contingut a emmagatzemar, com per obtenir-lo quan es consultin les dades d'un mateix.

\begin{enumerate}
\item \textbf{Llista models d'un tipus.} Retorna les metadades de tots els models existents d'un tipus determinat.
\item \textbf{Obté un model.} Retorna les metadades i un string amb el contingut d'un model donat un identificador i un \textit{ModelType}.
\item \textbf{Crea un nou model.} Emmagatzema un nou model amb les metadades passades al Model Repository Manager i guarda un fitxer a disc amb el contingut, el nom i l'extensió determinats a les metadades.
\item \textbf{Actualitza les metadades d'un model.} Donat un identificador i un \textit{Model Type}, actualitza els valors de les metadades d'un model i el contingut, nom i/o extensió del fitxer (quan s'escaigui). 
\item \textbf{Eliminar les metadades d'un model.} Donat un identificador i un \textit{Model Type}, elimina la instància de metadades d'aquest model i elimina el fitxer del repositori.
\end{enumerate}

A l'apèndix ~\ref{AppendixA} es troba definida l'API utilitzada per la integració d'aquest component.

\subsection{Model Repository Client}

Aquest subcomponent del Model Repository s'encarrega d'actuar de pont entre el Model Repository Manager, que gestiona les dades/metadades dels models, i l'Adapter, component que utilitzarà el Model Repository per obtenir les instàncies dels models que necessita per gestionar les adaptacions de reconfiguracions. El disseny i la implementació d'aquest component han estat principalment realitzats per partners tercers del projecte, amb algunes col·laboracions i aportacions específiques per l'adaptabilitat dels models, especialment per la seva orientació a validar el cas d'ús de reconfiguració de monitors. Per tant, procedirem a explicar únicament aquells aspectes realitzats com a part del treball d'aquest TFG i els conceptes necessaris per entendre el seu funcionament.\\

Des d'un punt de vista de \textbf{disseny}, aquest component defineix una interfície \textit{IModelRepository} que defineix, per una banda, els mètodes CRUD per cadascun dels 6 tipus de models existents al sistema:
 
\begin{itemize}
\item 
\end{itemize}
 
Addicionalment, defineix mètodes de cerca més complexos en base a les metadades dels models. Aquests mètodes seran els que ens permetran executar les adaptacions de forma automatitzada, sense necessitat de definir models específics, a mode de l'exemple introduït anteriorment sobre la cerca de la darrera configuració computada. D'aquests mètodes, ens interessaran especialment els següents:

\begin{itemize}
\item \textbf{Obté el \textit{Base Model} més actual} - Permet obtenir el darrer \textit{Base Model} del sistema, que representa per tant l'estat actual del sistema de monitoratge, i ens servirà per agafar com a model per aplicar els canvis de reconfiguració
\item \textbf{Obté la darrera \textit{Feature Configuration} computada} - Obté la darrera \textit{Feature Configuration} amb l'atribut \textit{status} amb valor \textit{Computed}. D'aquesta manera, podem obtenir aquella darrera proposta de configuració que no s'ha executat encara (i per tant, que no té l'atribut \textit{status} amb valor \textit{Enacted})
\item \textbf{Obté la darrera \textit{Feature Configuration} executada} - Obté la darrera \textit{Feature Configuration} amb l'atribut \textit{status} amb valor \textit{Enacted}. D'aquesta manera, podem obtenir la darrera configuració executada, i juntament amb l'anterior mètode, podem computar les diferències entre les dues.
\item \textbf{Obté els \textit{Adaptability Models} candidats} - Obté el llistat de \textit{Adaptability Models} donat un \textit{systemId}. En el nostre cas, ens interessar obtenir aquells amb sistema \textit{MonitoringReconfiguration}, que representaran totes les adaptacions possibles a aplicar dins el nostre context.
\end{itemize}

La implementació d'aquesta interfície defineix la interacció amb el Model Repository Manager i la gestió de persistència dels models per tal de poder referenciar i treballar amb ells des del component Adapter. A l'apèndix ~\ref{AppendixA} es pot consultar la interfície definida amb tots els mètodes que formen part del Model Repository Client, i que seran els utilitzats per executar la reconfiguració d'un monitor. 

\section{Model Adapter}


\section{Adapter}

Satisfets els requisits tècnics i funcionals del Model Repository disposem d'un component que ens permet carregar dinàmicament tots els models i utilitzar els mètodes que la llibreria UML2 defineix amb els mateixos. Paral·lelament, el Model Adapter ens satisfà la necessitat d'adaptació dinàmica de diagrames de classe UML. El següent pas és definir el component que assumirà la responsabilitat de \textbf{gestionar i computar les reconfiguracions}. Aquesta responsabilitat serà assumida per l'Adapter, la responsabilitat principal del qual serà la interacció amb el Model Repository i el Model Adapter per aplicar l'algorisme que a continuació definim per computar reconfiguracions.\\

\subsection{Algorisme d'adaptabilitat}

L'\textbf{algorisme d'adaptabilitat} de models, o bé reconfiguració de monitors pel nostre cas d'estudi, consisteix en la implementació d'un algorisme que resolgui la següent problemàtica:

\begin{center}
\textit{Davant una petició de reconfiguració del sistema, calculem totes les diferències entre la \textbf{darrera Feature Configuration aplicada} i la \textbf{darrera Feature Configuration computada} i trobem totes les diferències de seleccions. Per cada diferència (és a dir, per cada \textbf{selection} present a una de les dues \textbf{Feature Configurations} però no a l'altra), apliquem les \textbf{accions definides als Adaptability Models} corresponents al \textbf{Base Model}, segons si aquesta s'activa a la nova configuració (apareix a la nova FC) o bé es desactiva (no apareix).}
\end{center}

Per entendre millor aquest plantejament, anem a analitzar pas per pas quins són els passos que s'han de realitzar per passar d'identificar una \textit{feature} a modificar el \textit{Base Model}:

\begin{enumerate}
\item Obtenim la \textbf{darrera \textit{Feature Configuration}} amb \textit{status = Computed}, que representa la última configuració del sistema proposada
\item Obtenim la \textbf{darrera \textit{Feature Configuration}} amb \textit{status = Enacted}, que representa la última configuració del sistema aplicada
\item Obtenim el \textbf{darrer \textit{Base Model} del sistema}, que representa l'estat actual del sistema de monitoratge
\item Computem totes les diferències a nivell de \textbf{selections}, que apareixen només a una de les dues \textit{Feature Configurations}.
\item Per cada \textit{selection} diferent trobada:
\begin{enumerate}
\item Obtenim els \textbf{\textit{Adaptability Models} associats a la \textit{feature}} que representa aquella \textit{selection}
\item Per cada \textit{Adaptability Model} obtingut:
\begin{enumerate}
\item Obté tots els \textbf{\textit{pointcuts} definits per l'\textit{Adaptability Model}}
\item Per cada \textit{pointcut} obtingut:
\begin{enumerate}
\item Obté el \textbf{\textit{pattern} definit per aquell \textit{pointcut}}
\item Utilitza el component \textit{Model Query} per trobar tots els \textbf{elements al \textit{Base Model}} segons el \textbf{\textit{pattern} obtingut}
\item Etiqueta amb el \textit{role} definit pel \textit{pointcut} tots els elements trobats pel \textit{Model Query}
\end{enumerate}
\item Obté les \textbf{\textit{compositions} definides per l'\textit{Adaptability Model}}
\item Per cada \textit{composition}:
\begin{enumerate}
\item Comprova si aquella \textit{composition} defineix l'\textbf{activació o desactivació} de la \textit{feature}. 
\item En cas que coincideixi amb el valor de l'activació/desactivació de la \textit{feature}, es comunica amb el Model Adapter per aplicar l'adaptació corresponent i obté el \textit{Base Model} adaptat.
\end{enumerate}
\end{enumerate}
\end{enumerate}
\item Actualitza el Model Repository amb el \textbf{nou \textit{Base Model}} i la nova \textit{Feature Configuration}.
\end{enumerate}

\subsection{Disseny i diagrama de seqüència}

\begin{figure}[!h]
\centering
\includegraphics[width=13cm]{Figures/Figure22}
\decoRule
\caption{Diagrama de seqüència de la reconfiguració del sistema amb l'Adapter}
\label{fig:Figura22}
\end{figure}

\section{Enactor}

% Chapter Template

\chapter{Dashboard} % Main chapter title

\label{DissenyDashboard} % Change X to a consecutive number; for referencing this chapter elsewhere, use \ref{ChapterX}

Com a complement addicional al desenvolupament del projecte, es defineix la implementació d'un \textit{dashboard} en mitjà d'aplicació web que ens permeti interactuar amb el nostre sistema per tal de, mitjançant una usabilitat molt bàsica i senzilla, poder executar reconfiguracions i visualitzar les diferents adaptacions que s'han realitzat. Buscarem, per tant, un disseny senzill que no requereixi una gran dedicació del projecte (ja que entenem que no és una prioritat), però si que ens permeti la interacció necessària per poder satisfer aquesta necessitat.

\section{Anàlisi de requisits}

Per satisfer aquest objectiu, volem que el nostre \textit{dashboard} satisfaci les següents necessitats:

\begin{itemize}
\item \textbf{Visualitar propostes d'adaptacions.} A través dels models definits al Model Repository, permetre mostrar les dades principals de cada \textbf{Feature Configuration} emmagatzemada al nostre sistema.
\item \textbf{Executar una adaptació.} A partir del llistat de \textit{Feature Configurations} anterior, el \textit{dashboard} ha de permetre a l'usuari seleccionar una proposta d'adaptació i demanar la seva execució al sistema corresponent. Aquesta serà la interacció principal, que ens permetrà executar els casos de validació de la reconfiguració del nostre sistema de monitoratge. 
\item \textbf{Visualitzar adaptacions executades.} Com en la 1a vista, permetre mostrar les dades principals de cada \textbf{Feature Configuration} amb \textit{status == Enacted}, i per tant executada.
\end{itemize}

Des d'un punt de vista funcional, els requisits són relativament senzills, ja que traduït a necessitats de disseny i implementació únicament caldrà definir dues vistes i una interacció amb l'usuari, que es reprodueixi en una interacció amb l'Adapter. Aquesta interacció, com en la resta de casos, es realitza a través de l'eina IF.

\section{Disseny de la interfície}

Partint dels requisits anteriors, necessitem definir dues vistes a la interfície: la vista de \textbf{adaptacions suggerides} i \textbf{adaptacions executades}.\\

\begin{figure}
\centering
\includegraphics[width=14cm]{Figures/Figure32}
\decoRule
\caption{Disseny de la vista d'adaptacions suggerides del \textit{dashboard}}
\label{fig:Figura32}
\end{figure}  

La vista d'adaptacions suggerides consisteix en una taula on cada fila representa una d'aquestes entitats (i, per tant, una \textit{Feature Configuration} del sistema). Per cada instància, mostrarem les següents dades:

\begin{itemize}
\item \textbf{Adaptation id}. Identificador de la \textit{Feature Configuration} que representa
\item \textbf{Name}. Nom únic associat a aquella \textit{Feature Configuration}, usualment associat semànticament a les seves característiques 
\item \textbf{Computation timestamp}. \textit{Timestamp} de la creació de la \textit{Feature Configuration}. Ens serveix per ubicar temporalment les diferents configuracions creades
\item \textbf{Actions}. Cada \textit{Feature Configuration} defineix una sèrie de \textit{features}, o accions, que s'activen o desactiven en funció de les seves característiques. Per cada acció a aplicar, mostrarem:
\begin{itemize}
\item \textbf{Action id}. Identificador de l'acció a realitzar
\item \textbf{Action name}. Nom únic per aquella \textit{Feature Configuration} que rep l'acció o \textit{feature} a modificar
\item \textbf{Action description}. Descripció textual que defineix la modificació i canvis implicats per aquella acció
\item \textbf{Action enabled}. Indica si l'acció indica l'activació d'una \textit{feature} (true) o la seva desactivació (false).
\end{itemize}
\end{itemize}

Addicionalment, en aquesta vista cal contemplar el 2n requisit funcional d'\textbf{execució d'una adaptació}. Mitjançant la selecció d'una de les adaptacions suggerides al nostre sistema (és a dir, la selecció d'una fila a la taula), afegim un botó a la part inferior esquerra amb l'etiqueta \textit{"Enact"}. Aquesta interacció activa la comunicació amb l'Adapter, i inicia l'algorisme d'adaptació de models (en aquest cas, no de forma completament automàtica, sinó donada una \textit{Feature Configuration} específica). Com a resultat, una nova adaptació s'executa al nostre sistema; s'actualitzaran les dades al Model Repository, que podem consultar a través del \textit{dashboard}, i es modificarà l'acció real dels components adaptats (monitors pel nostre cas), d'acord amb les modificacions anteriors.\\

En aquesta segona vista d'adaptacions executades, mostrarem de forma paral·lela al disseny anterior una taula on cada fila representa una \textit{Feature Configuration}, però aquest cop que ja ha estat executada al nostre sistema. Mostrarem algunes de les dades bàsiques d'aquella \textit{Feature Configuration}, com en el cas anterior, però afegirem les següents dades:

\begin{itemize}
\item \textbf{Enactment request time}. \textit{Timestamp} que conté la data i hora en què s'ha realitzat la petició d'execució de l'adaptació
\item \textbf{Enactment completion time}. Mostra el temps total de l'execució d'aquesta reconfiguració, amb una precisió de mil·lisegons. 
\item \textbf{Result}. Indica si l'adaptació s'ha realitzat satisfactòriament (SUCCESS) o bé si s'ha produït algun error (FAILURE).
\end{itemize}

\begin{figure}
\centering
\includegraphics[width=14cm]{Figures/Figure33}
\decoRule
\caption{Disseny de la vista d'adaptacions executades del \textit{dashboard}}
\label{fig:Figura33}
\end{figure} 

Complementàriament s'ha afegit l'opció d'eliminar una entitat (adaptació suggerida o executada) en cada vista, amb objectius orientats al desenvolupament i al testeig de l'aplicació.\\

Amb aquestes dades i amb la interacció definida, tenim un component bàsic usable que ens permetrà executar les reconfiguracions del sistema de monitoratge.

\section{Implementació}

La implementació del \textit{dashboard} consisteix en l'extensió d'un projecte pare definit com el \textit{front-end} del projecte SUPERSEDE. Aquest projecte consisteix en una aplicació web desenvolupada amb \textit{Spring Framework}, que permet el desenvolupament d'aplicacions web i la gestió i \textit{mapping} de persistència. Per desenvolupar el \textit{dashboard} per la reconfiguració de models, implementarem un projecte que actuarà de submòdul dins aquest \textit{front-end}.\\

Seguint l'arquitectura proposada pel desenvolupament d'aplicacions web amb Spring, a la figura ~\ref{fig:Figura34} podem veure el disseny del domini disseny i implementat al \textit{dashboard}. Primerament, cal definir les classes de domini que representen les diferents entitats presents al \textit{dashboard}. Distingirem entre \textit{Adaptation} com a entitat que representa una \textit{Feature Configuration} suggerida, i \textit{Enactment}, que representa una \textit{Feature Configuration} executada. Addicionalment, i per millorar el disseny, definirem l'entitat \textit{Action}, ja introduïda anteriorment, del qual una \textit{Adaptation} en pot tenir vàries instàncies. Per cadascuna d'aquestes classes de domini (amb els atributs definits anteriorment), cal implementar una extensió de \textit{JpaRepository}, que actuen com a controladors de la capa de domini \cite{jpa}. Es tracta de classes que ofereix el framework Spring, i que defineix les operacions bàsiques per les classes de domini (operacions CRUD bàsicament).\\

\begin{figure}
\centering
\includegraphics[width=14cm]{Figures/Figure34}
\decoRule
\caption{Disseny del domini del \textit{dashboard}}
\label{fig:Figura34}
\end{figure} 

Addicionalment, per implementar les vistes i l'aplicació web necessitem definir un controlador RESTful que s'utilitzarà per invocar els diferents mètodes del \textit{dashboard} des de la pròpia aplicació. A la figura ~\ref{fig:Figura35} podem veure els 3 controladors i les operacions implementades. Generalment aquestes estan orientades a test i validació, i únicament 5 seran utilitzades dins el context de l'aplicació: \textit{getAdaptations(),} \textit{deleteAdaptation()}, \textit{enactAdaptation()}, \textit{getEnactments()} i \textit{deleteEnactment()}. Cadascun d'aquests mètodes utilitza els repositoris implementats amb la lògica addicional necessària per satisfer la seva funcionalitat, com és el cas de l'operació \textit{enactAdaptation}, que es comunica a través de IF amb l'Adapter.

\begin{figure}
\centering
\includegraphics[width=14cm]{Figures/Figure35}
\decoRule
\caption{Disseny dels controladors REST del \textit{dashboard}}
\label{fig:Figura35}
\end{figure} 
% Chapter Template

\chapter{Reconfiguració dels monitors} % Main chapter title

\section{Exemple de cas d'ús}
% Chapter Template

\chapter{Conclusions} % Main chapter title

\label{Conclusions} % Change X to a consecutive number; for referencing this chapter elsewhere, use \ref{ChapterX}

\section{Justificació de l'assoliment de competències}

\section{Desviacions i dificultats}

\section{Avaluació del desenvolupament del projecte}

%----------------------------------------------------------------------------------------
%	THESIS CONTENT - APPENDICES
%----------------------------------------------------------------------------------------

\appendix % Cue to tell LaTeX that the following "chapters" are Appendices

% Include the appendices of the thesis as separate files from the Appendices folder
% Uncomment the lines as you write the Appendices

% Appendix Template

\chapter{Twitter Monitor API} % Main appendix title

\label{TwitterMonitor} % Change X to a consecutive letter; for referencing this appendix elsewhere, use \ref{AppendixX}

\section{TwitterMonitor}\label{twittermonitor}

\subsection{Monitor {[}/configuration{]}}\label{monitor-configuration}

\subsubsection{Create a new configuration for this monitor
{[}POST{]}}\label{create-a-new-configuration-for-this-monitor-post}

\begin{itemize}
\item
  Request (application/json)

\begin{verbatim}
{
    "SocialNetworksMonitoringConfProf": {
        "toolName": "TwitterAPI",
        "timeSlot": "30",
        "kafkaEndpoint": "http://localhost:9092",
        "kafkaTopic": "tweeterMonitoring",
        "keywordExpression": "(tweet OR follow) AND (me)",
        "accounts": ["QuimMotger"]
    }
}
\end{verbatim}
\item
  Response (application/json)

  \begin{itemize}
  \item
    Body

\begin{verbatim}
{
    "SocialNetworksMonitoringConfProfResult":{
        "idConf":"1",
        "status":"success"
    }
}
\end{verbatim}
  \end{itemize}
\end{itemize}

\subsection{Configuration
{[}/configuration/\{id\}{]}}\label{configuration-configurationid}

\subsubsection{Updates an existing configuration
{[}PUT{]}}\label{updates-an-existing-configuration-put}

\begin{itemize}
\item
  Request (application/json)

\begin{verbatim}
{
    "SocialNetworksMonitoringConfProf": {
        "toolName": "TwitterAPI",
        "timeSlot": "60",
        "kafkaEndpoint": "http://localhost:9092",
        "kafkaTopic": "tweeterMonitoring",
        "keywordExpression": "(tweet OR follow) AND (me)",
        "accounts": ["QuimMotger"]
    }
}
\end{verbatim}
\item
  Response (application/json)

  \begin{itemize}
  \item
    Body

\begin{verbatim}
{
    "SocialNetworksMonitoringConfProfResult":{
        "idConf":"1",
        "status":"success"
    }
}
\end{verbatim}
  \end{itemize}
\end{itemize}

\subsubsection{Deletes a configuration instance
{[}DELETE{]}}
% Appendix Template

\chapter{Google Play Monitor API} % Main appendix title

\label{GooglePlayMonitor} % Change X to a consecutive letter; for referencing this appendix elsewhere, use \ref{AppendixX}

\section{GooglePlayMonitor}\label{googleplaymonitor}

\subsection{Monitor {[}/configuration{]}}\label{monitor-configuration}

\subsubsection{Create a new configuration for this monitor
{[}POST{]}}\label{create-a-new-configuration-for-this-monitor-post}

\begin{itemize}
\item
  Request (application/json)

\begin{verbatim}
{
    "GooglePlayConfProf": {
        "toolName": "AppTweak",
        "timeSlot": "30",
        "kafkaEndpoint": "http://localhost:9092",
        "kafkaTopic": "MarketPlace",
        "packageName": "com.facebook.katana"
    }
}
\end{verbatim}
\item
  Response (application/json)

  \begin{itemize}
  \item
    Body

\begin{verbatim}
{
    "GooglePlayConfProfResult":{
        "idConf":"1",
        "status":"success"
    }
}
\end{verbatim}
  \end{itemize}
\end{itemize}

\subsection{Configuration
{[}/configuration/\{id\}{]}}\label{configuration-configurationid}

\subsubsection{Updates an existing configuration
{[}PUT{]}}\label{updates-an-existing-configuration-put}

\begin{itemize}
\item
  Request (application/json)

\begin{verbatim}
{
    "GooglePlayConfProf": {
        "toolName": "AppTweak",
        "timeSlot": "60",
        "kafkaEndpoint": "http://localhost:9092",
        "kafkaTopic": "MarketPlace",
        "packageName": "com.facebook.katana"
    }
}
\end{verbatim}
\item
  Response (application/json)

  \begin{itemize}
  \item
    Body

\begin{verbatim}
{
    "GooglePlayConfProfResult":{
        "idConf":"1",
        "status":"success"
    }
}
\end{verbatim}
  \end{itemize}
\end{itemize}

\subsubsection{Deletes a configuration instance
{[}DELETE{]}}\label{deletes-a-configuration-instance-delete}
% Appendix Template

\chapter{App Store Monitor API} % Main appendix title

\label{AppStoreMonitor} % Change X to a consecutive letter; for referencing this appendix elsewhere, use \ref{AppendixX}

\section{AppStoreMonitor}\label{appstoremonitor}

\subsection{Monitor {[}/configuration{]}}\label{monitor-configuration}

\subsubsection{Create a new configuration for this monitor
{[}POST{]}}\label{create-a-new-configuration-for-this-monitor-post}

\begin{itemize}
\item
  Request (application/json)

\begin{verbatim}
{
    "AppStoreConfProf": {
        "toolName": "AppTweak",
        "timeSlot": "30",
        "kafkaEndpoint": "http://localhost:9092",
        "kafkaTopic": "MarketPlace",
        "appId": "567630281"
    }
}
\end{verbatim}
\item
  Response (application/json)

  \begin{itemize}
  \item
    Body

\begin{verbatim}
{
    "AppStoreConfProfResult":{
        "idConf":"1",
        "status":"success"
    }
}
\end{verbatim}
  \end{itemize}
\end{itemize}

\subsection{Configuration
{[}/configuration/\{id\}{]}}\label{configuration-configurationid}

\subsubsection{Updates an existing configuration
{[}PUT{]}}\label{updates-an-existing-configuration-put}

\begin{itemize}
\item
  Request (application/json)

\begin{verbatim}
{
    "AppStoreConfProf": {
        "toolName": "AppTweak",
        "timeSlot": "60",
        "kafkaEndpoint": "http://localhost:9092",
        "kafkaTopic": "MarketPlace",
        "appId": "567630281"
    }
}
\end{verbatim}
\item
  Response (application/json)

  \begin{itemize}
  \item
    Body

\begin{verbatim}
{
    "AppStoreConfProfResult":{
        "idConf":"1",
        "status":"success"
    }
}
\end{verbatim}
  \end{itemize}
\end{itemize}

\subsubsection{Deletes a configuration instance
{[}DELETE{]}}\label{deletes-a-configuration-instance-delete}
% Appendix Template

\chapter{Monitor Manager API} % Main appendix title

\label{MonitorManager} % Change X to a consecutive letter; for referencing this appendix elsewhere, use \ref{AppendixX}

\section{MonitorManager}\label{monitormanager}

\subsection{Monitor
{[}/\{monitorName\}/configuration{]}}\label{monitor-monitornameconfiguration}

\subsubsection{Create a new monitor configuration
{[}POST{]}}\label{create-a-new-monitor-configuration-post}

\begin{itemize}
\item
  Request (application/json)

\begin{verbatim}
{
    "SocialNetworks": {
        "toolName": "TwitterAPI",
        "timeSlot": "30",
        "kafkaEndpoint": "http://localhost:9092",
        "kafkaTopic": "tweeterMonitoring",
        "keywordExpression": "(olympics) AND (streaming)"
    }
}
\end{verbatim}
\item
  Response (application/json)

  \begin{itemize}
  \item
    Body

\begin{verbatim}
{
    "idConf": "1"
    "status": "success"
}
\end{verbatim}
  \end{itemize}
\end{itemize}

\subsection{Monitor Configuration
{[}/\{monitorName\}/configuration/\{confId\}{]}}\label{monitor-configuration-monitornameconfigurationconfid}

\subsubsection{Updates an existing configuration
{[}PUT{]}}\label{updates-an-existing-configuration-put}

\begin{itemize}
\item
  Request (application/json)

\begin{verbatim}
{
    "SocialNetworks": {
        "toolName": "TwitterAPI",
        "timeSlot": "60",
        "kafkaEndpoint": "http://localhost:9092",
        "kafkaTopic": "tweeterMonitoring",
        "keywordExpression": "(olympics) AND (streaming)"
    }
}
\end{verbatim}
\item
  Response (application/json)

  \begin{itemize}
  \item
    Body

\begin{verbatim}
{
    "idConf": "1"
    "status": "success"
}
\end{verbatim}
  \end{itemize}
\end{itemize}

\subsubsection{Deletes a configuration instance
{[}DELETE{]}}\label{deletes-a-configuration-instance-delete}

% Appendix Template

\chapter{Base de dades de l'Orchestrator} % Main appendix title

\label{OrchestratorDB} % Change X to a consecutive letter; for referencing this appendix elsewhere, use \ref{AppendixX}

Disseny i implementació de la base de dades del component \textit{Orchestrator}, d'acord amb els criteris establerts pel projecte SUPERSEDE. Per aquesta, s'ha decidit fer un disseny que garanteixi l'emmagatzemament de l'\textbf{historial de la base de dades} (duplicant les taules per mantenir totes les modificaicons realitzades). Per altra banda, les configuracions s'han materialitzat aplicant la tècnica de \textit{Single Table Inheritance}, i per tant no es distingeix entre tipus de configuracions (aquelles que no defineixin un atribut, tenen per valor per defecte \textit{null}).\\

Podeu consultar l'esquema a la següent pàgina.

\begin{figure}[!h]
\centering
\includegraphics[width=12cm]{Figures/orchestratordb}
\decoRule
\caption{Disseny de la base de dades de l'Orchestrator}
\label{fig:orchestratordb}
\end{figure}
% Appendix Template

\chapter{Orchestrator API} % Main appendix title

\label{Orchestrator} % Change X to a consecutive letter; for referencing this appendix elsewhere, use \ref{AppendixX}

\section{Supersede Orchestrator API}\label{supersede-orchestrator-api}

\subsection{Monitor types
{[}/monitoring/MonitorTypes{]}}\label{monitor-types-monitoringmonitortypes}

\subsubsection{List all monitor types
{[}GET{]}}\label{list-all-monitor-types-get}

\begin{itemize}

\item
  Response 200 (application/json)

\begin{verbatim}
[
	{
  		"id": 1,
  		"name": "SocialNetworks",
 		"createdAt": "2017-03-11 04:09:30.97"
	},
	{
		"id": 1,
		"name": "MarketPlaces",
		"createdAt": "2017-03-11 04:09:30.97"
	}
]
\end{verbatim}
\end{itemize}


\subsubsection{Insert a monitor type
{[}POST{]}}\label{insert-a-monitor-type-post}

\begin{itemize}
\item
  Request (application/json)

\begin{verbatim}
{
  "name": "SocialNetworks"
}
\end{verbatim}
\item
  Response 201 (application/json)

\begin{verbatim}
{
  "id": 1,
  "name": "SocialNetworks",
  "createdAt": "2017-03-11 04:09:30.97"
}
\end{verbatim}
\end{itemize}

\subsection{Monitor types
{[}/monitoring/MonitorTypes/\{monitor-type-name\}{]}}\label{monitor-types-monitoringmonitortypesmonitor-type-name}

\subsubsection{Get a single monitor type
{[}GET{]}}\label{get-a-single-monitor-type-get}

\begin{itemize}
\item
  Response 200 (application/json)

\begin{verbatim}
{
  "id": 1,
  "name": "SocialNetworks",
  "createdAt": "2017-03-11 04:09:30.97"
}
\end{verbatim}
\end{itemize}

\subsubsection{Delete a monitor type
{[}DELETE{]}}\label{delete-a-monitor-type-delete}

\subsection{Monitor tools
{[}/monitoring/MonitorTypes/\{monitor-type-name\}/Tools{]}}\label{monitor-tools-monitoringmonitortypesmonitor-type-nametools}

\subsubsection{Insert monitor tool for a monitor type
{[}POST{]}}\label{insert-monitor-tool-for-a-monitor-type-post}

\begin{itemize}
\item
  Request (application/json)

\begin{verbatim}
{
  "name": "TwitterAPI-2",
  "monitorName": "TwitterAPI"
}
\end{verbatim}
\item
  Response 201 (application/json)

\begin{verbatim}
{
  "id": 6,
  "name": "TwitterAPI-2",
  "monitorTypeId": 1,
  "monitorName": "TwitterAPI",
  "createdAt": "2017-03-11 04:09:30.97"
}
\end{verbatim}
\end{itemize}

\subsection{Monitor tools
{[}/monitoring/MonitorTypes/\{monitor-type-name\}/Tools/\{monitor-tool-name\}{]}}\label{monitor-tools-monitoringmonitortypesmonitor-type-nametoolsmonitor-tool-name}

\subsubsection{Get a single monitor tool for a monitor type
{[}GET{]}}\label{get-a-single-monitor-tool-for-a-monitor-type-get}

\begin{itemize}
\item
   Response 200 (application/json)

\begin{verbatim}
{
  "id": 6,
  "name": "TwitterAPI-2",
  "monitorTypeId": 1,
  "monitorName": "TwitterAPI",
  "createdAt": "2017-03-11 04:09:30.97"
}
\end{verbatim}
\end{itemize}

\subsubsection{Delete a monitor tool for a monitor type
{[}DELETE{]}}\label{delete-a-monitor-tool-for-a-monitor-type-delete}

\subsection{Monitor configurations
{[}/monitoring/MonitorTypes/\{monitor-type-name\}/Tools/\{monitor-tool-name\}/ToolConfigurations{]}}\label{monitor-configurations-monitoringmonitortypesmonitor-type-nametoolsmonitor-tool-nametoolconfigurations}

\subsubsection{Create a monitor configuration for a monitor type and a
monitoring tool
{[}POST{]}}\label{create-a-monitor-configuration-for-a-monitor-type-and-a-monitoring-tool-post}

\begin{itemize}
\item
  Request (application/json)

\begin{verbatim}
{
  "kafkaEndpoint":"http://localhost:9092",
  "keywordExpression":"keyword1 AND keyword2",
  "kafkaTopic":"olympicGamesTwitterMonitoring",
  "state":"active",
  "timeSlot":"300",
  "configSender":"WP4",
  "timeStamp":"2017-03-11 04:09:30.97"
}
\end{verbatim}
\item
  Response 201 (application/json)

\begin{verbatim}
{
  "id": 6,
  "monitorToolId": 1,
  "configSender": "WP4",
  "timeStamp": "Sat June 08 02:16:57 2016",
  "timeSlot": "300",
  "kafkaEndpoint": "http://localhost:9092",
  "kafkaTopic": "olympicGamesTwitterMonitoring",
  "state": "active",
  "keywordExpression": "keyword1 AND keyword2",
  "createdAt": "2017-03-11 04:09:30.97"
}
\end{verbatim}
\end{itemize}

\subsection{Monitor configurations
{[}/monitoring/MonitorTypes/\{monitor-type-name\}/Tools/\{monitor-tool-name\}/ToolConfigurations/\{tool-configuration-id\}{]}}\label{monitor-configurations-monitoringmonitortypesmonitor-type-nametoolsmonitor-tool-nametoolconfigurationstool-configuration-id}

\subsubsection{Get a single monitor configuration for a monitor type and
a monitoring tool
{[}GET{]}}\label{get-a-single-monitor-configuration-for-a-monitor-type-and-a-monitoring-tool-get}

\begin{itemize}
\item
  Response 201 (application/json)

\begin{verbatim}
{
  "id": 6,
  "monitorToolId": 1,
  "configSender": "WP4",
  "timeStamp": "Sat June 08 02:16:57 2016",
  "timeSlot": "300",
  "kafkaEndpoint": "http://localhost:9092",
  "kafkaTopic": "olympicGamesTwitterMonitoring",
  "state": "active",
  "keywordExpression": "keyword1 AND keyword2",
  "createdAt": "2017-03-11 04:09:30.97"
}
\end{verbatim}
\end{itemize}

\subsubsection{Update a monitor configuration for a monitor type and a
monitoring tool
{[}PUT{]}}\label{update-a-monitor-configuration-for-a-monitor-type-and-a-monitoring-tool-put}

\begin{itemize}
\item
  Request (application/json)

\begin{verbatim}
{
  "kafkaEndpoint":"http://localhost:9092",
  "keywordExpression":"keyword1 AND keyword2",
  "kafkaTopic":"olympicGamesTwitterMonitoring",
  "state":"active",
  "timeSlot":"100",
  "configSender":"WP4",
  "timeStamp":"2017-03-11 04:09:30.97"
}
\end{verbatim}
\item
  Response 200 (application/json)

\begin{verbatim}
{
  "id": 7,
  "monitorToolId": 1,
  "configSender": "WP4",
  "timeStamp": "Sat June 08 02:16:57 2016",
  "timeSlot": "100",
  "kafkaEndpoint": "http://localhost:9092",
  "kafkaTopic": "olympicGamesTwitterMonitoring",
  "state": "active",
  "keywordExpression": "keyword1 AND keyword2",
  "createdAt": "2017-03-11 04:09:30.97"
}
\end{verbatim}
\end{itemize}

\subsubsection{Delete a monitor configuration for a monitor type and a
monitoring tool
{[}DELETE{]}}\label{delete-a-monitor-configuration-for-a-monitor-type-and-a-monitoring-tool-delete}

% Appendix Template

\chapter{Model Repository Manager API} % Main appendix title

\label{ModelRepositoryManager} % Change X to a consecutive letter; for referencing this appendix elsewhere, use \ref{AppendixX}

\section{Model Repository Manager}\label{model-repository-manager}

\subsection{Model type
{[}/models/\{modelType\}\{?systemId, status, name, url, authorId, creationDate, lastModificationDate, fileExtension, relativePath\}{]}}\label{model-type-modelsmodeltypesystemidstatusnameurlauthoridcreationdatelastmodificationdatefileextensionrelativepath}

\subsubsection{Get model instances for a given modelType
{[}GET{]}}\label{get-model-instances-for-a-given-modeltype-get}

\begin{itemize}
\item
  Parameters

  \begin{itemize}
  \item
    systemId (optional) - A valid systemId
  \item
    status (optional) - A valid status
  \item
    name (optional) - The model name
  \item
    url (optional) - The model url
  \item
    authorId (optional) - The authorId of the model
  \item
    creationDate (optional) - The creation date of the model
  \item
    lastModificationDate (optional) - The last modification date of the
    model
  \item
    fileExtension (optional) - The file extension of the model
  \item
    relativePath (optional) - The relative path where the model is
    stored
  \item
    featureId (optional) - The featureId of the Aspect model (only valid
    for this model type)
  \item
    status (optional) - The status of the Base model (only valid for
    this model type)
  \end{itemize}
\item
  Response 200 (application/json)

\begin{verbatim}
[
    {
        "id":"2",
        "name":"MonitoringBaseModel",
        "authorId":"zavala",
        "creationDate":"2016-09-30 01:25:37.0",
        "lastModificationDate":"2016-09-30 01:25:37.0",
        "fileExtension":".uml",
        "systemId":"MonitoringReconfiguration",
        "featureId":"GooglePlay",
        "relativePath": "repository/models",
        "modelContent":"StringOfModelFileContent",
        "dependencies": [
            {
                "modelType": "PatternModel",
                "number": "1"
            }
        ]
    }
]
\end{verbatim}
\end{itemize}

\subsection{Model type
{[}/models/\{modelType\}{]}}\label{model-type-modelsmodeltype}

\subsubsection{Create new model instances of a given model type
{[}POST{]}}\label{create-new-model-instances-of-a-given-model-type-post}

\begin{itemize}
\item
  Request (application/json)

\begin{verbatim}
{
    "sender": "Adapter",
    "timeStamp": "2016-10-20T20:10:30:201",
    "modelInstances": [
        {
            "name": "googleplay_api_googleplay_tool",
            "authorId": "zavala",
            "creationDate": "2016-10-13 12:54:21.0",
            "lastModificationDate": "2016-10-20 16:22:01.0",
            "fileExtension": ".aspect",
            "systemId": "MonitoringReconfiguration",
            "featureId": "GooglePlay",
            "relativePath": "repository/models",
            "modelContent":"StringOfModelFileContent",
            "dependencies": [
                {
                    "modelType": "PatternModel",
                    "number": "1"
                }
            ]
        }
    ]
}
\end{verbatim}
\item
  Response 201 (application/json)

  \begin{verbatim}
  [
        {
            "id": "2"
            "name": "googleplay_api_googleplay_tool",
            "authorId": "zavala",
            "creationDate": "2016-10-13 12:54:21.0",
            "lastModificationDate": "2016-10-20 16:22:01.0",
            "fileExtension": ".aspect",
            "systemId": "MonitoringReconfiguration",
            "featureId": "GooglePlay",
            "relativePath": "repository/models",
            "modelContent":"StringOfModelFileContent",
            "dependencies": [
                {
                    "modelType": "PatternModel",
                    "number": "1"
                }
            ]
        }
    ]
  \end{verbatim}
\end{itemize}

\subsection{Model instance
{[}/models/\{modelType\}/\{modelId\}{]}}\label{model-instance-modelsmodeltypemodelid}

\subsubsection{Get a model instance by id and modelType
{[}GET{]}}\label{get-a-model-instance-by-id-and-modeltype-get}

\begin{itemize}
\item
  Response 200 (application/json)

\begin{verbatim}
{
    "id":"2",
    "name":"googleplay_api_googleplay_tool",
    "authorId":"zavala",
    "creationDate":"2016-09-30 01:25:37.0",
    "lastModificationDate":"2016-09-30 01:25:37.0",
    "fileExtension":".aspect",
    "systemId":"MonitoringReconfiguration",
    "featureId":"GooglePlay",
    "relativePath": "repository/models",
    "modelContent":"StringOfModelFileContent",
    "dependencies": [
        {
            "modelType": "PatternModel",
            "number": "1"
        }
    ]        
}
\end{verbatim}
\end{itemize}

\subsubsection{Update an existing model instance
{[}PUT{]}}\label{update-an-existing-model-instance-put}

\begin{itemize}
\item
  Request (application/json)

\begin{verbatim}
{
    "sender": "Adapter",
    "timeStamp": "2016-10-20T20:10:30:201",
    "values": {
        "authorId": "marc",
        "featureId": "GooglePlay_API",
        "dependencies": [
            {
                "modelType": "PatternModel",
                "number": "2"
            }
        ]  
    }
}
\end{verbatim}
\item
  Response 200 (application/json)

\begin{verbatim}
{
    "id":"2",
    "name":"googleplay_api_googleplay_tool",
    "url":"http://...",
    "authorId":"marc",
    "creationDate":"2016-09-30 01:25:37.0",
    "lastModificationDate":"2016-09-30 01:25:37.0",
    "fileExtension":".aspect",
    "systemId":"MonitoringReconfiguration",
    "featureId":"GooglePlay_API",
    "relativePath": "repository/models",
    "modelContent":"StringOfModelFileContent",
    "dependencies": [
        {
            "modelType": "PatternModel",
            "number": "2"
        }
    ]
}
\end{verbatim}
\end{itemize}

\subsubsection{Delete a model instance
{[}DELETE{]}}\label{delete-a-model-instance-delete}


%----------------------------------------------------------------------------------------
%	BIBLIOGRAPHY
%----------------------------------------------------------------------------------------

\setquotestyle{english}
\printbibliography[heading=bibintoc]

%----------------------------------------------------------------------------------------

\end{document}  
